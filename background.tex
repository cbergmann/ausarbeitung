%*******************************************
\section{Background}
%*******************************************
\label{s:background}
Introducing sentences...

%===========================================
\subsection{Definition of Phishing}
%===========================================
\label{s:phishing_def}
Our goal is to educate users to detect phishing websites. Since phishing is important in our work, we are going to define our understanding of the term. 

\begin{center}
\textit{``Definition of Phishing''}
\end{center}

The next section dwells on different phishing types. 


%-------------------------------------------
\subsection{Phishing Techniques}
%-------------------------------------------
\label{s:phishing_techs}
In this section we are going to describe the different phishing techniques that are distinguished in literature. Furthermore we state and reason which technique(s) of phishing we focus on in our work.. Phishing techniques include, but are not limited to:
%master_thesis/notes/phishing
\begin{description}[leftmargin=3cm]
	\item[Deceptive Phishing]
	\item[Malware Based Phishing] (including keyloggers and screenloggers)			
	\item[Host File Poisoning]
	\item[DNS Based Phishing] (Pharming)
	\item[Man-in-the-Middle Phishing]  	
 
\end{description}
For our research, we focus on deceptive phishing...
%(eventuell liste oder aufzählung)

 
%EXAMPLE TABLE WHICH MIGHT BE USEFUL :D
%\begin{table}
%	\centering
%	\begin{tabularx}{.9\textwidth}{m{2.6cm} m{3.8cm} m{4.0cm} m{4.12cm}}
%	\hline	
%	\rowcolor{rowColorHead}
%										& Spalte 1 												& Spalte 2 			& Spalte 3\\
%	\hline
%	\rowcolor{rowColor1}
%	Zeile 1 					& Inhalte, \newline Inhalte			&	Inhalt			 		&	Inhalt \\		
%	\rowcolor{rowColor2}
%	Zeile 2 			& Inhalt, \newline Inhalt			&	Inhalt					&	Inhalt, \newline Inhalt	\\	
%	\hline
%	\end{tabularx}
%	\caption{Description}
%	\label{table:label}
%\end{table}



%-------------------------------------------
\subsection{Phishing Attack Channels}
%-------------------------------------------
\label{s:attack_channels}
\begin{description}[leftmargin=3cm]
	\item[E-Mail]
	\item[SMS]  			
	\item[Instant Messaging]
	\item[Online Social Networks]
	\item[Fake Website]
	\item[VoIP]
	\item[Malicious Downloads]
\end{description}

We focus on fake websites. Usually, the links to fake websites are distributed via e-mails, SMS, instant messengers or online social networks, Thus, our approach automatically covers the attack channels e-mail, sms, instant messaging and online social networks.

%-------------------------------------------
\subsection{Variations of Phishing}
%-------------------------------------------
\label{s:phishing_variations}
Do we need this subsection?
\begin{description}[leftmargin=3cm]
	\item[Mass Phishing]
	\item[Spear Phishing]  
	\item[Persistent Spear Phishing]
	\item[Clone Phishing]
	\item[Whaling]
\end{description}

We cover in particular mass phishing. However, the URL checking can be applied in case of any variant, as long as the attack is executed via a fake website.
%===========================================
\subsection{Scope}
%===========================================
\label{s:scope}

%eine grafik wäre nice.. die zusammenfasst, was für eine Art von Phishing wir hier betrachten