%*******************************************
\section{Teaching and Learning Content}
%*******************************************
In this section we will describe and elaborate on different teaching and learning contents which can potentially be communicated to the user. At the same time we will reason our decision whether to communicate the specific content or not.
%Documents master_thesis/notes/android_browser bla -> BEGRÜNDUNG WARUM manches nicht sinnvoll ist (diese Sachen vielleicht eher in Appendix vor allem versionsunterschiede)
%Documents master_thesis/konzepte/android browser elemente UND browser comparison
%CHECK IF I FORGOT SOMETHING!!!!
%===========================================
\subsection{Phishing URLs}
%===========================================
Focus on distinguishing phishing URLs from legitimate ones.

%...........................................
\subsubsection{Phishing URL Categorization}
%...........................................

Potential phishing URL categories/phishing attcks on URLs
\begin{description}
	\item[Subdomain] covered
	\item[IP Address] covered
	\item[Nonsense Domain] covered
	\item[Trustworthy, But Unrelated Domain] covered
	\item[Similar and Deceptive Domains] covered Typo, Typosquatting (Buchstabendreher), Misspelling
	\item[Homographic Attack] covered (the type of homographic visible by user...)
	\item[Tiny URLs] Not covered
	\item[Cloaked URLs] Not covered - because redirect (use of @)
	\item[Encoding Tricks] Not covered - because redirect
\end{description}

%...........................................
\subsubsection{Problems and Challenges With The Categorization}
%...........................................

%...........................................
\subsection{Android Elements}
%...........................................
Eventuell Titel umbenennen, anders strukturieren...
\begin{description}
	\item[Invisible Address Bar] Find URL Bar, Browser
	\item[Use of Https Within Websites] Browser
	\item[Analyze Complete URL Via Address Bar] Browser
	\item[Show URL Before Click] In E-Mail (not always possible), while surfing (long touch)
	\item[Copy and Paste URL] too much effort, additionally: redirects still possible
\end{description}

%...........................................
\subsection{Android Browser Security Indicators}
%...........................................

\begin{description}
		\item[Https Padlock] Browser
		\item[Displayed Webaddress on Https Sites] Browser
		\item[Certificate Verification]
		\item[Touch Padlock] to see whole URL.. problems: see document...
\end{description}

%...........................................
\subsection{E-Mail Spoofing}
%...........................................

\begin{description}
	\item{From Field} not trustworthy
	\item{E-Mail Content} in hand of attacker
	\item{Links in E-Mails} do not necessarily go where it claims to go (not only in e-mail links).
\end{description}

\subsection{General Recommended Behavior}
\begin{description}
	\item[Do Not Click]
	\item[Do Not Download Attachment]
	\item[Look at URL]
	\item[Data Economy]
	\item[Date Entry Via Https]
\end{description}

%...........................................
\subsection{Conclusion / Summary}
%...........................................

Summarize what to communicate to user here...

