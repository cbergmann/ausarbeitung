
%===========================================
\section{Development}
%===========================================
This chapter deals with the development process of our app.
We do not provide in-depth insight to our source code. 
Instead we give a brief overview of our approach for the development of a user friendly and understandable app.
%===========================================
\subsection{Mock Up}
%===========================================
After we have decided about the work flow and structure of our app we built a mock up in order to get a more concrete idea of what needs to be implemented and to reveal flaws in our thought process.
Also, we showed it to a couple of friends and relatives so we could expose aspects we have not yet thought about.
All in all, the work flow and structure of the mock up was quite understandable.
However, the first texts explaining how to access the address bar and about the structure of a URL seemed to be incomprehensible.
As a consequence, we adjusted these texts in the app (only thos of the first three levels) and showed them to other friends and relatives who seemed to understand the descriptions.
Based on these initial texts we wrote all remaining texts without including it into the app yet.
The next section deals with the elaboration of these texts.
%===========================================
\subsection{App Texts}
%===========================================

%===========================================
\subsection{Pilot Study}
%===========================================
%===========================================
\subsection{URL Generation}
%===========================================
While playing the app the user is presented with URLs that he has to categorize as phish or valid.
While reviewing the previous works and games in this area we found that many of them use a fixed set of examples.
On some games this set is very small and therefore you are always confronted with the same URLs.
As we layed out in Section \ref{s:url_structure} we want to teach the user how to detect phishing URLs in general.
To accomplish this goal we think that it is essential that the user sees as much different URLs as possible so he can build his own mental model.
Therefore we decided on generating URLs rather than composing a fixed list.
We will lay out the general process here and cover interesting parts of it in the following sections.
\subsubsection{Example URLs}
To present attacked URLs to the user we found int most realistic to take valid URLs and apply attacks on them.
Therefore we needed a set of valid URLs.
To build this set we used Alexa\ref{alexa} to find the top 100 domains for german users.
We then went to each of these sites and by navigating tried to find 6 URLs for each domain.
We tried to find some short and some long URLs.
\begin{description}
\item[generate attacks for level]When starting a new level we generate a list of Attacks that we want to show the user.
\item[select valid URL]When we want to show a new URL to the user we first randomly select a valid URL from the before mentioned set.
\item[apply generator]Then we apply a generator to the URL that does not invalidate the URL but modifies it.
\item[apply attack]After that we select a random attack from the previously build list and apply it to the URL.
\item[repeat]In some situations we need to try again.
\end{description}
\subsubsection{generate attacks for level}
\textbf{Was zur historie?}
The types of URLs the user is presented is dependent on the level.
Each level introduces on or more attacks.
Which attack is introduced in which level is layed out in section\ref{s:knowledgetransferperlevel}.
In general the URLs of each level $n$ are distributed as follows:
\begin{table}[hHtbp]
\centering
\begin{tabular}{llll}
Total number of URLs&$u$&$6+2*n$&starting with 6 URLs each level has 2 more URLS.\\
Number of Phishes&$p$&$u/2$&Half of the URLs are phishes.\\
Number of repeats&$r$&$\left\lfloor p/2 \right\rfloor$&Half of the phishes are repeats.
\end{tabular}
\caption{distribution of URLs per level.}
\label{t:levelurls}
\end{table}

The repeats are always one attack from each previous level. The rest of the repeats is filled up randomly.

There are two main exception to these rules:
\begin{description}
\item[Level 1] In Level 1 the game is modified in the form that the user is only presented with valid URLs and has to select the domain.
To prevent boring the user in this level we only present 5 URLs.
None of them is a phish.
\item[Level 1+2] The first level that contain repeats is level 3 because level 2 is the first real game level.
\end{description}

The generated list of attacks also contains a special attack that does no real attack. This is to simplify the URL generation. When we generated the list of attacks we save it for later reference.
\subsubsection{apply generator}
We were unsure if we still have enough valid URLs so we prepared a way to automatically modify the URLs in such a way that they could still be valid URLs. Some Ideas where to add subdomains or path strings to the URL. Query or fragments are also possible. We later found out that it is currently not needed to implement generators because there are a lot of URLs in our set. If however we will some time in the future find out that these URLs are not enough we have this scheme in place.
\subsubsection{apply attack}
After we generated a valid URL we chose a random attack from the previously build set of attacks and apply it to the URL. With this we also store which attack we currently applied. This is important when the user is failing this round. In this situation we will simply readd this attack to the set of attacks.
\subsubsection{repeat}
There are combinations of base-URL and attack where the attack doe's not alter the URL.
Therefore it would be impossible for the user to detect the Attack and he will be confused and might stop using the app.
In this situation we repeat the whole URL generation process until we find a matching URL.