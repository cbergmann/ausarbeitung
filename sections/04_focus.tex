%*******************************************
\section{Focus}
%*******************************************
\label{s:assumptions}
Introductory sentences.
..
Based on the discussion of the previous section we decided to.
..... (not only based on previous section)

%===========================================
\subsection{Coverage}
%===========================================
\begin{description}[leftmargin=0cm]
	\item[Deceptive Phishing as Phishing Technique] Within the scope of this work we focus on deceptive phishing.
 In particular, we target the detection of phishing websites resp. phishing URLs.
 
	\item[Several Attack Channels] We focus on the detection of spoofed websites resp. phishing URLs.
 Phishing websites can be reached in several ways.
 Links to fake websites are usually distributed via e-mails, instant messages or online social networks.
 However, they can also be spread via SMS or even phone calls.
 Ultimately, a phishing website can also be reached by just surfing in the Internet.
 As a consequence, our approach covers all attack channels, as long as the user is tricked to divulging sensitive information via a phishing website.

	\item[Mass Phishing as Variation of Phishing] We cover in particular mass phishing.
 However, the URL checking can be applied in case of any variant, as long as the attack includes a website which lures the user to type in his credentials.

	\item[Game and Quiz Based Learning as Communication Medium] \textbf{wenn oben geschrieben, hier auch schreiben}		
	\item[URL Based Knowledge as Learning Content] The advantages of telling the user what to pay attention to within e-mails are the following: if the user recognizes the phishing e-mail before clicking on a link he does not even get onto a fake website where he could be lured to divulge his credentials.
 This also would mean, that the user would not be forwarded to a page where a malicious download might be started.
 On the other hand, these fake e-mails become more and more sophisticated and thus it becomes harder to distinguish them from legitimate ones~\cite{microsoftphishing, spamfighter}. Additionally, e-mail is not the only attack channel where links to phishing websites can be distributed.
  Those links are also spread via instant messaging systems, online social networks, or SMS, where the messages would differ from those in e-mails.
 Moreover, phishing websites can also be reached by surfing~\cite{kasperskyreport2013}, where the e-mail based knowledge approach would completely fail.
 For these reasons we decided to focus on communicating URL based knowledge to the user.
 This way, the disadvantages of e-mail based knowledge are mitigated.
 Furthermore, we believe that URL based knowledge gives the most reliable hint regarding its "belonging", i.e. whether a URL in fact belongs to a legitimate website or not.

	\item["After Click" URL Analysis] We have decided to consider the "after click" scenario for the following reasons: Firstly, we cannot hinder users from clicking on links and make them type in the whole URL into the address bar.
 This is too effortful, especially on smartphones, and thus will not be followed by them.
 Secondly, many links contain redirects.
 Such redirects are not recognizable before the click.
 A further problem the "before click" scenario raises is that the stock e-mail client of Android does not provide the functionality of viewing the destination URL before clicking on it.
 The only way to have a look at the URL before clicking on it is to make a long press onto the link, copy it into the clipboard, paste it somewhere else and then analyze it.
 Then, after the analysis the URL has to be sent to the browser.
 However, as this is also involves too much effort, no user will follow such a suggestion.
 Finally, even if there are many other e-mail clients which offer viewing the destination URL via long press only, we believe that this should not be communicated to the user for two reasons.
 Firstly, we do not know how many Android users actually make use of the stock e-mail client, which does not offer this functionality.
 Secondly, and most importantly, this functionality has the potential to mislead the user.
 A drawback of the URL destination preview is that the end of it is cut in case the URL is too long.
 Well-crafted URLs might thus look legitimate even though they are not because the most important part of the URL was cut out.
 For example, the subdomains of the URL can be long and well-crafted so that a legitimate looking subdomain is exactly at the end of the preview.
 This will cause the user think, that the subdomain at the end of the preview is the domain of the URL.
 Ultimately, the user will trust this URL even he should not.
 For the reasons explained above we have decided to consider the "after click" scenario.
 This approach suffers the disadvantage that users might click on a link which has a download of malware behind it.
 Also the visit of the linked phishing website might show the phisher that the used Mail-Address is valid and active and this can result in further attacks on this Mail address. Also the mere request and display of the phishing webpage might provide additional information to the phisher or even expose the user to attacks.
  For now, we consider this as future work, as there is no possibility to detect the real target of a link before clicking it.

%However, this drawback is mitigated by the fact that such downloaded malicious software is %not harmful as long as it is not opened and installed. Thus, the user should be told that %downloads he did not intend should immediately be deleted and should not be opened %under any circumstances.

% SCHÄDLICHE SW KANN SICH AUCH AUTOMATISCH AUSFÜRHEN / ÖFFNEN VON PDF IN BROWSER IST DANN AUCH SCHON GEFÄHRLICH
%Add to future work.

%Falls unverständlich vielleicht ein screenshot mit einem Beispiel ;) das mit well-crafted url
	\item[Considered Browser] für screenshots benutzen wir Android standard browser, aber kann auf jeden browser übertragen werden.
 wir haben überlegt, sahcen wie lock icon Blabla einzubauen, aber da immer sehr unterschiedlich haben wir uns dagegen entschieden.
 um unsere Methode möglichst allg.
 zu halten-- siehe bitte section blablabla

\end{description}	

%===========================================
\subsection{System Requirements}
%===========================================
In the following we are listing the system requirements which need to be met for the final app.


\begin{description}[leftmargin=0cm]
	\item[Android] Today the mobile phone market is split between two major competitors. Android(81.0\%) and iOS(12.9\%). (Cite: http://www.idc.com/getdoc.jsp?containerId=prUS24442013)
	We have decided to develop an app for the Android operation system as there are more users and we believe we have greater freedom here compared to an iOS app. 
 The publication of an iOS app, for example, is connected with more requirements, which is not the case for Android apps~\cite{publishios, publishandroid}. This allows us to implement and test an android app more quickly and cost effective then on iOS. 
	\item[Version] Our original intention was to develop an Android app for version 4.0 and upward.
 However, during the app development we have encountered that about 24\% of all Android users still have Android 2.3.3 to 2.3.7~\cite{}. For this reason we have decided to modify the code so that these users can also install and use our app.
 
%	\item[Android Standard Browser] Android standard browser is kein system requirement - raus. muss irgendwoanders erwähnt werden. (was betrachten wir beim erklären oder so)...
\end{description}

%===========================================
\subsection{Assumptions}
%===========================================
We have to make some assumptions about the system the user uses. If one or more of these are not met the user, disregarding of his skill, is not able to detect when he is target of an attack.
\begin{description}
	\item[Secure DNS] We have to be sure that DNS is not under the control of the attacker.
	Our approach is to show the user how to identify phishing by analysing the URL of the shown page.
	This is of no use if the Phisher can control the DNS system of the user.
	This includes local host files and all used DNS servers.
	\item[Secure Smartphone] Also we have to assume that the system of the user is in a secure state.
	This means that the attacker is not able to e.g. replace the browser or read the users input.
	\item[Secure SSL] 
	For a Man-in-the-middle it is possible to interfere with the send or received data and to collect the user data directly.
	Therefore we warn the user that he should not enter personal data in non-HTTPS pages.
	But even in HTTPS environments we can not be sure of the servers identity when SSL is not secure.
	We are aware that recent events show that there are a multitude of SSL providers which fail to ensure that certificates are only handed to legitimate users or do this on purpose.
	When SSL is not secure the user has no practical possiblity to detect a Man-in-the-middle attack.		
\end{description}

%===========================================
\subsection{Limitations of Our Approach}
%===========================================
In addition to the general assumptions we make because of the total inability of the user to detect phishing if one of these assumptions are not met.
There are also limitations of our approach that result in the target group for our app.
The targeted user is not a computer expert and has no time nor interest to analyze the shown webpage throughly before entering data.
Therefore we don't tell the user about possible attacks that an experienced user might find.
\begin{description}
	\item[Cross-Site Scripting]
	Cross-Site Scripting (XSS) is an a attack where the attacker enters code into a legitimate webpage.
	For a later viewer of this page this form seems to be legitimate content of the attacked webpage and he might be lured to enter personal data in this area. 
	Depending on attacked webpage this can not be detected by the user.
	This is a vulnerability of the attacked webpage and can be prevented by checking user input.
	Therefore we think that preventing this attack is in the responsibility of the website owner.
	\item[URL Hiding Techniques]
	Most modern mobile phones have small screens.
	Therefore most browser hide away the URL-bar when the user views a webpage to gain additional screen real estate.
	The URL-bar is shown only when the user scrolls up beyond the top of the page.
	There is a possible attack where the attacker prevents the user from scrolling all the way up and instead displays a fake URL-bar with a fake URL.
	A problem that the attacker must solve is that mobile browser look very diffently and this must be reflected by the fake URL-bar.
	This problem is not easy to solve and it is today not needed for the attacker to do such sophisticated attacks because enough users fall for the simple attacks. 
	We think this is the reason that this attack is not yet seen in the wild.
	Therefore we don't consider this attack. 
\end{description}

