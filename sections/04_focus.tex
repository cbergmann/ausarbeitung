%*******************************************
\section{Scope of Our Approach}
%*******************************************
\label{s:assumptions}
This chapter elaborates on the determination of our scope to educate people about phishing. 
Here, we gather all of our choices and the corresponding reasoning for our defintion of our scope. 
These include the various classes of phishing and phishing learning techniques that we mentioned in \autoref{s:background} as well as new aspects.
Subsequently, we summarize the system requirements and what assumptions we had to make. 
Finally, the limitations of our work are stated.

%===========================================
\subsection{Coverage}
%===========================================
\label{s:coverage}
\begin{description}[leftmargin=0cm]
	\item[Deceptive Phishing as Phishing Technique] Within the scope of this work we focus on deceptive phishing.
 In particular, we target the detection of phishing websites resp. phishing URLs.
 
	\item[Several Attack Channels] As mentioned above, we concentrate on the detection of phishing URLs.
 Phishing websites can be reached in several ways.
 Links to fake websites are usually distributed via e-mails, instant messages, or online social networks.
 Moreover, they can be spread via SMS or even phone calls.
 Ultimately, a phishing website can also be reached by just surfing in the Internet.
 By teaching on identifying spoofed URLs, our approach covers all attack channels as long as the user is tricked into divulging sensitive information on a phishing website.

	\item[Mass Phishing as Variation of Phishing] We cover mass phishing, as already stated in~Section~\autoref{s:phishing_variations}.
 However, the URL checking can be applied in case of any variant, as long as the attack includes a website which lures the user to type in his credentials.

	\item[Game and Quiz Based Learning as Communication Medium] As discussed in \autoref{s:medium_classification}, we have decided to develop a quiz game to create an incentive for the users and at the same time reach a large audience. 

	\item[URL Based Knowledge as Learning Content] As argued in \autoref{s:content_classification} we decided to educate the users about phishing based on URLs. 
We believe that URL based knowledge gives the most reliable hint regarding its ``origin'', i.e. whether a URL in fact belongs to a legitimate website or not.

	\item[After Click URL Analysis] The analysis of a URL can follow before or after clicking on a link (if a link is involved), i.e. with a URL preview option or directly in the address bar. 
Analyzing the URL before clicking on it brings several benefits:

\begin{enumerate}
	\item \textbf{No Malicious Download} If a spoofed URL is detected before clicking on a link the potential download of  malicious software can be avoided. 
	\item \textbf{No Phisher Obtains No Information} Recognizing the spoofed URL before visiting the website prevents the phisher from obtaining any information of the user. Such information include, for example, the activeness and validity of the user's e-mail address. 
\end{enumerate}

On the other hand, the before click scenario has severe downsides:

\begin{enumerate}
	\item \textbf{Redirects Not Recognizable} Many links contain redirects. Such redirects, which can be malicious, are not recognizable before the click.
	\item \textbf{Unavailability of URL Preview Functionality} The stock e-mail client of Android does not provide the functionality of previewing the destination URL. 
Here, the only way to preview the URL to apply a long press to the link, copy it into the clipboard, paste it somewhere else and then view it. 
Then, after the analysis the URL has to be sent to the browser.
As this is involves too much effort, no user will follow such a suggestion.
	\item \textbf{Deception With URL Preview} There are other e-mail clients which provide the option of previewing the destination URL of a link.
Yet, we believe that this should not be communicated to the user for two reasons.
First, we do not know how many Android users actually make use of the stock e-mail client, which does not offer the preview option.
Second, and most importantly, this functionality has the potential to mislead the user.
A severe downside of the URL preview is that the end of the preview is cut in case the URL is too long.
 Well-crafted URLs might thus look legitimate even though they are not because the most important part of the URL, i.e. the actual domain, was cut out.
 For example, the subdomains of the URL can be long and well-crafted so that a legitimate looking subdomain is exactly at the end of the preview.
 This will result in the user thinking that the subdomain at the end of the preview is the domain of the URL.
 Ultimately, the user will trust this URL even he should not.
	\item \textbf{Users Like Clicking} Clicking on links is practical and convenient. As a matter of fact, users like clicking on links. Hence, we cannot hinder users from clicking on links.
\end{enumerate}

The after click scenario does not exhibit all these drawbacks which is why we chose to follow this approach.
Still, this scenario might suffer from potential malicious downloads and providing information to the phisher (before click benefits).
If a user confirms his e-mail address to the phisher by clicking on a link, further attacks towards this e-mail address are likely to follow.
Also, the pure request and displaying of a phishing website might provide additional information to the phisher or even expose the user to attacks.
For now, we consider this as future work, as there is no possibility to detect the real target of a link before clicking it.

	\item[Considered Browser] As a matter of fact, the user is only communicated general browser skills which can be transferred to any other browser.
Nevertheless, when the user gets browser screenshots, for example, we made use of the Android standard browser to be sure that most users are familiar with the pictures they are shown.
\end{description}	

%===========================================
\subsection{System Requirements}
%===========================================
In the following we are listing the system requirements which need to be met for the final app.


\begin{description}[leftmargin=0cm]
	\item[Android] Today the mobile phone market is split between two major competitors. Android~(81.0\%) and iOS~(12.9\%)~\cite{androidiosmarketshare}. 
	We have decided to develop an app for the Android operation system as there are more users and we believe we have greater freedom here compared to an iOS app. 
 The publication of an iOS app, for example, is connected with more requirements, which is not the case for Android apps~\cite{publishios, publishandroid}. This allows us to implement and test an android app more quickly and cost effective then on iOS. 
	\item[Version] Our original intention was to develop an Android app for version 4.0 and upward.
 However, during the app development we have encountered that about 24\% of all Android users still have Android 2.3.3 to 2.3.7~\cite{versionsandroid}. For this reason we have decided to modify the code so that these users can also install and use our app.
 
%	\item[Android Standard Browser] Android standard browser is kein system requirement - raus. muss irgendwoanders erwähnt werden. (was betrachten wir beim erklären oder so)...
\end{description}

%===========================================
\subsection{Assumptions}
%===========================================
We have to make some assumptions about the system the user uses. If one or more of these are not met the user, disregarding of his skill, is not able to detect when he is target of an attack.
\begin{description}[leftmargin=0cm]
	\item[Secure DNS] We have to be sure that DNS is not under the control of the attacker.
	Our approach is to show the user how to identify phishing by analysing the URL of the shown page.
	This is of no use if the Phisher can control the DNS system of the user.
	This includes local host files and all used DNS servers.
	\item[Secure Smartphone] Also we have to assume that the system of the user is in a secure state.
	This means that the attacker is not able to e.g. replace the browser or read the users input.
	\item[Secure SSL] 
	For a Man-in-the-middle it is possible to interfere with the send or received data and to collect the user data directly.
	Therefore we warn the user that he should not enter personal data in non-HTTPS pages.
	But even in HTTPS environments we can not be sure of the servers identity when SSL is not secure.
	We are aware that recent events show that there are a multitude of SSL providers which fail to ensure that certificates are only handed to legitimate users or do this on purpose.
	When SSL is not secure the user has no practical possiblity to detect a Man-in-the-middle attack.		
\end{description}

%===========================================
\subsection{Limitations of Our Approach}
%===========================================
In addition to the general assumptions we make because of the total inability of the user to detect phishing if one of these assumptions are not met.
There are also limitations of our approach that result in the target group for our app.
The targeted user is not a computer expert and has no time nor interest to analyze the shown webpage throughly before entering data.
Therefore we don't tell the user about possible attacks that an experienced user might find.
\begin{description}[leftmargin=0cm]
	\item[Cross-Site Scripting]
	Cross-Site Scripting (XSS) is an a attack where the attacker enters code into a legitimate webpage.
	For a later viewer of this page this form seems to be legitimate content of the attacked webpage and he might be lured to enter personal data in this area. 
	Depending on attacked webpage this can not be detected by the user.
	This is a vulnerability of the attacked webpage and can be prevented by checking user input.
	Therefore we think that preventing this attack is in the responsibility of the website owner.
	\item[URL Hiding Techniques]
	Most modern mobile phones have small screens.
	Therefore most browser hide away the URL-bar when the user views a webpage to gain additional screen real estate.
	The URL-bar is shown only when the user scrolls up beyond the top of the page.
	There is a possible attack where the attacker prevents the user from scrolling all the way up and instead displays a fake URL-bar with a fake URL.
	A problem that the attacker must solve is that mobile browser look very diffently and this must be reflected by the fake URL-bar.
	This problem is not easy to solve and it is today not needed for the attacker to do such sophisticated attacks because enough users fall for the simple attacks. 
	We think this is the reason that this attack is not yet seen in the wild.
	Therefore we don't consider this attack. 
\end{description}

