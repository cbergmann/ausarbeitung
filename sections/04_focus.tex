%*******************************************
\section{Scope of Our Approach}
%*******************************************
\label{s:assumptions}
This chapter elaborates on the determination of our scope to educate people about phishing. 
Here, we gather all of our choices and the corresponding reasoning for our defintion of our scope. 
These include the various classes of phishing and phishing learning techniques that we mentioned in \autoref{s:background} as well as new aspects.
Subsequently, we summarize the system requirements and what assumptions we had to make. 
Finally, the limitations of our work are stated.

%===========================================
\subsection{Coverage}
%===========================================
\label{s:coverage}
\begin{description}[leftmargin=0cm]
	\item[Deceptive Phishing as Phishing Technique] Within the scope of this work we focus on deceptive phishing.
 In particular, we target the detection of phishing websites resp. phishing URLs.
	\item[Several Attack Channels] As mentioned above, we concentrate on the detection of phishing URLs.
 Phishing websites can be reached in several ways.
 Links to fake websites are usually distributed via e-mails, instant messages, or online social networks.
 Moreover, they can be spread via SMS or even phone calls.
 Ultimately, a phishing website can also be reached by just surfing in the Internet.
 By teaching on identifying spoofed URLs, our approach covers all attack channels as long as the user is tricked into divulging sensitive information on a phishing website.

	\item[Mass Phishing as Variation of Phishing] We cover mass phishing, as already stated in \autoref{s:phishing_variations}.
 However, the URL checking can be applied in case of any variant, as long as the attack includes a website which lures the user to type in his credentials.

	\item[Game and Quiz Based Learning as Communication Medium] As discussed in \autoref{s:medium_classification}, we have decided to develop a quiz game to create an incentive for the users and at the same time reach a large audience. 

	\item[URL Based Knowledge as Learning Content] As argued in \autoref{s:content_classification} we decided to educate the users about phishing based on URLs. 
We believe that URL based knowledge gives the most reliable hint regarding its ``origin'', i.e. whether a URL in fact belongs to a legitimate website or not.
Additional we had a look at the phishing URLs provided by PhishTank~\cite{phishtank}. The majority of these URLs were not or only loosely related to the attacked website. If the users would be aware of the importance of the URL and were able to interpret it the phishers would put more effort in forging valid-looking URLs. Obviously, there are enough users falling for these primitive attacks. Therefore, we think that it is important to inform the users about the significance of URLs and to teach them how to interpret those.

	\item[After Click URL Analysis] The analysis of a URL can follow before or after clicking on a link (if a link is involved), i.e. with a URL preview option or directly in the address bar. 
Analyzing the URL before clicking on it brings several benefits:

\begin{enumerate}
	\item \textit{No Malicious Download} If a spoofed URL is detected before clicking on a link the potential download of  malicious software can be avoided. 
	\item \textit{Phisher Obtains No Information} Recognizing the spoofed URL before visiting the website prevents the phisher from obtaining any information of the user. Such information include, for example, the activeness and validity of the user's e-mail address. 
\end{enumerate}

On the other hand, the before click scenario has severe downsides:

\begin{enumerate}
	\item \textit{Redirects Not Recognizable} Many links contain redirects. Such redirects, which can be malicious, are not recognizable before the click.
	\item \textit{Unavailability of URL Preview Functionality} The stock e-mail client of Android does not provide the functionality of previewing the destination URL. 
Here, the only way to preview the URL to apply a long press to the link, copy it into the clipboard, paste it somewhere else and then view it. 
Then, after the analysis the URL has to be sent to the browser.
As this is involves too much effort, it is likely that no user will follow such a suggestion.
	\item \textit{Deception With URL Preview} There are other e-mail clients which provide options to display the destination URL of a link.
Yet, we believe that this should not be communicated to the user for two reasons.
First, we are elaborating on a general approach that does not rely on third party e-mail clients besides the default one, which is pre-installed and comes with the device itself.
Second, and most importantly, this functionality has the potential to mislead the user.
A severe downside of the URL preview is that the end of the preview is cut in case the URL is too long.
 Well-crafted URLs might thus look legitimate even though they are not because the most important part of the URL, i.e. the actual domain, was cut out.
 For example, the subdomains of the URL can be long and well-crafted so that a legitimate looking subdomain is exactly at the end of the preview.
 This will cause the user to think that the subdomain at the end of the preview is the domain of the URL.
 Ultimately, the user will trust this URL even he should not.
	\item \textit{Users Like Clicking} Clicking on links is practical and convenient. As a matter of fact, users like clicking on links. Hence, we cannot hinder users from clicking on links.
\end{enumerate}

The after click scenario does not exhibit all these drawbacks which is why we chose to follow this approach.
On the other hand, this scenario might suffer from potential malicious downloads and providing information to the phisher (benefits of before click scenario).
If a user confirms his e-mail address to the phisher by clicking on a link, further attacks towards this e-mail address are likely to follow.
Also, the pure request and displaying of a phishing website might provide additional information to the phisher or even expose the user to attacks.
For now, we consider this as future work, as there is no possibility in our target scenario to detect the real target of a link before clicking on it.

	\item[Considered Browser] As a matter of fact, the user is only taught general browser skills, which can be transferred to any other browser.
Nevertheless, when the user gets browser screenshots, for example, we made use of the Android standard browser to be sure that most users are familiar with the pictures they are shown.
\end{description}	

%===========================================
\subsection{System Requirements}
%===========================================
In the following we list the system requirements which need to be met to install and play with the final app.


\begin{description}[leftmargin=0cm]
	\item[Android] Currently the mobile phone market is split between two major competitors. Android~(81.0\%) and iOS~(12.9\%)~\cite{androidiosmarketshare}. 
	We have decided to develop an app for the Android operation system as there are more users and we believe we have greater freedom here compared to an iOS app. 
 	Additionally, Android is an open operating system and imposes less requirements and barriers that allow us a quick development and publication of our app~\cite{publishios, publishandroid}. 
	\item[Version] Our initial intention was to develop an Android app for version 4.0 and upward.
 However, during the app development we have encountered that about 24\% of all Android users still have Android 2.3.3 to 2.3.7~\cite{versionsandroid}. For this reason we have decided to modify the code so that these users can also install and use our app.
	\item[Samsung Galaxy S3 or S4] Actually, it is not necessarily required to install the app on Samsung Galaxy S3 or S4 devices. 
Yet, we did not have the possibility to test our app (design and functionality) on different devices.
The above mentioned devices are those we tested and used for our final user study.
\end{description}

%===========================================
\subsection{Assumptions}
%===========================================
We have to make some assumptions about the user's system. 
If one or more of these are not met the user, disregarding of his skills, might not able to detect when he is a target of an attack.
\begin{description}[leftmargin=0cm]
	\item[Secure DNS] We have to assume that DNS is not under the control of the attacker.
	Our approach is to show the user how to identify phishing attempts by analyzing the URL of the shown page.
	In fact, this is of no use if the phisher can control the DNS system of the user.
	Therefore, we assume local host files and all used DNS servers as untouched by the attacker.
	\item[Secure Smartphone] We imply that the user's system is in a secure state.
	This means that the attacker is not able to, for example, exploit browser vulnerabilities, replace the browser or read the user's input.
	\item[Secure SSL] 
	For a man-in-the-middle it is possible to intercept sent or received data and to collect the user data directly without notice.
	Therefore, we warn the user against entering personal data on non-HTTPS pages.
	But even in HTTPS environments we can not be sure of the server's identity if SSL is not secure.
	We are aware of events showing that there are a multitude of SSL providers which, intentionally or not, fail to award certificates to legitimate users only.
	When the certificates cannot be trusted, SSL cannot be considered as secure and, ultimately, the user has no practical possiblity to detect a man-in-the-middle attack.	
	\item[Malware] 
	Sending e-mails with malicious attachments or links to malicious downloads are a form of deceptive phishing.
	In our approach we assume that the attacker does not lure the user into downloading such malicious software, which captures the user's confidential data.
	We focus on attacks where the user is actively encouraged to provide his sensitive data himself.
	Yet, we believe that preventing users from being lured into downloading malware is an important aspect.
	Therefore, in future work one should consider how our approach can be expanded with regard to this problem.
	
\end{description}

%===========================================
\subsection{Limitations of Our Approach}
%===========================================
In addition to the general assumptions pointed out in the previous section, there are limitations of our approach resulting from the chosen target group for our app.
As described in \autoref{s:target_group}, users from our target audience are no computer experts and have neither time nor are they willing to analyze the shown website thoroughly before entering data.
Therefore, we do not intend to tell the users about possible attacks that only experienced user might find.
\begin{description}[leftmargin=0cm]
	\item[Cross-Site Scripting]
	Cross-Site Scripting (XSS) is an attack where the attacker enters code, such as a form, into a legitimate webpage.
	For a later viewer of this page this form seems to be legitimate content of the attacked webpage and he might be lured to enter personal data in this area. 
	Depending on the attacked webpage this cannot be detected by the user.
	This is a vulnerability of the attacked webpage and can be prevented by checking user input.
	Therefore, we think that preventing this attack is in the responsibility of the website owner.
	\item[URL Hiding Techniques]
	Most modern mobile phones have small screens.
	Therefore, most browser hide away the URL bar to increase the viewport when the user browses webpages.
	The URL bar is shown only when the user scrolls up beyond the top of the webpage.
	There is a possible attack where the attacker prevents the user from scrolling all the way up and instead displays a fake URL bar with a fake URL.
	A problem that the attacker faces is that mobile browsers look very diffently and this must be reflected by the fake URL bar - a tedious and difficult task.
	Moreover, attackers do not need such sophisticated attacks because enough users fall for the simple attacks these days. 
	We think this is the reason why this kind of attack has not yet been observed in the wild and the reason why we do not consider it as well.
\end{description}

