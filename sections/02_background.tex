%*******************************************
\section{Background}
%*******************************************
\label{s:background}
The objective of this chapter is to provide the required background knowledge for our further design elaborations. 
We split this chapter into two parts.
The first part deals with the term phishing in general which includes common phishing techniques, attack channels, and variations of phishing.
In the second part we introduce different phishing learning techniques.
For better readability and comprehensibility we divided the available learning techniques into their content, i.e. what specific content is the user told, and the used media, i.e. how is this content communicated to the user.


%===========================================
\subsection{Phishing in General}
%===========================================
\label{s:phishing_general}
This section elaborates on the topic of phishing in general.
Phishing is a term which is referred to for various scenarios and techniques.
Consequently, there are different definitions of phishing found in literature.
Therefore, we start with a definition that entails all types of phishing.
Subsequently, we introduce different phishing techniques, used attack channels and variations of phishing.
Finally, we state our scope with respect to the term of phishing and provide our own definition of it which we consider in this work.
%............................................................................................................
\subsubsection{Abstract Definition of Phishing}
%............................................................................................................
\label{s:phishing_def}
The goal of this work is to help users distinguish phishing websites from legitimate ones. 
 Since phishing is important within the scope of this work, we define the term first. In fact, phishing is a term that is used by many people in different contexts. Therefore, the following definition is deliberately kept abstract in order to cover all possible scenarios of phishing. At the end of this chapter we will state our definition of phishing which we consider in this work.

\begin{center}
\textit{``Phishing is the practice of obtaining confidential information from users and describes a form of identitfy theft.
 Targeted confidential information includes, but is not limited to, user names, passwords, social security numbers, credit card numbers, or account information.
''}~\cite{jakobsson2006phishing}
\end{center}

%............................................................................................................
\subsubsection{Phishing Techniques}
%............................................................................................................
\label{s:phishing_techs}
There are various possibilities how phishers can obtain users' confidential information.
 In the following we describe phishing techniques that can be distinguished~\cite{jakobsson2006phishing, phishingtechniques}.
 This is important to know in order to determine what we are able to teach our target group.
%Online Identity Theft: Phishing Technology, Chokepoints and Countermeasures.
% ITTC Report on Online Identity Theft Technology and Countermeasures
%master_thesis/notes/phishing

\begin{description}[leftmargin=0cm]
	\item[Deceptive Phishing] In deceptive phishing social engineering plays a key role.
 Here, users are deluded into disclosing their confidential data directly to the phisher without being aware of it.
 A typical scenario is the unsuspecting user receiving an e-mail from an institution he trusts.
 In fact, this e-mail is malicious and links to a fake website, where the phisher intends to steal the user's data by capturing the fields the user enters trustfully.
 Once the phisher obtains the user's data, he is able to impersonate the victim's identity and benefit from this.

	\item[Malware Based Phishing] As the term already reveals, malware-based phishing embraces some kind of malicious software running on the user's computer.
 There are several ways of infecting the user's computer with such malware.
 Social engineering techniques can be used to convince the user to open malicious e-mail attachments or download malevolent files from a website.
 Another possibility is to exploit security vulnerabilities.
 Once the malware resides on the target, various technologies can be utilized to get at the users' data.
 Keyloggers and screenloggers, for example, track users' data input and send relevant information to a phishing server.
 Recent research has shown that mobile phone operating systems are as vulnerable to such attacks as desktop systems.
 Another way is to make use of so-called web trojans, which appear when users intend to log in.
 While the user thinks he is logging into a website of his trust, the entered information is actually transmitted to the phisher.

\end{description}

The above mentioned phishing techniques are the most common ones which influence the public understanding of the term most.
Despite these, there are other possible attacks that could be considered as phishing.

\begin{description}[leftmargin=0cm]
	\item[DNS Hijacking] This kind of phishing is also referred to as pharming and includes the manipulation of a system's host file or domain name system (DNS).
 These kinds of tampering result in returning a fraudulent IP address for URL requests and thus leading the user to a malicious website, even though the URL of a legitimate website had been entered.
 As a consequence, the unaware user enters his credentials into this fake website and the attacker obtains these which he can misuse.
 For the user these attacks are almost impossible to detect.

	\item[Man-in-the-Middle Attack] In this form of attack the phisher positions himself between the legitimate website and the user.
 The user's data input is delivered to the phisher, where he stores the information and then forwards it to the legitimate website.
 Responses are also forwarded back to the user so that the interference of the phisher does not affect the user's interactions.
 The gained sensitive information can then be sold or misused in any other way.
 As everything works as usual for the user, it is very difficult for him to detect such an attack.
 
	\item[Content Injection/XSS] Content injection refers to the practice of embedding additional harmful content into legitimate websites.
 This content can be, for example, malevolent code to log users' sensitive information and deliver the input to the phishing server.
 Well-known types of content injection include, for example, cross-site scripting (XSS).
XSS vulnerabilities result from a web application's usage of content from external sources, such as search terms, auctions or user reviews of a product.
 This type of data supply can be misused and instead of delivering the expected kind of data malicious scripts can be injected.

	\item[Search Engine Poisoning] Other phishing attempts involve search engines.
	With the aid of common search engine optimization techniques the phisher aspires to rank his phishing website higher than the legitimate website. By doing this he might trick users who use search engines to access websites into visiting his fraudulent page.
	
\end{description}

%Besides the different kinds of techniques of phishing, there also exist a number of attack channels a phisher can exploit.
 %The following section deals with these attack channels.

%(eventuell liste oder aufzählung) 
%EXAMPLE TABLE WHICH MIGHT BE USEFUL :D
%\begin{table}
%	\centering
%	\begin{tabularx}{.9\textwidth}{m{2.6cm} m{3.8cm} m{4.0cm} m{4.12cm}}
%	\hline	
%	\rowcolor{rowColorHead}
%										& Spalte 1 												& Spalte 2 			& Spalte 3\\
%	\hline
%	\rowcolor{rowColor1}
%	Zeile 1 					& Inhalte, \newline Inhalte			&	Inhalt			 		&	Inhalt \\		
%	\rowcolor{rowColor2}
%	Zeile 2 			& Inhalt, \newline Inhalt			&	Inhalt					&	Inhalt, \newline Inhalt	\\	
%	\hline
%	\end{tabularx}
%	\caption{Description}
%	\label{table:label}
%\end{table}

%............................................................................................................
\subsubsection{Phishing Attack Channels}
%............................................................................................................
Several attack channels exist that can be exploited by phishers to reach their victims.
 This section introduces some possible attack channels~\cite{phishing2010ramazan, phishingtechniques}.
 
In general the user receives a message from a phisher which is disguised as a party he trusts.
The message usually contains a link to a website asking the user for his personal information (e.g. user name and password). As the link seemed to come from a trusted party there might be users who do not expect something harmful behind this and thus enter their data~\cite{phishing2010ramazan}.
When the phisher acquires the user's credentials he can continue playing this game with the contacts of the newly derived user's account, which has just been compromised.
 
\label{s:attack_channels}
\begin{description}[leftmargin=0cm]
	\item[E-Mail] E-Mail spoofing is a common way for a phisher to reach his victims.
 These e-mails usually imitate renowned institutions, organizations, companies or banks the recipients trust.
 They usually contain a text which will deceive the recepient into doing what it says. Therefore psychological manipulation techniques e.g. pressure and threat are used.
 Typically a link to a malicious website, whose look and feel is almost identical to the original one, is included.
 On the malicious website the user is lured to enter his sensitive data which is captured by the phisher.
 Other alternatives are embedded forms in an email where a user fills in his data directly instead of being forwarded to a website.
 Finally, sometimes users are even asked to directly send back their confidential data.

	\item[SMS] An alternative to acquire confidential user data is making use of cell phone text messages.
 As with e-mails, the text message may contain a link to a fake website, where the user is induced to divulge his sensitive information.
 The user may also be asked to send back the information directly.
 Another possibility is to be asked to call back a fraudulent or expensive telephone number indicated in the text message.
 This number usually leads to an automated voice response system which is intended to gain the confidential information from the calling user.
 This form of phishing is also referred to as smishing, derived from the two terms ``SMS'' and ``phishing''.
 
	\item[Instant Messaging and Online Social Networks] The general approach outlined above can also be applied to IM services. Using online social networks is similar to using instant messaging services.
 However, online social networks provide additional valuable information to the phisher.
 With the aid of user profiles and pinboard entries etc.
 he can make his baits even more credible and trustworthy. For example if the phisher sees in the social network that the user likes playing a specific game. He might impersonate the developer of the game and refer to a severe problem with the users account and ask him to enter his credentials. 

	\item[Voice Phishing] A further possibility for a phisher is to send out spoofed e-mails asking the victim to call back the telephone number indicated in the e-mail.
 To deceive the user, the phisher as usual claims to be from a legitimate and trustworthy institution or organization.
 The number in the e-mail commonly leads to a voice response system by which the user is tricked to disclose confidential information.
 Alternatively, the phisher can directly call the user and lure him into divulging his senstitive information.
 Voice-over-IP (VoIP) further facilitates these kinds of attacks.
 It makes them easy to execute and inexpensive.
 Voice Phishing is also referred to as Vishing.
 
	\item[Physical letters] The phisher might even send out real letters to a big number of users. This is however unlikely because it is in contrast to the digital channels associated with costs and more effort.

\end{description}

%............................................................................................................
\subsubsection{Variations of Phishing}
%............................................................................................................
In the course of time different variations of phishing have evolved.
 This section deals with some of these variations that can be found in literature.

\label{s:phishing_variations}

\begin{description}[leftmargin=0cm]
	\item[Mass Phishing] Here, for example, the phisher sends out a tremendous amount of spoofed e-mails to random users.
 These e-mails usually link to the phisher's fake website where he tricks his victims to disclose their credentials.
 In this variation the phisher is not forced or even able to customize the mail to the attacked user.
 He tries to formulate the mail so that it might fit most users and accepts that some users might not fall for it.
 The principle of mass attacks is very common and effective, since sending e-mails and setting up websites is almost of no cost and effort nowadays.
 Even if not all phishing e-mails make it through the spam filters or are not opened: sending out a tremendous amount of spoofed e-mails evidently results in a high amount of victims, not in relative, but in absolute numbers.
 For example, there exist estimations of 156 million phishing e-mails being sent out daily.
 Only 16 million of these e-mails win the fight against spam filters.
 The half of these are opened.
 800,000 users of these 8 million e-mail recipients actually click on the contained link and still 80,000 users take the bait according to the estimations~\cite{takethebait}. As discussed in \autoref{s:stats} the reliability of phishing statistics is questionable. Yet these number indicate a rough overview of the problem.
	\item[Spear Phishing] Unlike mass phishing attacks, spear phishing mainly aims at sensitive information like business secrets, intellectual property or even military secrets.
 While in mass phishing attacks, spoofed e-mails are sent to millions of random users, spear phishing targets specific individuals resp.
 groups within organizations to acquire sensitive information.
 In order to make a deceptive request more credible and personal, knowledge of the targeted individuals and organizations is used.
 Usually, victims of spear phishing receive e-mails with a malicious attachment and are lured to download it~\cite{trendlabs2012spear}.
 As sharing documents via e-mail is normal in an organization this does usually not arouse suspicion if the e-mail is from a known person with a legitimate context.
 This makes spear phishing attacks very hard to detect\cite{trendlabs2012spear,statephishinghong}.
When a phisher attacks senior executives or other leaders in positions of influence this is sometimes called Whaling~\cite{whaling}.

\end{description}

%............................................................................................................
\subsubsection{Scope of Phishing in Our Analysis}
%............................................................................................................
\label{s:scope}
We have shown that phishing is a wide area and covering it in a whole will go way beyond the scope a masters thesis. Therefore we have to constrain the scope of the term phishing for this work.
 In literature most of the time phishing is described as the act of gaining sensitive information with the aid of fake websites which trick unsuspecting users into disclosing their credentials~\cite{sheng2007antiphishingphil, antiphishingtrendreport2013, kasperskyreport2013}.
This type of attack is the mostly observed one and is a form of deceptive phishing.
 For this reason we have decided to focus on deceptive phishing.
 
 As aforementioned, phishing websites can be distributed in several ways, including but not limited to e-mail, SMS, or online social networks.
 Additionaly these services might be accessed via multiple Applications (different e-mail Apps, dedicated Apps, Web-browser).
 If we want to cover all this it would increases the amount of information that we have to tell the user to an extend that we do not think that it will fit into a still easy to use application.
 E.g. we could tell the user that he can use an mail client that displays mails in plain text but this would only protect him from Links that come in via mail and would force him to use a certain mail application. Additionally it is unlikely that the user will check that each time before he clicks because that will interfere with his workflow.
 Therefore we set our focus on the analysis of URLs before entering private data such that any attack channel distributing a link to a fake website will be covered by our approach.
 However we, and the user should, know that by mere clicking the link to come to the website some information might already be send to the phisher.
 This includes the validity and activeness of the communication path (e-mail address, phone number, OSN account) and additional information (browser data, used ISP).
 
  Finally, there are three variations of phishing we have introduced.
 Our main focus is the mass phishing attack, since this is the common one.
 However, if any spear phishing or whaling attack involves fake websites, this would be covered by our approach as well.
Now that we have restricted our understanding of phishing, the next section provides our concrete definition of phishing for the scope of this thesis.

%............................................................................................................
\subsubsection{Our Definition of Phishing}
%............................................................................................................
In the following we present a concrete definition which encompasses our understanding of phishing for the scope of this work:

\begin{center}
\textit{``Phishing is the practice of obtaining confidential information from users and describes a form of identitfy theft. This attack exploits a user's trust rather than system vulnerabilities. More specifically, the user is fooled into believing that he is communicating with a party he trusts and lured into divulging confidential data. This usually happens through phishing websites which look deceptively similar to the originals. Targeted confidential information includes, but is not limited to, user names, passwords, social security numbers, credit card numbers, or account information.
''}~\cite{jakobsson2006phishing}
\end{center}

%===========================================
\subsection{Phishing Learning Techniques}
%===========================================

This section deals with different learning techniques used for phishing education in previous work.
 For better readability and comprehensibility we divided the related work into two categories: the \textit{content}, i.e.
 what the user is taught, and the 
%WHAT WAS THE USER TOLD$ and the 
\textit{medium}, i.e. how the user is taught.
These two categories can be further divided into several classes. 
In the following, we are going to provide an overview of these classes, before we provide specific examples of previous work in the next chapter.


%============================================
\subsubsection{Content Classification}
%============================================
\label{s:content_classification}
The content classification deals with the concrete content of learning which is communicated to the user. 
The objective of this section is to introduce the different classes of learning content that we identified in previous work.

\begin{description}[leftmargin=0cm]
	\item[General Knowledge Transfer] Renowned and targeted websites, such as PayPal, eBay or Microsoft provide general and superficial information about phishing~\cite{generalknowledgemicrosoft, generalknowledgepaypal, generalknowledgeebay}.
	Usually, they deal with questions like what is phishing, how does phishing happen, what the symptoms of phishing are and how to report phishing attempts.
 Providing the user only with text to the topic of phishing makes it possible to communicate almost any kind of content, so that the learning objectives can get as complex as one wishes.
 However, it is likely that users do not like reading too much, especially when it gets complex and difficult to comprehend.
 Of course the user's willingness to read a lot of complex text about computer security also depends on the user's motivation. 
	\item[E-Mail Based Knowledge] In this class of content, the user is told about the ``anatomy'' of phishing e-mails~\cite{antiphishingphyllis, sonicwall}. Particularly, they are informed about what kind of hints in an e-mail give indications for a phishing attempt.
 Indications for a phishing e-mail can be potentially malicious  attachments, impersonal salutation, requesting personal and confidential information as well as exerting pressure and threatening the user with, for example, account closure.
 The benefit of detecting phishing attempts before even clicking on a link in an e-mail is that the user would not confirm the existence and active usage of his e-mail address to the phisher.
 More importantly, the user would not unknowingly download malicious software.
 The problem with the e-mail based approach is that detecting phishing e-mails by looking at their content becomes more and more difficult~\cite{microsoftphishing,spamfighter}. Even if today  many phishing e-mails exhibit the obvious characteristic we expect that phishing e-mails will improve and therefore these obvious hints will not remain in future. 

	\item[URL Based Knowledge] Sending spoofed e-mails with links to fake websites is a common trick of phishers.
 On the target website the user is lured to disclosing his credentials.
 Thus, detecting such fake websites is another possibility to protect oneself against phishing.
 Here the user is taught to distinguish phishing URLs from legitimate ones~\cite{sheng2007antiphishingphil, arachchilage2012designing}. 
Links to phishing websites are not only distributed by phishing e-mails.
 Such links can be spread via any communication channel, such as online social networks or SMS.
 It is even possible to land on a phishing website by just browsing the web.
 In these situations knowing how to distinguish phishing URLs from valid ones will help whereas knowledge about phishing e-mails in general will not.
 The problem with this approach is that as soon as the DNS or host file is attacked even for experts it will get difficult to distinguish a phishing website from the legitimate one (cf.~DNS~Based~Phishing~in \autoref{s:phishing_techs}).
 Also, it is unlikely that the user is checking the URL after each click. So the user must develop a strategy when to check the URL (e.g. before entering personal data) and when not.

\end{description}


%============================================
\subsubsection{Medium Classification}
%============================================
\label{s:medium_classification}
The learning medium describes how the learning content is communicated to the user. 
The objective of this section is to introduce the different classes of learning media that we identified in previous work.

\begin{description}[leftmargin=0cm]
    \item[Simple Text] One possible media to deliver knowledge is simple text. It can be in written or spoken form. For example most people in Germany learn reading in the elementary school with textbooks. Textbooks and lectures are also commonly used in university education.
    The main problem is that in modern time there are more interactive alternatives that some people might prefer.
    Additionally some facts can better be transported with graphics than with text.
    
	\item[Game Based Learning] Game based learning tries to communicate the learning content vividly and playfully through a game.
 Such a game usually has a ``background story'' and a ``mission'' the user has to accomplish~\cite{sheng2007antiphishingphil,antiphishingphyllis}. The game design is important and depends on the target group.
 Previous work in the area of phishing, for example, has focused on a fish as starring role in their game, cf. \autoref{s:related_work}. This might work well for a target group of young age, but will most likely not be appealing to a larger audience.
 This is also reflected by our phishing survey, cf. \autoref{s:survey}.
	\item[Quiz Based Learning] The quiz based approach is a type of a game which relies on a question-answer cycle without using a specific background story~\cite{onguardonline}. The advantage of a quiz based approach is that it seems more appropriate for adults and thus will likely be appealing to a larger audience.

	\item[Comparison Based Learning] A further way to teach users is to let them compare legitimate websites, URLs, or e-mails, with fake ones.
 Here the user has to decide which of the shown examples are the secure ones~\cite{staysafeonline}. 
We believe that this form of learning would increase the user awareness, as with this approach one could visualize to the user how difficult it can be to distinguish an original from a fake, especially when they appear almost identical.
 On the contrary, this way of learning does not reflect the reality, which is a major drawback in our point of view.
 In real life the user does not have the luxury of chosing between two options, he has only one and has to decide whether this option is trustful or not.

	\item[Emdedded Learning] The aim of embedded learning is to educate the user on the topic of phishing during his every day life.
 For this reason the user is sent simulated phishing e-mails.
 In case the user falls for this simulated phishing attempt he is notified and gets more information regarding phishing and how to protect himself~\cite{embedded2011jansson, kumaraguru2009phishguru}. 
This approach benefits from the so called ``teachable moment''. 
The moment the user realizes that he has almost become a victim to a phishing attack, he will be highly motivated to prevent this happening again and thus be highly receptive for input related to this topic.
 The teachable moment itself will not suffice to make the user stay on and consult the educational page, though.
 For example, there was a study in Germany to assess the effectiveness of CMU/APWG's landing page~\cite{TUD-CS-2013-0167}. 
 During the study the authours found that people just closed the window immediately after or shortly after landing on the educational page without reading on,  because they thought they were on the wrong website and were not aware that they landed on an educational site.
 Even if with an effective landing page, the missing positive feedback is a major flaw of this strategy in our opinion.
 The user is only notified in case of a mistake and not in case he has successfully discarded the simulated phishing e-mail.
 A further problem is raised with the implementation of such an approach.
 Legal issues will arise when sending simulated phishing e-mails which claim to come from a reputable vendor, such as an online shop.

\end{description}

Due to the drawbacks of embedded learning (legal issues) and comparison based (unrealistic) approaches we believe that a mixture of the game and quiz based approach containing relevant informative text is the best way to go.
This approach might be more appealing to a larger audience compared to, for example, just offering simple text.
Yet, testing whether a quiz and game based approach is in fact more appealing and appropriate remains for future work.
At the very least our phishing survey reveals that potential users tend to vote for quiz based approaches in comparison to simple text or a game with a fish as main character (cf. \autoref{s:survey_results}).
Regarding the content which will be communicated to the user we decided to focus on detecting phishing URLs for the reasons explored in this section.
A major aspect to consider in further research is the knowledge retainment.
For this purpose our approach should be tested in long-term studies and possibly compared to alternative approaches.