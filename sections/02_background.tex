%*******************************************
\section{Background}
%*******************************************
\label{s:background}
The objective of this chapter is to provide the required background knowledge for our further design elaborations. 
We split this chapter into four parts.
The first part deals with the term phishing in general which includes common phishing techniques, attack channels, and variations of phishing.
This section illustrates the vastness of the term phishing and is intended to narrow it to a definition which reflects the general understanding of it and which we consider in our work.
Irrespective of how a phisher obtained sensitive information from his victims, it has consequences for the fooled person as well as for the targeted company.
These consequences are briefly illustrated in the second part of this section.
As argued in the previous section our intention is to develop a complementary approach to technical solutions which raises security awareness and offers the user an educational service which trains him to detect phishing attempts.
We decided to offer this service as a smartphone app which is reasoned in the third part of this section.
In the last part we provide a brief overview of anti-phishing education approaches.
For better readability and comprehensibility we divided the available approaches into their content, i.e. what specific content is the user told, and the used media, i.e. how is this content communicated to the user. Specific examples of previous work are provided in the next chapter.

%===========================================
\subsection{Phishing in General}
%===========================================
\label{s:phishing_general}
This section elaborates on the topic of phishing in general.
Phishing is a term which is referred to for various scenarios and techniques.
Consequently, there are different definitions of phishing found in literature.
Therefore, we start with a definition that entails all types of phishing.
Subsequently, we introduce different phishing techniques, used attack channels and variations of phishing.
Finally, we state our scope with respect to the term of phishing and provide our own definition of it which we consider in this work.
%............................................................................................................
\subsubsection{Abstract Definition of Phishing}
%............................................................................................................
\label{s:phishing_def}
The goal of this work is to help users distinguish phishing websites from legitimate ones. 
 Since phishing is important within the scope of this work, we define the term first. In fact, phishing is a term that is used by many people in different contexts. Therefore, the following definition is deliberately kept abstract in order to cover all possible scenarios of phishing. At the end of this chapter we will state our definition of phishing which we consider in this work.

\begin{center}
\textit{``Phishing is the practice of obtaining confidential information from users and describes a form of identitfy theft.
 Targeted confidential information includes, but is not limited to, user names, passwords, social security numbers, credit card numbers, or account information.
''}~\cite{jakobsson2006phishing}
\end{center}

%............................................................................................................
\subsubsection{Phishing Techniques}
%............................................................................................................
\label{s:phishing_techs}
There are various possibilities how phishers can obtain users' confidential information.
 In the following we describe phishing techniques that can be distinguished~\cite{jakobsson2006phishing, phishingtechniques}.
 This is important to know in order to determine what we are able to teach our target group.
%Online Identity Theft: Phishing Technology, Chokepoints and Countermeasures.
% ITTC Report on Online Identity Theft Technology and Countermeasures
%master_thesis/notes/phishing

\begin{description}[leftmargin=0cm]
	\item[Deceptive Phishing:] In deceptive phishing social engineering plays a key role.
 Here, users are deluded into disclosing their confidential data directly to the phisher without being aware of it.
 A typical scenario is the unsuspecting user receiving an e-mail from an institution he trusts.
 In fact, this e-mail is malicious and links to a fake website, where the phisher intends to steal the user's data by capturing the fields the user enters trustfully.
 Once the phisher obtains the user's data, he is able to impersonate the victim's identity and benefit from this.

	\item[Malware Based Phishing:] As the term already reveals, malware-based phishing embraces some kind of malicious software running on the user's computer.
 There are several ways of infecting the user's computer with such malware.
 Social engineering techniques can be used to convince the user to open malicious e-mail attachments or download malevolent files from a website.
 Another possibility is to exploit security vulnerabilities.
 Once the malware resides on the target, various technologies can be utilized to get at the users' data.
 Keyloggers and screenloggers, for example, track users' data input and send relevant information to a phishing server.
 Recent research has shown that mobile phone operating systems are as vulnerable to such attacks as desktop systems.
 Another way is to make use of so-called web trojans, which appear when users intend to log in.
 While the user thinks he is logging into a website of his trust, the entered information is actually transmitted to the phisher.

\end{description}
The above mentioned phishing techniques are the most common ones which influence the public understanding of the term most.
Despite these, there are other possible attacks that could be considered as phishing.

\begin{description}[leftmargin=0cm]
	\item[DNS Hijacking:] This kind of phishing is also referred to as pharming and includes the manipulation of a system's host file or domain name system (DNS).
 These kinds of tampering result in returning a fraudulent IP address for URL requests and thus leading the user to a malicious website, even though the URL of a legitimate website had been entered.
 As a consequence, the unaware user enters his credentials into this fake website and the attacker obtains these which he can misuse.
 For the user these attacks are almost impossible to detect.

	\item[Man-in-the-Middle Attack:] In this form of attack the phisher positions himself between the legitimate website and the user.
 The user's data input is delivered to the phisher, where he stores the information and then forwards it to the legitimate website.
 Responses are also forwarded back to the user so that the interference of the phisher does not affect the user's interactions.
 The gained sensitive information can then be sold or misused in any other way.
 As everything works as usual for the user, it is very difficult for him to detect such an attack.
 
	\item[Content Injection/XSS:] Content injection refers to the practice of embedding additional harmful content into legitimate websites.
 This content can be, for example, malevolent code to log users' sensitive information and deliver the input to the phishing server.
 Well-known types of content injection include, for example, cross-site scripting (XSS).
XSS vulnerabilities result from a web application's usage of content from external sources, such as search terms, auctions or user reviews of a product.
 This type of data supply can be misused and instead of delivering the expected kind of data malicious scripts can be injected.

	\item[Search Engine Poisoning:] Other phishing attempts involve search engines.
	With the aid of common search engine optimization techniques the phisher aspires to rank his phishing website higher than the legitimate website. By doing this he might trick users who use search engines to access websites into visiting his fraudulent page.
	
\end{description}

%Besides the different kinds of techniques of phishing, there also exist a number of attack channels a phisher can exploit.
 %The following section deals with these attack channels.

%(eventuell liste oder aufzählung) 
%EXAMPLE TABLE WHICH MIGHT BE USEFUL :D
%\begin{table}
%	\centering
%	\begin{tabularx}{.9\textwidth}{m{2.6cm} m{3.8cm} m{4.0cm} m{4.12cm}}
%	\hline	
%	\rowcolor{rowColorHead}
%										& Spalte 1 												& Spalte 2 			& Spalte 3\\
%	\hline
%	\rowcolor{rowColor1}
%	Zeile 1 					& Inhalte, \newline Inhalte			&	Inhalt			 		&	Inhalt \\		
%	\rowcolor{rowColor2}
%	Zeile 2 			& Inhalt, \newline Inhalt			&	Inhalt					&	Inhalt, \newline Inhalt	\\	
%	\hline
%	\end{tabularx}
%	\caption{Description}
%	\label{table:label}
%\end{table}

%............................................................................................................
\subsubsection{Phishing Attack Channels}
%............................................................................................................
Several attack channels exist that can be exploited by phishers to reach their victims.
 This section introduces some possible attack channels~\cite{phishing2010ramazan, phishingtechniques}.
 
\label{s:attack_channels}
\begin{description}[leftmargin=0cm]
	\item[E-Mail:] E-Mail spoofing is a common way for a phisher to reach his victims.
 These e-mails usually imitate renowned institutions, organizations, companies or banks that the recipients trust.
 They usually contain a text which will deceive the recipient into doing what it says. 
For this purpose, psychological manipulation techniques are used, including, but not limited to, exerting pressure or issuing threats.
 Typically a link to a malicious website, whose look and feel is almost identical to the original one, is included.
 On this website the user is deluded into entering sensitive data which is captured by the phisher.
 An alternative is the usage of embedded forms in an e-mail where the user fills in the requested data directly instead of being forwarded to a fraudulent website.
 Finally, sometimes users are even asked to directly send back their confidential data.

	\item[SMS:] An alternative to acquire confidential user data is making use of cell phone text messages.
 As with e-mails, the text message may contain a link to a fake website, where the user is induced into divulging sensitive information.
 The user may also be asked to send back the information directly.
 Another possibility is being asked to call back a fraudulent or expensive telephone number.
 This number usually leads to an automated voice response system which is intended to gain the confidential information from the calling user.
 This form of phishing is also referred to as smishing, derived from the two terms ``SMS'' and ``phishing''.

	\item[Instant Messaging:] Spreading links via instant messages is another way for a phisher to reach his victims. Once the phisher has gained access to a victim's account he can pretend to be him and lure his contacts into disclosing their data as well. The phisher can continue this game repeatedly.
Obviously, this kind of deception can be applied in other attack channels, such as e-mails or online social networks, as well.
 
	\item[Online Social Networks:] Using online social networks is similar to using instant messaging services.
 However, online social networks provide additional valuable information to the phisher.
 With the aid of user profiles and pinboard entries etc. he can make his baits even more credible and trustworthy. 
For example, with the aid of a social network the phisher might find out that a potential victim likes a specific game. In order to delude this user the phisher might pretend to be the developer of this  game, refer to a severe problem with the user's account and ask him to enter his credentials.

	\item[Voice Phishing:] A further possibility for a phisher is to send out spoofed e-mails asking the victim to call back the telephone number indicated in the e-mail.
 To deceive the user, the phisher as usual claims to be from a legitimate and trustworthy institution or organization.
 The number in the e-mail commonly leads to a voice response system by which the user is induced into disclosing confidential information.
 Alternatively, the phisher may directly call the user.
 Voice-over-IP (VoIP) further facilitates these kinds of attacks.
 It makes them easy to execute and inexpensive.
 Voice phishing is also referred to as vishing.
 
	\item[Physical letters:] The phisher might even send out real letters to a number of users. However, we believe that this is unlikely because in contrast to the digital channels, this channel is associated with expenses and more effort.

\end{description}

%............................................................................................................
\subsubsection{Variations of Phishing}
%............................................................................................................
There exist two major variations of phishing which can be distinguished, mass phishing and spear phishing.
As the names reveal, mass phishing involves targeting a large number of users, while spear phishing rather refers to targeting a specific user or group of users.
In this section we discusses these two variations.

\label{s:phishing_variations}

\begin{description}[leftmargin=0cm]
	\item[Mass Phishing:] In the case of mass phishing the attacker sends out a tremendous amount of spoofed e-mails to random users.
 These e-mails usually link to the phisher's fake website where he tricks his victims into disclosing their credentials.
 In this variation the phisher is not forced or even able to customize the e-mail to the attacked user.
 He formulates the e-mails such that they might persuade most users and accepts that some users might not fall for it.
 The principle of mass attacks is very common and effective, since sending e-mails and setting up websites is almost of no cost and effort nowadays.
 Even if not all phishing e-mails make it through the spam filters or are not opened: sending out a tremendous amount of spoofed e-mails evidently results in a high amount of victims, not in relative, but in absolute numbers.
 For example, there exist estimations of 156 million phishing e-mails being sent out daily.
 Only 16 million of these e-mails win the fight against spam filters.
 The half of these are opened.
 800,000 users of these 8 million e-mail recipients actually click on the contained link and still 80,000 users take the bait according to the estimations~\cite{takethebait}. As discussed in \autoref{s:introduction} the reliability of phishing statistics is questionable. Yet, these number indicate a rough overview of the problem.
	\item[Spear Phishing:] Unlike mass phishing attacks, spear phishing mainly aims at sensitive information like business secrets, intellectual property or even military secrets.
 While in mass phishing attacks, spoofed e-mails are sent to millions of random users, spear phishing targets specific individuals resp.
 groups within organizations to acquire sensitive information.
 In order to make a deceptive request more credible and personal, information about the targeted individuals and organizations is used.
 Usually, victims of spear phishing receive an e-mail with a malicious attachment and are induced into downloading it~\cite{trendlabs2012spear}.
 As sharing documents via e-mail is normal in an organization this does usually not arouse suspicion, if the e-mail is from a known person with a genuine context.
 This makes spear phishing attacks very hard to detect~\cite{trendlabs2012spear,statephishinghong}.
When a phisher attacks senior executives or other leaders in positions of influence this is sometimes referred to as whaling~\cite{whaling}.
\end{description}

%............................................................................................................
\subsubsection{Scope of Phishing in Our Analysis}
%............................................................................................................
\label{s:scope}
We showed that phishing is a wide area. 
Covering it in a whole will go beyond the scope of a masters thesis. 
Therefore, we have to constrain the scope of this term.
 In literature phishing is described as the act of gaining sensitive information from unsuspecting users, usually with the aid of fake websites~\cite{sheng2007antiphishingphil, antiphishingtrendreport2013, kasperskyreport2013}.
Here, instead of exploiting system vulnerabilities the users themselves and their trust are exploited.
This form of attack is referred to as deceptive phishing.
This type of attack is the mostly observed one and influences the public understanding of the term phishing.
 For this reason, we decided to focus on deceptive phishing.
 
 As aforementioned, phishing websites can be distributed in several ways, including, but not limited to, e-mail, SMS, or online social networks.
 Additionaly, these services might be accessed via multiple applications (different e-mail applications, dedicated apps, browsers).
%If we want to cover all this it would increases the amount of information that we have to tell the user to an extent that we do not think that it will fit into a still easy to use application.
 %E.g. we could tell the user that he can use an e-mail client that displays e-mails in plain text but this would only protect him from Links that come in via e-mail and would force him to use a certain %e-mail application. Additionally it is unlikely that the user will check that each time before he clicks because that will interfere with his workflow.
 In order to be independent from the source a link may originate from, we set our focus on the analysis of URLs before entering private data (cf. \autoref{s:coverage}), i.e. on the website itself, such that any attack channel distributing a link to a fake website will be covered by our approach.
% However we, and the user should, know that by mere clicking the link to come to the website some information might already be send to the phisher.
% This includes the validity and activeness of the communication path (e-mail address, phone number, OSN account) and additional information (browser data, used ISP).
% das kommt spaeter eh..
 
  Finally, there are two major variations of phishing we introduced.
 Our main focus is the mass phishing attack, since this is the common one.
 However, if any spear phishing or whaling attack involves fake websites, this would be covered by our approach as well.
Yet, as discussed above spear phishing and whaling attacks are very difficult to detect~\cite{trendlabs2012spear,statephishinghong}.
Hence, it appears to be reasonable to target this issue separately from mass phishing in further research.

Now that we restricted our understanding of phishing, we provide our definition of the term for the scope of this thesis in the following section.

%............................................................................................................
\subsubsection{Our Definition of Phishing}
%............................................................................................................
In the following we present our definition of phishing which encompasses our and the genral public understanding of it:

\begin{center}
\textit{``Phishing is the practice of obtaining confidential information from users and describes a form of identitfy theft. This attack exploits a user's trust rather than system vulnerabilities. More specifically, the user is fooled into believing that he is communicating with a party he trusts and lured into divulging confidential data. This usually happens through phishing websites which look delusively similar to the originals. Targeted confidential information includes, but is not limited to, user names, passwords, social security numbers, credit card numbers, or account information.
''}~\cite{jakobsson2006phishing}
\end{center}

%-------------------------------------------
\subsection{Consequences of Phishing}
%-------------------------------------------
Irrespective of in which way the phisher obtained sensitive data, falling for a phishing attack has consequences for the fooled person as well as for the targeted company or organization.
In the following some of these consequences are briefly illustrated.

\begin{description}[leftmargin=0cm]
\item[Identity Theft:] The main goal of the attacker is to impersonate the attacked party by stealing his credentials. That is to say, a possible consequence of falling for a phishing attack is identity theft~\cite{jakobsson2006phishing}. With the obtained information the phisher can, for instance, do online shopping or access the corporate infrastructure on behalf of his victims.
\item[Data Theft:]
In a private environment the phisher might collect the user's contacts or all kinds of other sensitive information.
In case the attacker gains access to corporate systems he might be able to read and copy customer data or other confidential information.
\item[Reputational Damage:]
When the phisher gets access to a social network account he might be able to deceive ``friends'' of the victim as well. This might have a negative impact on the victim's reputation.
Moreover, if a customer falls for an attack he might blame the targeted company for not protecting him and his data appropriately. 
Ultimately, this customer might lose confidence in eCommerce operations and the Internet in general.

In another scenario an employee might fall for a phishing trap.
If such news reports are published this might undermine the trust of potential and current customers in the attacked company~\cite{mcafee, redcondor}. 
\item[Financial Loss:]
An attacker might be able to plunder private or corporate bank accounts which results in financial loss for each victim. Additionally, organizations have to face increased support expenses caused by the problem of phishing~\cite{rsa2013, mcafee}.
\end{description}

%-------------------------------------------
\subsection{Anti-Phishing Education on the Smartphone}
%-------------------------------------------
\label{s:antiphishing_on_smartphone}
The goal of our work is to increase users' security awareness and offer them a service with which they can learn to protect themselves against phishing attacks.
We decided to implement both an appealing as well as valuable service that incorporates the beforementioned goals as a smartphone app.
In the following we reason our decision:

\begin{description}[leftmargin=0cm]
\item[Mobility and Size:] The main characteristic of a smartphone is that it is mobile and smaller than the well-known desktop computers.
As a consequence, there is less space on the screen.
Many browsers, for example, generally hide their address bars due to the lack of space.
With the address bar, the URL and other potential security indicators are hidden.
The release of iOS7 features a key step towards better transparency for the user.
iOS7's Safari browser displays the host instead of the website's title or the URL itself.
This might make phishing attacks more difficult to succeed, assumed that users look at and assess this area of the browser.
Additionally, in portrait mode the host is displayed even when scrolling down the page, i.e. this relevant information is always visible.
An interesting question to ask here is whether, when and how many browsers will follow such an approach.
Currently, Android does not support such a functionality. 
Yet, displaying the host instead of the complete URL or the title only facilitates users to detect phishing.
There is still a need for URL parsing comprehension for these purposes.
\item[Distraction Caused by Mobility:] Users often use their smartphones while on the move, when walking or during a train or a bus ride, for example.
These circumstances imply distractions from the environment.
These distractions obviously will influence the user's attentiveness.
Hence, smartphone users might be even more vulnerable to phishing attacks than the traditional desktop user.
This is also indicated by Boodaei's report~\cite{trusteer2011}, which says that mobile users are three times more likely to access phishing websites than desktop users.
This might also be affected by the fact that mobile e-mail clients effectively provide no way to check the validity of an incoming e-mail.
The potential distraction raises the question whether it has an impact on the user's education and retentiveness.
According to the principles of learning (cf.~\autoref{s:learning_principles}) it most likely has an impact on the learning performance.
Yet, we believe that our exercise and repetition scheme (cf.~\autoref{s:learning_principles}) helps users to internalize the learning content despite potential distractions.
For further research it would be interesting to test how significantly distractions impact the learning results of our app, though.
\item[High Number of Smartphone Users:] In addition, given that the majority of the people use a smartphone on a regular basis in Spain, Germany, Italy, France, and the UK~\cite{smartphoneusage}, there is a need for the protection of smartphone users.
%With the smartphones, apps and app games became popular.
%Therefore, we believe an interactive game appears to be a good opportunity for education.

\end{description} 
Overall, educating the user on the smartphone provides two major benefits.
First, the user can access the app on the move, even outside of his desktop environment.
The app can be used during train or bus rides, or while bridging the time.
Moreover, it can be started and continued any time as a sideline.
Despite the fact that we mainly aim at users who want to do something about their unknowingness (cf.~\autoref{s:target_group}), we hope that an enjoyable app might reach even more users.
Second, we believe it is easier to transfer knowledge of smartphones to desktop computers regarding several aspects.
For example, the parsing of a URL can be easily transferred from smartphones to desktop computers, as desktop screens are bigger and a URL is easier to find compared to smartphones.
Transferring knowledge from desktop computers to smartphones, on the other hand, raises more complicated issues.
The parsing of a URL on a desktop computer, for example, cannot be easily transferred to smartphones.
The user needs to know how to access the generally hidden address bar and how to view the complete URL.
Icons or security warnings are probably not easy to transfer in any direction since those differ significantly among devices, versions and browsers.

In the following we discuss various anti-phishing education approaches that we divided into the content that is delivered to the users by them and how these specific contents are delivered to the users.
After giving a rough overview of these classes we provide in-depth insight to related work in the next chapter.

%===========================================
\subsection{Overview of Anti-Phishing Education Approaches}
%===========================================
This section provides an overview of anti-phishing education approaches of previous work.
 For better readability and comprehensibility we divided the approaches into two categories: the \textit{content}, i.e.
 what the user is taught, and the 
%WHAT WAS THE USER TOLD$ and the 
\textit{medium}, i.e. how the content is taught.
These two categories can be further divided into several classes. 
In the following, we are going to provide an overview of these classes, before we provide specific examples of previous work in the next chapter.


%============================================
\subsubsection{Classes of Delivered Content}
%============================================
\label{s:content_classification}
The content classification deals with the precise content of learning which is communicated to the user. 
The objective of this section is to introduce the different classes of learning content that we identified in previous work.

\begin{description}[leftmargin=0cm]
	\item[General Knowledge Transfer:] Renowned and targeted websites, such as PayPal, eBay or Microsoft provide general and superficial information about phishing~\cite{generalknowledgemicrosoft, generalknowledgepaypal, generalknowledgeebay}.
	Usually, they deal with questions like what is phishing, how does phishing happen, what are the symptoms of phishing and how to report phishing attempts.
	\item[E-Mail Based Knowledge:] In this class of content, the users are told about the ``anatomy'' of phishing e-mails~\cite{antiphishingphyllis, sonicwall}. Particularly, they are informed about what kind of hints in an e-mail give indications for a phishing attempt.
 Indications can be potentially malicious attachments, impersonal salutation, requesting personal and confidential information as well as exerting pressure and threatening the user with, for example, account closure.
 The benefit of detecting phishing attempts before even clicking on a link in an e-mail is that the user would not confirm the existence and active usage of his e-mail address to the phisher.
 More importantly, the user would not unknowingly download malicious software.
 The problem with the e-mail based approach is that detecting phishing e-mails by looking at their content becomes more and more difficult~\cite{microsoftphishing,spamfighter}. Even if today  many phishing e-mails exhibit obvious characteristics we expect that phishing e-mails will improve. Therefore, we believe it is likely that these obvious hints will not remain in the future. 

	\item[URL Based Knowledge:] Sending spoofed e-mails with links to fake websites is a common trick of phishers.
 On the target website the user is lured into disclosing his credentials.
 Thus, detecting such fake websites is another possibility to protect oneself against phishing.
 Here, the user is taught to distinguish phishing URLs from legitimate ones~\cite{sheng2007antiphishingphil, arachchilage2012designing}. 
Links to phishing websites are not only distributed by phishing e-mails.
 Such links can be spread via any communication channel, such as online social networks or SMS.
 It is even possible to land on a phishing website by just browsing the web.
 In these situations knowing how to distinguish phishing URLs from valid ones will help whereas knowledge about phishing e-mails in general will not.
 The problem with this approach is that as soon as the DNS or host file is attacked even for experts it will get difficult to distinguish a phishing website from the legitimate one (cf.~DNS~Hijacking~in \autoref{s:phishing_techs}).
 Also, it is unlikely that the user checks a URL after each click. This is why, the user should develop a strategy when to check a URL (for example, before entering personal data) and when not.
Despite its downsides, we believe that URL based knowledge gives the most reliable hint regarding its ``origin'', i.e. whether a URL in fact belongs to a legitimate website or not.
We had a look at the phishing URLs provided by PhishTank~\cite{phishtank}. The majority of these URLs were not or only loosely related to the attacked website. If the users would be aware of the importance of the URL and were able to interpret it the phishers would put more effort in forging valid looking URLs. Obviously, there are enough users falling for primitive attacks. Therefore, we think that it is important to inform the users about the significance of URLs and to teach them how to interpret those.
\end{description}


%============================================
\subsubsection{Classes of Applied Media to Deliver Content}
%============================================
\label{s:medium_classification}
The learning medium describes how the learning content is communicated to the user. 
The objective of this section is to introduce the different classes of learning media that we identified in previous work.

\begin{description}[leftmargin=0cm]
    \item[Simple Text:] One possible medium to provide information is simple text. It can be delivered in written or spoken form. For example, most people in Germany learn reading in elementary schools with textbooks. 
Textbooks and lectures are also commonly used in university education.
 Providing the user only with text to the topic of phishing makes it possible to communicate almost any kind of content, so that the learning objectives can get as complex as one wishes.
Yet, a user's willingness to read a lot of complex text about computer security depends on his motivation. 
 Moreover, some facts can better be transferred with graphics than with text and in modern time there are more interactive alternatives to simple texts that some people might prefer.
    
	\item[Game Based Learning:] Game based learning tries to communicate the learning content vividly and playfully through a game.
 Such a game usually has a ``background story'' and a ``mission'' the user has to accomplish~\cite{sheng2007antiphishingphil,antiphishingphyllis}. The game design is important and depends on the target group.
 Previous work in the area of phishing, for example, focused on a fish as starring role in their game (cf. \autoref{s:related_work}). This might work well for a target group of young age, but will most likely not be appealing to a larger audience.
 This is also supported by our survey (cf. \autoref{s:survey}).
	\item[Quiz Based Learning] The quiz based approach is a type of a game which relies on a question-answer cycle without using a specific background story~\cite{onguardonline}. The advantage of a quiz based approach is that it seems to be more appropriate for adults and thus will likely be appealing to a larger audience, which is also supported by our survey (cf. \autoref{s:survey}).

	\item[Comparison Based Learning:] A further way to teach users is to let them compare legitimate websites, for example, URLs, or e-mails, with fake ones.
 Here, the user has to decide which of the shown examples are the secure ones~\cite{staysafeonline}. 
We believe that this form of learning would increase the user awareness, as with this approach one could visualize to the user how difficult it can be to distinguish an original from a fake, especially when they appear almost identical.
 On the contrary, this way of learning does not reflect the reality, which is a major drawback in our point of view.
 In real life the user does not have the luxury of choosing between two options, he has only one and has to decide whether this option is trustful or not.

	\item[Emdedded Learning:] The aim of embedded learning is to educate the user on the topic of phishing during his every day life.
 For this reason, the user is sent simulated phishing e-mails.
 In case the user falls for this simulated phishing attempt he is notified and gets more information regarding phishing and how to protect himself~\cite{embedded2011jansson, kumaraguru2009phishguru}. 
This approach benefits from the so called ``teachable moment''. 
In the moment the user realizes that he has almost become a victim to a phishing attack, he will be highly motivated to prevent this happening again and thus be highly receptive for input related to this topic.
 Yet, a study in Germany revealed that the teachable moment renders superfluous in case the users do not realize what is happening~\cite{TUD-CS-2013-0167}.
 The authors of this study assessed the effectiveness of CMU/APWG's landing page. 
They found out that people just closed the window immediately after or shortly after landing on the educational page without reading on,  because they thought they were on the wrong website and were not aware that they landed on an educational site.
 Even with an effective landing page, the missing positive feedback is a major flaw of this strategy in our opinion.
 The user is only notified in case of a mistake and not in case he has successfully discarded the simulated phishing e-mail.
 A further problem is raised with the implementation of such an approach.
 Legal issues will arise when sending simulated phishing e-mails which claim to come from a reputable vendor, such as an online shop.
\end{description}
Due to the drawbacks of embedded learning (legal issues) and comparison based approaches (unrealistic) we believe that a mixture of the game and quiz based approach containing relevant informative text is the best way to go.
This approach might be more appealing to a larger audience compared to, for example, just offering simple text.
Yet, testing whether a quiz and game based approach is in fact more appealing and appropriate remains for future work.
At the very least our survey reveals that potential users tend to vote for quiz based approaches in comparison to simple text or a game with a fish as a main character (cf. \autoref{s:survey_results}).
Regarding the content which will be communicated to the user we decided to focus on detecting phishing URLs for the reasons explored in this section.
A major aspect to consider in further research is the knowledge retention.
For this purpose, our approach should be tested in long-term studies and possibly compared to alternative approaches.