\selectlanguage{ngerman}
\begin{abstract}
Betr{\"u}ger entdecken das Internet als einen geeigneten Ort f{\"u}r ihre kriminellen Aktivit{\"a}ten.
Zum Beispiel schicken sie Internetnutzern gef{\"a}lschte E-Mails, die Links zu wiederum gef{\"a}lschten Webseiten enthalten.
Die Webseiten fordern die Besucher auf, ihre vertraulichen Daten einzugeben.
Diese Art von Internetbetrug wird als Phishing bezeichnet.
%Solche Angriffe k{\"o}nnen f{\"u}r die Opfer sowohl wirtschaftliche als auch emotionale Folgen haben.
Es existieren verschiedene technische L{\"o}sungen, die zum Ziel haben, das Phishing-Problem zu l{\"o}sen, indem der Benutzer beispielsweise vor dem Zugriff auf eine bekannte Phishing-Webseite gewarnt wird.
Diese Ans{\"a}tze k{\"o}nnen allerdings keinen rumdum zuverl{\"a}ssigen Schutz garantieren, weil es immer L{\"o}sungen geben wird, wie diese Techniken umgangen werden k{\"o}nnen.
Dar{\"u}ber hinaus werden Sicherheitswarnungen und -hinweise solcher Ans{\"a}tze von den Benutzern nicht immer wahrgenommen oder gar ignoriert.
Aus diesen Gr{\"u}nden ist ein komplement{\"a}rer Ansatz erforderlich.
Bisherige Ans{\"a}tze greifen nicht auf einen entscheidenden Faktor zur{\"u}ck, um der Gefahr zu begegnen - den Benutzer selbst. Die Erh{\"o}hung des Bewusstseins f{\"u}r Sicherheit und vor allem die Benutzeraufkl{\"a}rung {\"u}ber die Gefahren des Internets sehen wir als weiteren wichtitgen Schritt in Richtung einer ausgereiften Strategie zur Bek{\"a}mpfung von Phishing.

Unsere Masterarbeit besch{\"a}ftigt sich mit der Entwicklung einer Smartphone-App, die das Bewusstsein f{\"u}r Sicherheit erh{\"o}ht und die Benutzer bez{\"u}glich Phishing-Erkennung aufkl{\"a}rt und trainiert.
Zur Sensibilisierung des Bewusstseins f{\"u}r Sicherheit, senden sich die Benutzer selbst eine ``gef{\"a}lschte'' E-Mail, direkt zu Anfang der App.
%Auf diese Weise veranschaulichen wir beispielhaft, wie trivial es ist E-Mails zu f{\"a}lschen. 
%Die m{\"o}glicherweise daraus resultierende Motivation und das Engagement k{\"o}nnen zus{\"a}tzlich zu einem verbesserten Lernerfolg f{\"u}hren.
Der Trainingsteil der App beinhaltet Informationen und Warnungen zu bekannten Techniken der Angreifer und hilft den Benutzern, diese durch {\"U}bungen und Wiederholungen zu verinnerlichen.
Mit diesen soll der Benutzer die F{\"a}higkeit erlangen, sich in Zukunft vor Phishing-Angriffen zu sch{\"u}tzen. 

Unsere App ist als Quiz-basiertes Spiel realisiert, die vor allem ihren Fokus auf die Erkennung von Phishing-URLs legt.
Um die Wirksamkeit der App zu bewerten, evaluieren wir unsere Anwendung in Form einer Benutzerstudie.
Die Studienergebnisse zeigen, dass unsere App den Benutzern hilft bessere Entscheidungen {\"u}ber die Legitimit{\"a}t von URLs zu treffen.
\end{abstract}

\selectlanguage{american}
\begin{abstract}
Scammers discover the Internet as a convenient place for their criminal activities. 
For instance, they send Internet users spoofed e-mails which link to fraudulent websites. 
These websites prompt visitors to enter their confidential data.
This kind of Internet fraud is referred to as phishing. 
%For a victim of a phishing attack the consequences can be of an economic as well as an emotional nature. 
There exist multiple technical solutions to approach the problem of phishing which, for example, warn the users against accessing a revealed phishing website. 
Yet, they all cannot guarantee 100\% accuracy since there will always be ways to circumvent these techniques.
Moreover, security warnings or indicators of such approaches are not always recognized or even ignored by some end users.
For these reasons, a complementary approach is required.
Previous approaches do not draw on a crucial factor to combat the threat - the users themselves. 
Therefore, the increase of security awareness and especially user education about the dangers of the Internet is a further key strategy to combat phishing. 

Our master thesis aims at developing a smartphone app, which increases security awareness and educates the user regarding the detection of phishing.
To increase security awareness, the users send themselves a ``spoofed'' e-mail right away when starting the app for the first time.
%This is supposed to exemplarily illustrate them how trivial e-mail spoofing is and to increase their motivation and engagement, ultimately leading to better knowledge retention.
The user education part entails alerts regarding known techniques of attackers and is supposed to assist the users to internalize these with the aid of practice and repetition.
By this means, we aspire to help the users achieve the capability of defending themselves against phishing attacks in the future.

In detail, our app is realized as a quiz based game which mainly focuses on the detection of phishing URLs.
In order to evaluate the effectiveness of the app a user study is conducted.
The study outcomes show that our app helps users make better decisions regarding the legitimacy of URLs.
\end{abstract}

\cleardoublepage