\selectlanguage{ngerman}
\begin{abstract}
%TODO:Deutschen Abstract anpassen. Siehe englischer abstract
Immer mehr Betrüger entdecken das Internet als Platz ihres Handelns. Dazu
versenden sie gefälschte Emails oder veröffentlichen gefälschte Webseiten, die
den Benutzer auffordern private Daten einzugeben. Dieses Vorgehen nennt man
Phishing. Dieses kann für die Opfer sowohl wirtschaftliche als auch private
Folgen haben. Neben technischen Verbesserungen ist auch eine ausreichende
Schulung der Benutzer über die Gefahren und Risiken im Internet ein wichtiger
Bestandteil einer Strategie gegen Phishing.
Unsere Master-Arbeit hat zum Ziel eine Smartphone-Anwe<wbr>ndung zu entwickeln, die
dem Benutzer Hilfestellungen an die Hand gibt, um in Zukunft seltener Opfer von
zu werden. Dazu macht sie den Benutzer mit Methoden der Angreifer
bekannt und hilft ihm durch Üben diese zu verinnerlichen.
\end{abstract}

\selectlanguage{american}
\begin{abstract}
Scammers discover the Internet as a convenient place for their criminal activities. 
For instance, they send Internet users spoofed e-mails or publish fraudulent websites which prompt users to enter their confidential data. 
This kind of Internet fraud is referred to as phishing. 
For a victim of a phishing attack the consequences can be of an economic as well as an emotional nature. 
There exist multiple technical solutions to approach the problem of phishing. 
Yet, they all cannot guarantee 100\% accuracy.
Moreover, sometimes security warnings or indicators of such approaches are ignored by end users.
For this reason a complementary approach is required.
We believe that the increase of security awareness and especially user education about the dangers of the Internet, is a further key strategy to combat phishing. 
Our master thesis aims at developing a smartphone app, which increases security awareness and educates the user regarding phishing.
To increase security awareness the users send themselves a spoofed e-mail right away when starting the app for the first time.
This shall exemplary illustrate them how trivial e-mail spoofing is and is intended to increase their motivation and engagement ultimately leading to better knowledge retention.
The user education part entails alerts regarding known techniques of attackers and helps them to internalize these by practice and repetition.
In detail, our app is realized as a quiz based game which mainly focuses on the detection of phishing URLs.
Ultimately, the app should enable the users to achieve the capability of defending themselves against phishing attacks in the future, in case technical solutions should fail.
In order to evaluate the effectiveness of the app a user study is conducted.
The study outcomes shows that our app in general receives positive feedback from the users and also helps them make better decisions on whether a given URL is legitimate or fraudulent.
\end{abstract}

