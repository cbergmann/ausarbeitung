\begin{abstract}	
Scammers discover the Internet as a convenient place for their criminal activities. 
For instance, they send Internet users spoofed e-mails or publish fraudulent websites which prompt users to enter their confidential data. 
This kind of Internet fraud is referred to as phishing. 
For a victim of a phishing attack the consequences can be of an economic as well as a private nature. 
In addition to technical solutions which are not completely accurate, a complementary approach, i.e. user education about the dangers of the Internet, is a key strategy to combat phishing. 
Our master thesis aims at developing a smartphone app, which makes important information and tips available to its users hoping that they may achieve the capability to defend themselves against such attacks in the future.
For this purpose, the app alerts the users regarding known techniques of attackers and helps him to internalize this by practicing.
In order to evaluate the effectiveness of the app  a user study is conducted.
The study outcomes shows that our app in general receives positive feedback from users and also helps them make better decisions on whether a given URL is legitimate or fraudulent.
\end{abstract}