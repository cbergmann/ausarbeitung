\selectlanguage{ngerman}
\begin{abstract}
Betrüger entdecken das Internet als einen geeigneten Ort für ihre kriminellen Aktivitäten.
Zum Beispiel schicken sie Internetnutzern gefälschte E-Mails oder veröffentlichen gefälschte Webseiten, die die Benutzer auffordern, ihre vertraulichen Daten einzugeben.
Diese Art von Internetbetrug wird als Phishing bezeichnet.
Solche Angriffe können für die Opfer sowohl wirtschaftliche als auch emotionale Folgen haben.
Es existieren mehrere technische Lösungen, die zum Ziel haben, das Problem von Phishing zu lösen.
Diese Ansätze können allerdings keinen rumdum zuverlässigen Schutz garantieren.
Darüber hinaus werden Sicherheitswarnungen und -hinweise solcher Ansätze von den Benutzern manchmal nicht wahrgenommen oder gar ignoriert.
Aus diesen Gründen ist ein komplementärer Ansatz erforderlich.
Wir glauben, dass die Erhöhung des Bewusstseins für Sicherheit und vor allem die Benutzeraufklärung über die Gefahren des Internets eine weitere wichtige Strategie zur Bekämpfung von Phishing ist.
Unsere Masterarbeit beschäftigt sich mit der Entwicklung einer Smartphone-App , die das Bewusstsein für Sicherheit erhöht und die Benutzer bezüglich Phishing-Erkennung unterrichtet und trainiert.
Zur Sensibilisierung des Bewusstseins für Sicherheit der Benutzer, senden diese sich selbst eine gefälschte E-Mail, direkt beim Start der App.
Auf diese Weise veranschaulichen wir beispielhaft, wie trivial es ist E-Mails zu fälschen. 
Die möglicherweise daraus resultierende Motivation und das Engagement können zusätzlich zu einem verbesserten Lernerfolg führen.
Der Trainingsteil der App beinhaltet Informationen und Warnungen zu bekannten Techniken der Angreifer und hilft den Benutzern, diese durch Übungen und Wiederholungen zu verinnerlichen.
Unsere App ist als Quiz-basiertes Spiel realisiert, die vor allem ihren Fokus auf die Erkennung von Phishing-URLs legt.
Letztlich sollte die App den Nutzern helfen die Fähigkeit zu erlangen, sich in Zukunft vor Phishing-Angriffen zu schützen, für Situationen, bei denen technische Lösungen scheitern. 
Um die Wirksamkeit der von uns entwickelten App zu bewerten, führen wir am Ende eine Benutzerstudie durch.
Die Studienergebnisse zeigen, dass unsere App in der Regel positives Feedback von den Anwendern erhält und ihnen hilft bessere Entscheidungen über die Legitimität von URLs zu treffen.
\end{abstract}

\selectlanguage{american}
\begin{abstract}
Scammers discover the Internet as a convenient place for their criminal activities. 
For instance, they send Internet users spoofed e-mails or publish fraudulent websites which prompt users to enter their confidential data. 
This kind of Internet fraud is referred to as phishing. 
For a victim of a phishing attack the consequences can be of an economic as well as an emotional nature. 
There exist multiple technical solutions to approach the problem of phishing. 
Yet, they all cannot guarantee 100\% accuracy.
Moreover, security warnings or indicators of such approaches are not recognized or even ignored by some end users.
For these reasons a complementary approach is required.
We believe that the increase of security awareness and especially user education about the dangers of the Internet, is a further key strategy to combat phishing. 
Our master thesis aims at developing a smartphone app, which increases security awareness and educates the user regarding phishing.
To increase security awareness the users send themselves a spoofed e-mail right away when starting the app for the first time.
This is supposed to exemplarily illustrate them how trivial e-mail spoofing is and to increase their motivation and engagement, ultimately leading to better knowledge retention.
The user education part entails alerts regarding known techniques of attackers and is supposed to help the users internalize these by means of practice and repetition.
In detail, our app is realized as a quiz based game which mainly focuses on the detection of phishing URLs.
Ultimately, the app should enable the users to achieve the capability of defending themselves against phishing attacks in the future, in case technical solutions should fail.
In order to evaluate the effectiveness of the app a user study is conducted.
The study outcomes show that our app receives positive feedback from the users and also helps them make better decisions on whether a given URL is legitimate or fraudulent.
\end{abstract}

