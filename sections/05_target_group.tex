%*******************************************
\section{Target Group}
%*******************************************
\label{s:target_group}
In this section we want to describe the targeted users for our app.
The main condition that must be met is that they can learn something from our app.
That means they are skilled enough to use the app and not to skilled so that they already know everything that the app tells them.
In detail this is modelized by the following conditions.
\begin{description}
\item[Attackabilty] The first precondition is that all our users must meet is that they are possible targets for phishing.
This means they must use the Internet.
They also should use the Internet often enough and have a common trust in the web so that they are in general willing to enter their personal data.
\textbf{DIVSI}
\item[Android users] The second precondition is that they should use a android smartphone.
Our evaluation shows that the app is also usable by iOS users but they are not the target group because they can not use the app on a regular basis.
\item[Language] The informative parts of our app are texts and they are written in german.
This means the target user should be able to read german texts.
\item[Motivation] The distribution plan for this app is to put it on the google play store and hope that users download and install it.
Therefore the target user must be willing to learn something about the internet.
\textbf{DIVSI} shows that some internet users are so sure about their knowledge that they are not willing to learn something.
We will not be able to reach these users.
\end{description}

After we have roughly decided what our target group is we wanted to be sure that we don't rule out to much of the population with these preconditions.
This means it is of no use to produce a app that is, after all, only of use for 1 \% of the population.
To proof this we looked at a big survey done by 
