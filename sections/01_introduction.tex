%*******************************************
%*******************************************
\section{Introduction}
%*******************************************
\label{s:introduction}

%===========================================
\subsection{Problem Statement}
%===========================================
Nowadays, a world without Internet is unimaginable for most people in developed countries.
As an example, 83\% of Germany's population has Internet access~\cite{globalfinance2012internetusage}. 
The benefits of the Internet are undeniable.
It enables global communication and fast provision of information.
Furthermore, it offers 24 hours of online shopping without leaving the house, meet new people or make charge free phone calls.
However, the Internet also entails disadvantages with respect to its threats it brings.
One major issue of today's digitalized world is phishing. 
Phishing is a form of fraud which lures users into disclosing confidential information, usually over fake websites, even though many definitions exists depending on the given scenario or technique (cf.~\autoref{s:phishing_general}).
Commonly, the phisher sends out a tremendous volume of e-mails to random users which contain links to those malicious websites.
According to the website Dr. Dobb's, for example, every day 500 million phishing e-mails are delivered to user inboxes~\cite{drdobb2012email}.

Several technical solutions to counter phishing have already been proposed in literature.
These include, but are not limited to, e-mail spam filters, URL blacklists, or website takedowns.
These systems intend to protect the user from the enormous amounts of phishing attempts. 
Spam filters, for example, sort out phishing e-mails before they even reach the receiver~\cite{bergholz2010new,chandrasekaran2006phishing,fette2007learning}.
The major drawbacks of spam filters are that phishers are constantly improving their techniques to circumvent these and that the strength of the filter controls the amount of false positives and negatives. Thus, it is possible that phishing e-mails can make it through these filters and might harm the user.
An alternative approach are browsers restricting the access to resp. displaying warnings before accessing phishing websites with the aid of so called URL blacklists~\cite{ma2009beyond, zhang2008highly}.
The major downside of blacklists is that most of them work reactively.
Thus, there is a certain time frame where phishing websites are active without being blacklisted.
In this time frame users can access fraudulent websites without being warned or restricted and thus are susceptible to an attack.
Even though some dynamic and predictive approaches have been proposed~\cite{prakash2010phishnet, obied2009fraudulent, balzarotti2012proactive}, similar to spam filters, there will always be malicious websites which can bypass protective systems (false negatives).
 Finally, there is the weakest link in the security chain: there exist users who ignore security warnings and thus remain susceptible to phishing and other threats.
A field study conducted by Akhawe et al.~\cite{akhawe2013alice} revealed that 10\% of Mozilla Firefox's and 25\% of Google Chrome's malware and phishing warnings are clicked through, i.e. ignored.
 As a matter of fact, such systems are rendered superfluous for users who disregard such warnings.
Finally, certain parties commonly urge hosting providers to take down revealed malicious websites.
Such parties include, for instance, banks, other organizations or specialized takedown companies~\cite{moore2007examining}.
 The removal of phishing websites is an effective solution, since it implicitly solves the aforementioned problem, where users ignore security warnings: a removed website cannot trick a user into entering sensitive data.
Yet, according to Moore et. al~\cite{moore2007examining} the average life time of a phishing website is still 61.69 hours, i.e. 2.5 days.
Thus, this approach cannot entirely defeat phishing. During the uptime of fraudulent websites falling for them remains a threat.
Obviously, the technical solutions provide protection to a certain degree but cannot assure an overall protection from phishing for two reasons:
First, the technical solutions can never provide 100\% accuracy.
Second, even if a system detects a phishing website, there is still the user who might ignore the warning of the system~\cite{akhawe2013alice}.

The insufficiency of technical approaches and the problem that is raised by phishing is also reflected in many statistics of various reports. 
 According to the Anti-Phishing Working Group~(APWG) approximately 40,000 unique phishing websites are detected each month~\cite{antiphishingtrendreport2013}. Statistics published by Kaspersky Lab, a well-respected provider for IT security solutions, state that from year 2011-2012 to 2012-2013 the number of attacked users increased by about 87\%. 
While in 2011-2012, 19.9 million users were subject to phishing attempts, in 2012-2013 the numbers climbed up to 37.3 million. 
 Every day about 100,000 Internet users fall victims to phishing attacks, which is twice as much compared to the previous period. An immense increase can also be observed in the number of unique attack sources (i.e. IP addresses), which has tripled from 2012 to 2013~\cite{kasperskyreport2013}. 
Finally, RSA and ECM estimate worldwide costs caused by phishing at about \$1.5 billion for the year of 2012~\cite{rsa2013}. 

Note that according to Moore et al.~\cite{moore2010hard} phishing statistics might be inherently biased. 
The problem is, there are several ways to interpret collected data. 
Hence, every party might assess their data with respect to their interests resulting in diverse statistics. 
Diversity can also result from setting different foci.
Therefore, the reliability of such statistics, including the ones mentioned above, is questionable. 
Anderson et al.~\cite{anderson2012measuring} tried to give an independent view on this topic.
%We cannot know whether the disparity between several statistics is the result of different foci or of personal interest. 
Regardless of the reliability and accuracy of the above mentioned statistics it is undeniable that the problem of phishing is prevalent and needs to be additionally addressed with other approaches than technical solutions since they do not seem to suffice.

 %-------------------------------------------
 \subsection{Security Awareness and User Education}
 %-------------------------------------------
 \label{s:awareness}
We briefly introduced common technical solutions which intend to protect users from phishing attacks and discussed their downsides.
We furthermore supported these drawbacks by providing statistics which show that the phishing problem is not resolved with these pure technical solutions.
In the following we summarize the two major issues of the technical approaches and motivate the need for complementary strategies.
\begin{enumerate}
	\item\textit{Accuracy of Technical Solutions:} First, attackers can always invent new, more sophisticated deceptions that bypass current prevention systems.
	 The attackers are always first in row, i.e. they create a deception technique and once it is captured and resolved by detection systems, they simply create a new technique or adapt the old one so that it is no longer detected.
	 Second, there will always be false negatives.
Therefore, users should not rely on technical solutions only. 
Otherwise there is still the chance that users will fall into the attackers' traps whenever the technical assistance fails.
	\item\textit{User Behavior and Knowledge:} Another major problem with approaches to counter phishing is user behavior.
 As indicated above users tend to overlook or deliberately ignore security warnings.
 If the user behavior does not change such approaches will remain unhelpful for those who do not take them seriously.
 The problem is that users primarily make use of the Internet for purposes like online shopping, online banking, communicating with relatives and friends etc.
 Aspects related to security are not of their primary interest.
%or they just implicitly assume the system to be secure.
 Another factor for overlooking and ignoring these warnings might be the lack of security awareness~\cite{akhawe2013alice}.
 Some users might just not be aware of how easy it is for even unexperienced attackers to duplicate a website or send out fake e-mails on behalf of trusted companies or persons.
 Even if users are aware that there is a certain degree of threat in the Internet, people tend to believe the probability of facing such an attack is very low and that it will not happen to them, until it actually happens to them or to relatives/friends.
\end{enumerate}
For these reasons, we believe that a complementary approach to technical solutions is required.
We regard the raising of user security awareness and the offering of a service for education as a further key step against phishing.
Increased security awareness may change user behavior and attitude towards taking the warnings of protective tools more seriously.
The user education can help users defend themselves in cases such technical tools fail or also in cases were no tools are available.
The opinions on whether user education and increased security awareness will help combat phishing are divided among researchers.
There exist security researchers and experts who argue that user education is pointless~\cite{useredupointless, bruceschneieronsecuritytraining}.
Other sources emphasize the need for increased security awareness and education of the users~\cite{usereducebit, usereduscmagazine}.
It also seems that there already exist promising and effective anti-phishing education approaches~\cite{kumaraguru2007protecting, sheng2007antiphishingphil}, yet with the need for further improvements which we discuss in \autoref{s:related_work}.
Ultimately, we believe that technical solutions will never suffice to protect the end user entirely.
Therefore, there is the need for complementary approaches.
The user needs to learn that he has to protect himself and how he can achieve this protection.
 
%===========================================
\subsection{Goals}
%===========================================
\label{s:goals}
This section summarizes the primary goals of this thesis.
The major goal of this work is to offer users a service which educates them about phishing so that they are less likely to fall for fake webpages in the future.
 We think that the following steps are important to achieve this goal.

\begin{enumerate}
	\item Increasing the users' security awareness to complement pure technical solutions for countering phishing attempts.
	\item Elaborate on a both an appealing as well as valuable approach that educates the user with the capabilities required to identify phishing websites.
	\item Evaluate the effectiveness of the approach in a final user study.
\end{enumerate}
The lack of the users' security awareness seems to be a major issue concerning their security related behavior (cf. \autoref{s:awareness}).
 For this reason we want to raise the users' security awareness hoping that this will increase their attention while it will decrease their vulnerability.
 Moreover, besides technical solutions and increasing user awareness, it is important to provide the users information so that they can protect themselves against phishing attacks in the future.
Finally, we want to evaluate the effectiveness of our approach in a user study with respect to whether it actually can help the user protect themselves and how well it is received by them.

%===========================================
\subsection{Outline}
%===========================================


This thesis consists of ... main chapters: .... Their purpose is as follows:

Chapter 1 motivates this work.
..

Chapter 2 ...

Chapter 3 ...

...

Chapter ... finally summarizes this work and provides an outlook on future work.


