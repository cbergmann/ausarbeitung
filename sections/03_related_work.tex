\selectlanguage{american}

%*******************************************
\section{Related Work}
%*******************************************
\label{s:related_work}

This chapter deals with previous work on anti-phishing education.
 For better readability and comprehensibility we divided the related work we have found in literature into two categories: the \textit{content}, i.e.
 what the user is taught, and the 
%WHAT WAS THE USER TOLD$ and the 
\textit{medium}, i.e. how the user is taught.
 In the following, we will provide an overview of this content and medium classification.
 Subsequently, we will provide examples of previous work.

%HOW WAS THE USER TOLD ABOUT THE WHAT%. %EVENTUELL ENUM OR SO

%============================================
\subsection{Content Classification}
%============================================
The objective of this section is to introduce the different classes of learning content which we could identify in previous work.


\begin{description}[leftmargin=0cm]
	\item[General Knowledge Transfer] Renowned and targeted websites, such as PayPal, eBay or Microsoft provide general and superficial information about phishing~\cite{generalknowledgemicrosoft, generalknowledgepaypal, generalknowledgeebay}.
	Usually, they deal with questions like what is phishing, how does phishing happen, what the symptoms of phishing are and how to report phishing attempts.
 Providing the user only with text to the topic of phishing makes it possible to communicate any kind of content, so that the learning objectives can get as complex as one wishes.
 However, it is likely that users do not like reading too much, especially when it gets complex and difficult to comprehend.

	\item[E-Mail Based Knowledge] In this class of content, the user is told about the ``anatomy'' of phishing e-mails~\cite{antiphishingphyllis, sonicwall}. Particularly, they are informed about what kind of hints in an e-mail give indications for a phishing attempt.
 Indications for a phishing e-mail can be impersonal salutation, requesting personal and confidential information as well as exerting pressure and threatening the user with, for example, account closure.
 The benefit of detecting phishing attempts before even clicking on a link in an e-mail is that the user would not confirm the existence and active usage of his e-mail address to the phisher.
 More importantly, the user would not unknownlingly download malicious software.
 The problem with the e-mail based approach is that detecting phishing e-mails by looking at their content becomes more and more difficult~\cite{microsoftphishing,spamfighter}. Even if today still many phishing e-mails exhibit the obvious characteristic of having no personal salutation or being urgent and threatening, even today we observe a growing number of phishing e-mails that don't make these mistakes and it is likely that these obvious hints will not remain in future.

	\item[URL Based Knowledge] Sending spoofed e-mails with links to fake websites is a common trick of phishers.
 On the target website then, the user is lured to disclosing his credentials.
 Thus, detecting such fake websites is another possibility to protect oneself against phishing.
 Here the user is taught to distinguish phishing URLs from legitimate ones~\cite{sheng2007antiphishingphil, arachchilage2012designing}. Links to phishing websites are not only distributed by phishing e-mails.
 Such links can be spread via any communication channel.
 It is even possible to land on a phishing website by just surfing.
 Thus, for these cases knowing how to determine whether an e-mail is fake or legitimate is of no use.
 In these situations knowing how to distinguish phishing URLs from valid ones will help.
 The problem with this approach is that as soon as the DNS or host file is attacked, cf.~Section~\ref{s:phishing_techs}, even for experts it will get difficult to distinguish a phishing website from the legitimate one.
 Also it is unlikely that the user is checking the URL after each click. So the user must develop a strategy when to check the URL (e.g. before entering personal data) and when not.

\end{description}




%============================================
\subsection{Medium Classification}
%============================================
The objective of this section is to introduce the different classes of learning media which we could identify in previous work.

\begin{description}[leftmargin=0cm]
    \item[Simple Text] The simplest way of transferring knowledge to the user is to just write text about it.
    This is the most researched kind of medium and generations over generations pedagogues have researched and improved this medium.
    Alone in this medium there are multiple genres and subgenres which all might be used to transfer knowledge.
    The main problem is that in modern time many people see the simple text as old fashioned and prefer more interactive learning approaches. 
    
	\item[Game Based Learning] Therefore another way to communicate the learning content to the user is to use a traditional game.
 Such a game usually has a ``background story'' and a ``mission'' the user has to accomplish~\cite{sheng2007antiphishingphil,antiphishingphyllis}. The game design is important and depends on the target group.
 Previous work, for example, has focused on a fish as starring role in their game, cf.
~Section~\ref{s:prev_work}. This might work well for a target group of young age, but will most likely not be appealing to a larger audience.
 This is also reflected by our prestudy, cf.
~Section~\ref{s:prestudy}.
	\item[Quiz Based Learning] The quiz based approach is a form of a game which relies on a question-answer cycle without using a specific background story~\cite{onguardonline}. The advantage of a quiz based approach is that it seems more appropriate for adults and thus will likely be appealing to a larger audience.

	\item[Comparison Based Learning] A further way to teach users is to let them compare legitimate websites, URLs or e-mails with fake ones.
 Here the user has to decide which of the shown examples are the secure ones~\cite{staysafeonline}. We believe that this form of learning would increase the user awareness, as with this approach one could visualize to the user how difficult it can be to distinguish an original from a fake, since they look almost identical.
 However, this way of learning does not reflect the reality, which is a major drawback in our point of view.
 In real life the user does not have the luxury of chosing between two options, he has only one and has to decide whether this option is trustful or not.

	\item[Emdedded Learning] The aim of embedded learning is to educate the user on the topic of phishing during his every day life.
 For this reason the user is sent simulated phishing e-mails.
 In case the user falls for this simulated phishing attempt he is notified and gets more information regarding phishing and how to protect himself~\cite{embedded2011jansson, kumaraguru2009phishguru}. This approach benefits from the so called ``teachable moment''. The moment the user realizes that he has almost become a victim to a phishing attack, he will be highly motivated to prevent this happening again and thus be highly receptive for input related to this topic.
 However, the missing positive feedback is a major flaw of this strategy.
 The user is only notified in case of a mistake and not in case he has successfully rejected to react to the simulated phishing e-mail.
 A further problem is raised with the realization of such an approach.
 Legal issues will arise when sending simulated phishing e-mails which claim to come from a reputable vendor, for example, Amazon.

\end{description}

%============================================
\subsection{Previous Work}
%============================================
\label{s:prev_work}
In the previous section we have introduced you to the different classes of learning contents and communication media for better readability and comprehensibility. 
This section summarizes previous work on anti-phishing education. 
Furthermore, we point out our contributions which were not covered by previous work.
%...............................................................................................................
\subsubsection{Game and URL Based Approaches}
%...............................................................................................................
Anti-Phishing Phil is a game based approach~\cite{sheng2007antiphishingphil}. 
The three main objectives of this game are the following: 
(1) learn to detect phishing URLs, (2) learn where to look for indications in browsers for trustworthy/untrustworthy websites, and (3) learn to use search engines to find legitimate websites. 
The major focus, however, is set on the detection of phishing URLs. 
The main character of the game is a little fish, named Phil, who has to grow to a big fish by eating worms. 
These worms can either be good, i.e. real worms, or bad, i.e. fake worms, with which fishers try to hook the fish of the sea. 
Good worms of the game are associated with URLs of legitimate websites, while bad worms are associated with the URLs of phishing websites. 
Phil's task is to feed on legitimate URLs only. 
He must reject phishing URLs to grow to a big and healthy fish. 
The game consists of four rounds in total, each round endures two minutes. 
For correct actions Phil is rewarded with a certain amount of points. 
If Phil falsely rejects a legitimate URL, he is slightly penalized by having the time left decremented for a couple of seconds. 
However, if Phil eats a phishing URL he is severely penalized by losing one of three lives. 
In this way, the authors try to simulate the real world effects of their behavior. 
Each round the focus of deception techniques is shifted and phishing URLs get more difficult to identify. 
In the first round the users get introduced to IP address URLs. 
The second round mainly deals with deceptive subdomain URLs, where the brand name occurs in the subdomain. 
In the third round, the users are taught about similar and deceptive domains. 
In the last round finally, the user has to deal with all kinds of deceptions he has dealt with so far. 
The information material provided to the user is delivered by so called training messages. 
Anti-Phishing Phil features four kinds of training messages. 
First, the user gets direct feedback during the game, whether the answer he has given is correct or not and why. 
Second, the user has the possibility to receive help in case he needs it. 
In this case, Phil's experienced father will give a tip. 
Third, at the end of each round a score sheet is displayed, which summarizes the user's answers, whether they were correct or wrong, and why they were correct or wrong. 
Finally, there are anti-phishing tips in-between the rounds. 
To evaluate the effectiveness of the game the authors conducted a between-subjects experiment with three training conditions: 
(1) existing training material, e.g. from eBay or Microsoft, (2) anti-phishing tutorials which were created based on the game, and (3) the game itself. 
Each group had to decide on ten websites (in total 20) about their authenticity before and after the training step. 
The results showed that the participants in the game condition performed better than those in the other two conditions. 
All in all, we believe that the approach is a good step towards user education and features many good aspects. 
In the first place, the game based approach is an attractive incentive. 
Furthermore, the training messages are kept short and simple. 
Finally, the training messages, especially of the categories of help during the game and the score sheet after each round are very valuable. 
Due to time restrictions we could not integrate those kind of messages. 
However, we believe that this approach has some flaws and thus is not optimal for user education. 
Even though, using a fish as main character for this game is a funny idea, we do not think that this is an appropriate solution for adults. 
This is also reflected by the results of our pre-study, cf.~Section~\ref{prestudy}. 
As aforementioned, the training messages are simple and easy to understand, however, we are afraid that the phrasing is too vague. 
For example, for IP address URLs the following alert message is displayed to the user: 
"Don't trust URLs with all numbers in the front". 
For subdomain attacks the following wording is used "Don't be fooled by the word ebay.com in there, this site belongs to ttps.us". 
These kind of messages are susceptible to misinterpretation. 
Another downside we see, which is ultimately related to the vague formulations, is that the user is not concretely explained how he has to parse the URL in order to make healthy decisions on the authenticity of such. 
Here again he is only told that the most important part of the URL is between the "https://" and "/" and that the name of the website is the text right before the "/". 
In our point of view this is a vague phrasing and there is a lack of emphasizing the importance of the domain, which we do in our solution. 
Finally, the game does not cover some spoofing techniques, which are still relevant in our opinion, cf.~Section~\ref{phishing_url_categories}, and thus covered by us. 
For example, the difference between http and https is not introduced, as well as the fact that https websites can also be phishing websites. 
Furthermore, the game does not explicitly mention that the domain name, the host or even the entire URL can be part of the path to fool the user. 
Finally, there are different ways of making use of deceptive domains, which were not explicitly covered in Anti-Phishing Phil. 
For example homographic attacks which are detectable by the human eye, typos or scrambled letters, which should be distinguished in our opinion, in order to exemplify to the user how mean and hard to detect such URL spoofing techniques can be. 
There exist further proposals which are very similar to Anti-Phishing Phil~\cite{arachchilage2011designing,arachchilage2012designing}

%...............................................................................................................
\subsubsection{Game/Quiz and E-Mail Based Approaches}
%...............................................................................................................
Anti-Phishing Phyllis~\cite{antiphishingphyllis} is a game based approach and focuses on teaching the user to detect a variety of phishing traps in e-mails. 
These include, for example, fake links, attractive offers, urgent requests or malicious attachments. 
The main character of this game is a fish named Phyllis. 
Phyllis has to decide whether potential traps (marked with red bubbles) in an e-mail he is shown are real phishing traps or are harmless by disarming or ignoring them. 
The playing user gets hints during the game and direct feedback on his actions. 
SonicWALL provides an IQ test on phishing e-mails~\cite{sonicwall}. 
The user is shown e-mails consecutively and has to decide whether the displayed e-mail is legitimate or not. 
The user does not receive direct feedback on his decisions. 
At the end he receives an overview of the answers he has given and whether they were correct. 
If the user wants to know why his answer was correct or wrong he has to click on a link to get this information. 
As aforementioned, teaching users to detect phishing e-mails before even giving them the possibility to land on phishing websites has the advantage that they do not confirm the activeness of their e-mail address, and more importantly do not have the chance to accidentally download malicious software. 
However, as phishing e-mails become more and more sophisticated, i.e. convincing and credible, and since phishing websites are also reachable via SMS, online social networks or just surfing in the Internet, we did decide against the e-mail based approach. 

%...............................................................................................................
\subsubsection{General Knowledge Transfer With Embedded Learning}
%...............................................................................................................

There are several proposals in literature for embedded learning~\cite{jannson2011simulating, kumaraguru2009phishguru,alnajim2009antiphishing}. 
Jansson et al. proposed a solution where simulated phishing e-mails with links to fake websites or malicious download attachments are sent out to users~\cite{jannson2011simulating}. 
The moment a user falls for a trap of these simulated e-mails he receives a notification to inform him that he could have fallen for a real phishing attempt. 
Also, the e-mail includes a link to a website with a training program with general information and tips on how to detect phishing and malicious attachments. 
After consulting the training program the user is asked to complete a questionnaire in order to verify whether he received the messages of the training program. 
A very similar approach, the so called PhishGuru, is proposed by Kumaraguru~\cite{kumaraguru2009phishguru}. 
Another possibility is to leave out the step where simulated phishing e-mails are sent to users. 
Instead real phishing e-mails are made use of. 
For example, the APWG and Carnegie University's CyLab Usable Privacy and Security Laboratory (CUPS) work on the project "Phishing Education Landing Page"~\cite{apwg2009landingpage}. 
The moment a user clicks on a link of a real phishing website which has already been taken down, i.e. the moment the user behaves riskily, he is redirected to the anti-phishing landing page.
There he is told that he had almost become a victim of phishing and provided with educational material to this topic. 
Finally, there is an approach where the intervention does not happen after clicking on a dangerous link, but while surfing~\cite{alnajim2009antiphishing} instead. 
When the user lands on a blacklisted phishing website and is about to disclose his sensitive data (i.e. presses the submit button) the system interferes: 
the user is warned and given tips on how to detect phishing websites (e.g. abstract information on the detection of spoofed URLs). 
All of these solutions benefit from the so called teachable moment, the moment the users place themselves at risk by either clicking on a link in a (simulated) phishing e-mail or by submitting sensitive information to blacklisted phishing websites. 
This moment of risk presents a teachable moment for those who almost fell for such a trap. 
For this reason giving the warnings, hints and training to the user in this moment will most likely result in higher motivation and retention so that the tips are more likely to help avoid similar dangers in future. 
A downside of these approaches is that they do only give negative feedback to the user. 
Consequently, the user is not "rewarded" when he rejects to click on a phishing link or to submit data on a phishing website, which is an important thing to do in our view. 
Moreover, the amount of information provided on such an educational website should be reasonable, i.e. the user should not be flooded with information. 
Otherwise he will not retain or even consult everything. 
To overcome these issues, a reasonable consideration for future work might be to combine embedded learning with another approach, for example, playing an educational game. 
In this way positive feedback can be included and the information can be transferred bit by bit to the user. 
For example, the website the user is redirected to might contain just the most important information, just enough to motivate the user to click on the provided link to our final app, in case the user is interested in gaining in-depth insight on this topic. 
For now, we do not follow this approach since the step of sending simulated phishing e-mails to users raises legal issues.

%...............................................................................................................
\subsubsection{Comparison and URL Based Approach}
%...............................................................................................................
Symantec offers a "race to stay safe"~\cite{staysafeonline}, where the user is shown two snapshots of two websites, while one website is a fake and the other is a legitimate one. 
Within very short time the user has to compare the snapshots and decide which way is the safe one to go. 
The focus of this training is set on the URL and address bar. 
We believe that such an approach is likely to increase the user awareness of how deceptively similar phishing websites can be to the original ones. 
However, the approach of comparing two websites is not realistic enough, since the user does not have two websites and does not have the option to choose between them in reality. 
This is why we did not decide for the comparison based approach. 
However, adding time pressure to our approach, i.e. simulating a real life situation, is an aspect which might be worth to consider for future work.
