\selectlanguage{american}

%*******************************************
\section{Related Work}
%*******************************************
\label{s:related_work}

This chapter deals with previous work on anti-phishing education. We divided the related work we have found in literature into two categories: the \textit{content}, i.e. what the user is taught, and the 
%WHAT WAS THE USER TOLD$ and the 
\textit{medium}, i.e. how the user is taught. In the following, we will provide an overview of this content and medium classification. Subsequently, we will provide examples of previous work.
%HOW WAS THE USER TOLD ABOUT THE WHAT%. %EVENTUELL ENUM OR SO

%============================================
\subsection{Content Classification}
%============================================
The objective of this section is to introduce the different classes of learning content which we could identify in previous work.
\textbf{ICH BIN MIR UNSICHER OB MELANIE DIESE KLASSIFIZIERUNG GEFIEL. VIELLEICHT SOLLTEN WIR TATSCÄHLCIH EINFACH NUR RUNTERZÄHLEN WAS ES SO GIBT UND AUF VOR UND NACHTEILE EINGEHEN}
\begin{description}[leftmargin=0cm]
	\item[General Knowledge Transfer] Renowned and targeted websites, such as PayPal, eBay or Microsoft provide general and superficial  information about phishing~\cite{generalknowledgemicrosoft, generalknowledgepaypal, generalknowledgeebay}. Usually, they deal with questions like what is phishing, how does phishing happen, what the symptoms of phishing are and how to report phishing attempts. Providing the user only with text to the topic of phishing makes it possible to communicate any kind of content, so that the learning objectives can get as complex as one wishes. However, it is likely that users do not like reading too much, especially when it gets complex and difficult to comprehend.
	\item[E-Mail Based Knowledge] In this class of content, the user is told about the ``anatomy'' of phishing e-mails~\cite{antiphishingphyllis, sonicwall}. Particularly, they are informed about what kind of hints in an e-mail give indications for a phishing attempt. Indications for a phishing e-mail can be impersonal salutation, requesting personal and confidential information as well as exerting pressure and threatening the user with, for example, account closure. The benefit of detecting phishing attempts before even clicking on a link in an e-mail is that the user would not confirm the existence and active usage of his e-mail address to the phisher. More importantly, the user would not unknownlingly download malicious software. The problem with the e-mail based approach is that detecting phishing e-mails by looking at their content becomes more and more difficult~\cite{microsoftphishing,spamfighter}. Even if today still many phishing e-mails exhibit the obvious characteristic of having no personal salutation or being urgent and threatening, it is likely that these obvious hints will not remain in future.
	\item[URL Based Knowledge] Sending spoofed e-mails with links to fake websites is a common trick of phishers. On the target website then, the user is lured to disclosing his credentials. Thus, detecting such fake websites is another possibility to protect oneself against phishing. Here the user is taught to distinguish phishing URLs from legitimate ones~\cite{sheng2007antiphishingphil, arachchilage2012designing}. Links to phishing websites are not only distributed by phishing e-mails. Such links can be spread via any communication channel. It is even possible to land on a phishing website by just surfing. Thus, for these cases knowing how to determine whether an e-mail is fake or legitimate is of no use. In these situations knowing how to distinguish phishing URLs from valid ones will help. The problem with this approach is that as soon as the DNS or host file is attacked, cf.~Section~\ref{s:phishing_techs}, even for experts it will get difficult to distinguish a phishing website from the legitimate one.
\end{description}

%HIER SCHON NACHTEILE ODER UNTEN BEI EXAMPLES ODER BEIDES?!


%============================================
\subsection{Medium Classification}
%============================================
The objective of this section is to introduce the different classes of learning media which we could identify in previous work.

\begin{description}[leftmargin=0cm]
	\item[Game Based Learning] One way to communicate the learning content to the user is to use the traditional game. Such a game usually has a ``background story'' and a ``mission'' the user has to accomplish~\cite{sheng2007antiphishingphil,antiphishingphyllis}. The game design is important and depends on the target group. Previous work, for example, has focused on a fish as starring role in their game, cf.~Section~\ref{s:prev_work}. This might work well for a target group of young age, but will most likely not be appealing to a larger audience. This is also reflected by our prestudy, cf.~Section~\ref{s:prestudy}.
	\item[Quiz Based Learning] The quiz based approach is a form of a game which relies on a question-answer cycle without using a specific background story~\cite{onguardonline}. The advantage of a quiz based approach is that it seems more appropriate for adults and thus will likely be appealing to a larger audience.
	\item[Comparison Based Learning] A further way to teach users is to let them compare legitimate websites, URLs or e-mails with fake ones. Here the user has to decide which of the shown examples are the secure ones~\cite{staysafeonline}. We believe that this form of learning would increase the user awareness, as with this approach one could visualize to the user how difficult it can be to distinguish an original from a fake, since they look almost identical. However, this way of learning does not reflect the reality, which is a major drawback in our point of view. In real life the user does not have the luxury of chosing between two options, he has only one and has to decide whether this option is trustful or not.
	\item[Emdedded Learning] The aim of embedded learning is to educate the user on the topic of phishing during his every day life. For this reason the user is sent simulated phishing e-mails. In case the user falls for this simulated phishing attempt he is notified and gets more information regarding phishing and how to protect himself~\cite{embedded2011jansson, kumaraguru2009phishguru}. This approach benefits from the so called ``teachable moment''. The moment the user realizes that he has almost become a victim to a phishing attack, he will be highly motivated to prevent this happening again and thus be highly receptive for input related to this topic. However, the missing positive feedback is a major flaw of this strategy. The user is only notified in case of a mistake and not in case he has successfully rejected to react to the simulated phishing e-mail. A further problem is raised with the realization of such an approach. Legal issues will arise when sending simulated phishing e-mails which claim to com from a reputable vendor, for example, Amazon.
\end{description}

%============================================
\subsection{Previous Work}
%============================================
\label{s:prev_work}
Previous work here ... (e.g. Anti-Phishing Phil and Phyllis)
%%%ANTI PHISHING LANDING PAGE ALS PREV WORK EXAMPLE EINBAUEN FÜR EMBEDDED LEARNING

%============================================
%\subsection{Open Questions}
%============================================
%\label{related_work:open_questions}