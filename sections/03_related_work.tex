\selectlanguage{american}

%*******************************************
\section{Related Work}
%*******************************************
\label{s:related_work}
In the previous section we introduced the different classes of learning contents and communication media. 
Furthermore, we have decided to go for the game/quiz based approach while focusing on URL based knowledge. 
This section summarizes specific examples of previous work on anti-phishing education.
For each class we previously introduced at least one example is illustrated. 
We especially elaborate on the game and URL based approach as this is the path we take for our approach. 
Moreover, we state in which way our work is to be distinguished from previous work. 

%...............................................................................................................
\subsection{Game and URL Based Approaches}
%...............................................................................................................
Several work has been done in this area~\cite{arachchilage2011designing,arachchilage2012designing}. 
However, most approaches are similar to Anti-Phishing Phil~\cite{sheng2007antiphishingphil} the approach we focus on in this section.
Anti-Phishing Phil is a game based approach focusing on URL based knowledge. 
We will extensively discuss this game since the approach we envision resembles Anti-Phishing Phil the most.

\begin{description}[leftmargin=0cm]
	\item[Game Design and Rules:] The three objectives of this game are the following: 
(1)~learn to detect phishing URLs, (2)~where to look for indications in browsers for trustworthy/untrustworthy websites, and (3)~learn to use search engines to find legitimate websites. 
The major focus, however, is set on the detection of phishing URLs. 
The main character of the game is a little fish, named Phil, who has to grow to a big fish by eating worms. 
These worms can either be good, i.e. real worms, or bad, i.e. fake worms, with which fishers try to hook the fish off the sea. 
Good worms of the game are associated with URLs of legitimate websites, while bad worms are associated with the URLs of phishing websites. 
Phil's task is to feed on legitimate URLs only. 
He must reject phishing URLs to grow to a big and healthy fish. 
The game consists of four rounds in total, each round takes two minutes. 
For correct actions Phil is rewarded with a certain amount of points. 
If Phil falsely rejects a legitimate URL, he is slightly penalized by having the time left decremented for a couple of seconds. 
However, if Phil eats a phishing URL he is severely penalized by losing one of three lives. 
In this way, the authors try to simulate the extent of the real world effects of their behavior, i.e.
in reality rejecting a valid URL is not as bad as trusting a phishing URL.
Each round the focus of deception techniques is shifted and phishing URLs get more difficult to identify. 
In the first round the users get introduced to IP address URLs. 
The second round mainly deals with deceptive subdomain URLs, where the brand name occurs in the subdomain of a URL. 
In the third round the users are taught about similar and deceptive domains. 
In the last round finally, the user has to deal with all kinds of deceptions he has dealt with so far. 
\item[Feedback:]
The information material provided to the user is delivered by so called training messages. 
Anti-Phishing Phil features four kinds of training messages. 
First, the user gets direct feedback during the game, whether the answer he has given is correct or not and why. 
Second, the user has the possibility to receive help in case he needs it. 
In this case, Phil's experienced father will give a tip. 
Third, at the end of each round a score sheet is displayed, which summarizes the user's answers, whether they were correct or wrong, and why they were correct or wrong. 
Finally, there are anti-phishing tips in-between the rounds.
\item[Game Evaluation:] To evaluate the effectiveness of the game the authors conducted a between-subjects experiment with three training conditions, represented by three groups: 
(1)~existing training material, for example, from eBay or Microsoft, (2)~anti-phishing tutorials which were created based on the game, and (3)~the game itself. 
Each group had to decide on ten websites (in total 20) about their authenticity before and after the training step. 
The results showed that the participants in the game condition performed better than those in the other two conditions. 
\item[Positive Sides:] 
All in all, we believe that the approach is a good step towards user education and features many good aspects. 
In the first place, the game based approach might be an attractive incentive for the user to be educated. 
Furthermore, the training messages are kept short and simple. 
Finally, the training messages, especially the ones of help during the game and the score sheet after each round are very valuable. 
Due to time restrictions we could not consider those kinds of messages. 
\item[Downsides and Our Contributions:] 
On the other hand, we believe that this approach has some flaws and thus is not optimal for user education. 
Even though using a fish as a main character for this game is a funny idea, we do not think that this is an appropriate solution for adults. 
This is also reflected by the results of our survey (cf. \autoref{s:survey}). 
Therefore, we do not use a fish as a main character. 
Our approach will rather be a combination of a game, which includes lives and points, and a quiz, where users are required to answer questions directly, without any background story.
As aforementioned, Anti-Phishing Phil's training messages are simple and easy to understand, however, we are afraid that the phrasing is too vague. 
For example, for IP address URLs Anti-Phishing Phil displays the following alert message to the user: 
"Don't trust URLs with all numbers in the front". 
For subdomain attacks the following wording is used "Don't be fooled by the word ebay.com in there, this site belongs to ttps.us". 
These kinds of messages are susceptible to misinterpretation. 
Another downside we see, which is ultimately related to the vague formulations, is that the user is not precisely explained how he has to parse the URL in order to make healthy decisions on the authenticity of such. 
Here again, he is only told that the most important part of the URL is between the "https://" and "/" and that the name of the website is the text right before the "/". 
In our point of view, this is a imprecise phrasing and there is a lack of emphasizing the importance of the domain, which we do in our solution. 
Furthermore, the game does not cover some spoofing techniques, which are still relevant in our opinion and thus are covered by us (cf. \autoref{s:url_categories}). 
For example, the difference between HTTP and HTTPS is not introduced, as well as the fact that HTTPS websites can also be phishing websites. 
Furthermore, the game does not explicitly mention that the domain name, the host or even the entire URL can be part of the path to fool the user. 
Finally, there are different ways of making use of deceptive domains, which were not explicitly covered by Anti-Phishing Phil. 
For example, homograph attacks (cf. \autoref{s:url_categories}), typos and scrambled letters should be distinguished in order to exemplify how mean and hard to detect such URL spoofing techniques can be.
\end{description}


%...............................................................................................................
\subsection{Game/Quiz and E-Mail Based Approaches}
%...............................................................................................................
Anti-Phishing Phyllis~\cite{antiphishingphyllis} is a game based approach and focuses on teaching the user to detect a variety of phishing traps in e-mails. 
These include, for example, fake links, attractive offers, urgent requests, or malicious attachments. 
The main character of this game is a fish named Phyllis. 
Phyllis has to decide whether potential traps (marked with red bubbles) in a given e-mail are real phishing traps or are harmless by disarming or ignoring them. 
The playing user gets hints during the game and direct feedback on his actions. 
Another quiz and e-mail based approach is provided by SonicWALL~\cite{sonicwall}. 
The user is shown e-mails consecutively and has to decide whether the displayed e-mail is legitimate or not. 
The user does not receive direct feedback on his decisions. 
At the end he receives an overview of the answers he has given and whether they were correct. 
If the user wants to know why his answer was correct or wrong he has to click on a link to get this information. 
As aforementioned, teaching users to detect phishing e-mails before even giving them the possibility to land on phishing websites has the advantage that they do not confirm the activeness of their e-mail address, and more importantly, do not have the chance to accidentally download malicious software or be lured into disclosing sensitive information. 
However, as phishing e-mails become more and more sophisticated, i.e. convincing and credible, and since phishing websites are also reachable via other communication channels, such as SMS, online social networks or just surfing in the Internet, we decided against the e-mail based approach. 

%...............................................................................................................
\subsection{General Knowledge Transfer with Quizzes}
%...............................................................................................................
There exist online quizzes where the user is asked general questions to the topic of phishing~\cite{icicibank,onguardonline}. 
The design of these online quizzes is based on the association of phishing with fishing. 
That is to say, a fish is the main character of the quiz, which we do not find appropriate for adult users. 
Moreover, the number and variety of the questions asked in these quizzes are very restricted. 
Even if the examples of the quiz based approaches are not optimal for user education, we think that this approach is the most appropriate one for adults as target group. 
This is also reflected by the results of our survey (cf. \autoref{s:survey}).

%...............................................................................................................
\subsection{Comparison and URL Based Approach}
\label{s:comparison_approach}
%...............................................................................................................
Symantec offers a ``race to stay safe''~\cite{staysafeonline}, where the user is shown two snapshots of two websites, while one website is a fake and the other one is genuine. 
Within very short time the user has to compare the snapshots and decide which way is the safe one to go. 
The focus of this training is set on the URL and address bar. 
We believe that such an approach is likely to increase the user awareness of how deceptively similar phishing websites can be to the original ones. 
However, the approach of comparing two websites is not realistic enough, since the user does not have two websites and does not have the option to choose between them in reality. 
This is why we did not decide for the comparison based approach. 
However, adding time pressure to our approach, i.e. simulating a real life situation, is an aspect which might be worth to consider for future work.


%...............................................................................................................
\subsection{General Knowledge Transfer with Embedded Learning}
%...............................................................................................................

There are several proposals in literature for embedded learning~\cite{embedded2011jansson, kumaraguru2009phishguru,alnajim2009antiphishing}. 
One of these is a solution proposed by Jansson et al. where simulated phishing e-mails with links to fake websites or malicious download attachments are sent out to users~\cite{embedded2011jansson}. 
The moment a user falls for a trap of these simulated e-mails he receives a notification informing him that he could have fallen for a real phishing attempt. 
Also, the e-mail includes a link to a website with a training program with general information and tips on how to detect phishing and malicious attachments. 
After consulting the training program the user is asked to complete a questionnaire in order to verify whether he understood the content of the training program. 
A very similar approach, the so called PhishGuru~\cite{kumaraguru2009phishguru}, is proposed by Kumaraguru. 
Another possibility is to leave out the step where simulated phishing e-mails are sent to users. 
Instead actual phishing e-mails are utilized. 
For example, the APWG and Carnegie University's CyLab Usable Privacy and Security Laboratory (CUPS) work on the project ``Phishing Education Landing Page''~\cite{apwg2009landingpage}. 
The moment a user clicks on a link of a real phishing website which has already been taken down, i.e. the moment the user behaves riskily, he is redirected to the anti-phishing landing page.
There he is told that he had almost become a victim of phishing and provided with educational material to this topic. 
Finally, there is an approach where the intervention does not happen after clicking on a dangerous link, but while surfing~\cite{alnajim2009antiphishing}. 
When the user lands on a blacklisted phishing website and is about to disclose his sensitive data, i.e. presses the submit button, the system interferes: 
the user is warned and given tips on how to detect phishing websites (for example, he is provided with abstract information on the detection of spoofed URLs). 
As discussed before, all of these solutions benefit from the so called teachable moment (cf. \autoref{s:medium_classification}).
A downside of these approaches is that they do only give negative feedback to the user. 
Consequently, the user is not rewarded when he rejects to click on a phishing link or to submit data on a phishing website, which is an important thing to do in our view. 
Moreover, the amount of information provided on such an educational website should be reasonable, i.e. the user should not be flooded with information. 
Otherwise he will not retain or even consult everything. 
On top of this, it might happen that the user does not understand the situation and just clicks the warning away~\cite{TUD-CS-2013-0167}.
In this case there would be no education at all and the approach would render superfluous.
To overcome these issues, a reasonable consideration for future work might be to combine embedded learning with another approach, for example, playing an educational game. 
For example, the website the user is redirected to might contain just the most important information, enough to motivate the user to click on the provided link to an educational game, for instance our app, in case the user is interested in gaining in-depth insight on this topic. 
In this way, positive feedback can be included and the information can be transferred to the user bit by bit. 
For now, we do not follow this approach as the step of sending simulated phishing e-mails to users raises legal issues.

%...............................................................................................................
\subsection{Further Game Based Approaches on Other Computer Security Topics}
%...............................................................................................................
Besides the proposals for user education on the specific topic of phishing, there exist a variety of other approaches aiming at educating the everyday user on general or other specific topics of computer security.
Auction Hero~\cite{chiasson2011auction}, for example, is a simulation game which covers different topics of computer security, amongst other phishing. 
Its aim is to help users make more secure decisions in the Internet by modeling their regular Internet behavior. 
Real life is simulated by making security a secondary goal of the game, like it usually is the case with end users. 
The primary goal of the user, who is a trader, is to build and sell robots, and earn enough money and reputation to ultimately become an "Auction Hero". 
As in reality, the trader has to pay attention to various security risks, such as weak account passwords, out-dated antivirus software as well as phishing. 
Phishing, in particular, is dealt with as follows: the playing user receives e-mails within the game. While some of them are legitimate others are not (for example, an e-mail saying that the user has won an auction for an item on which he has never bid). 
The e-mails include links to websites where the user is asked to enter his in-game login data. 
An ultimate consequence of disclosing data to a phishing website is that the user will suffer loss of money and reputation. 
Also, an explanatory warning will be displayed. 
The user is taught about typical characteristics of phishing, potential consequences of falling for them, and how to deal with phishing attempts. 
This approach has the major benefit of simulating actual online behavior and thus provides a realistic context for the user. 
Additionally, the user does not only learn about phishing, but other security related aspects, such as having strong passwords and keeping antivirus software up-to-date. 
On the other hand, we want to focus on phishing attacks in detail instead of giving the user an overview of security topics which have to be considered when using the Internet.
There exists a variety of other online games and quizzes covering miscellaneous topics of computer security.
"Mission Laptop Security", for example, is a quiz based approach where the user's mission is to transport a laptop to a specific destination in a secure manner~\cite{laptopsecurity}. 
During his trip, the user is asked several questions about how to act in different situations. 
The mission can only be completed if the user gives enough correct answers.
Another game covers the topic of network security~\cite{wirelesshackers}. 
Hackers, represented by little red men, are surrounding the user's WLAN area. 
By clicking on a hacker man a question appears. 
When the user gives the correct answer, this specific hacker man disappears. 
When the user gives an incorrect answer all red men come closer to the user's computer in the center of the WLAN area.
Others have diverged from computer based games and rather suggested a physical card game primarily intended to increase the users' awareness of needs and challenges related to computer security in general~\cite{denning2013controlalthack}.
A further interesting approach is to change sides.
That is to say, the user does not play the role of an unknowing user who has to defend himself from attacker's.
Instead he is the attacker and his goal is to profit from criminal activities.
Data Dealer~\cite{datadealer}, for instance, provides a game dealing with the topic of data privacy, data abuse and surveillance.
The player's, i.e. attacker's, goal is to collect private data of friends, neighbours or any other person, sell the collected data over the black market and set up companies.







 

