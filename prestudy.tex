%*******************************************
\section{Pre-Survey}
%*******************************************
\label{s:prestudy}
Before elaborating on the concrete app design we ran a small pre-survey. To the best of our knowledge there do not exist other surveys which resemble ours and additionally were conducted in Germany. This chapter deals with the main objectives of the pre-survey. Furthermore, it provides some details and finally presents the results and evaluates the questionnaire.

%============================================
\subsection{Main Objectives}
%============================================
Our main objectives of this pre-survey were twofold:

\begin{enumerate}
	\item \textbf{Awareness and Knowledge} One goal of the pre-survey was to comprehend what exactly Internet users understand under phishing. With a Likert scale we furthermore tried to figure out how they evaluate their on knowledge on the topic of Internet security.
	\item \textbf{Preferences of Users} Another purpose of the survey was to get an idea of the users' preferences with regard to an educational app. For example, they were asked whether they found a quiz based game appropriate for learning purposes.
\end{enumerate}
%============================================
\subsection{Survey Details}
%============================================
This section provides some details about our questionnaire, how we distributed it and how we filtered the surveys in order to consider our target group for the results and evaluation.
\textbf{SURVEY IN APPENDIX???}

\subsubsection{Questionnaire}
In the following we present the structure of our questionnaire and the function of each section.
\begin{enumerate}
	\item \textbf{General Information} In this section the participant is asked to provide information regarding his gender, age, his professional qualification as well as his field of study or work. The main purpose of this section is to exclude participants which do not fit into our target group.
	\item \textbf{Internet Usage} Here, the participant is asked how often he uses the Internet, whether he owns a smartphone and which applications he uses on his desktop computer and which ones he uses on his smartphone. This section is intended to give us an overview of the users' Internet usage and helps us to exclude participants who do not fit into our target group.
	\item \textbf{Self-Assessment} In this part of the survey, the participant has to indicate how much he agrees to the presented statements with the aid of a Likert scale. The statements mainly refer to their self-assessment regaring their knowledge about Internet security. For example, they have to assess, whether they think they have enough knowledge, to avoid the dangers of the Internet or whether they think it is easy for them to distinguish legitimate e-mails from fake ones.
	\item \textbf{Phishing} Here, the participant gets concrete questions to the topic of phishing. In particular, he is asked which services and which user information are endangered by phishing attacks. This section purposes to find out what the participants know about and think of phishing.
	\item \textbf{Anti-Phishing App} This section asks the user for his preferences regarding an anti-phishing education app. With the aid of a Likert scale he is requested to assess, for example, whether the would like having a game with a fish, or whether he finds a text-based approach meaningful as well as whether he would have fun with a question-answer quiz game.
	\item \textbf{Further Survey Progress} In this part of the pre-survey the user can provide us his e-mail address in case he wants to get information about the further progress of the survey or would like to test the app.
\end{enumerate}

\subsubsection{Distribution}
In total 253 persons participated in our pre-survey. We set up an online survey as well as asked students to fill out our printed survey. In the following we briefly explain our distribution process.

\begin{description}
	\item[Printed Survey] To reach participants for our printed pre-survey we contacted multiple professors and asked them whether we could have 10 minutes of their lecture time to have their students fill out our printed survey. Moreover, we asked our friends and parents whether they can ask their friends, colleague or customers to fill out the questionnaire.
	\item[Online Survey] The online survey was mainly distributed digitally. We contacted our friends and asked them to participate in the survey. We also demanded to forward the link to their friends so we could reach a wider range of people.
\end{description}

\subsubsection{Filtering for Evaluation}

The following Table~\ref{table:prestudy_filer} summarizes what kind of answers we used in order to exclude participants from the survey who do not fit into our target group.

\begin{center}
    \begin{tabular}{ | p{5cm} | p{10cm} |}
    \hline\textbf{Question} & \textbf{Filtering}  \\  \hline
		\hline\  Age & We consider all adults ranging from 18 - 65 years. \\
    \hline\  Gender & We do not exclude any gender. \\ 
    \hline\  Professional qualification & The participant does not have to exhibit a specific professional qualification to be considered for the results and evaluation. \\ 
		\hline\  Field of study/work & Students, employees or employers in the field of computer science or electrical engineering are filtered out as they do not belong to our target group. \\ 
	  \hline\ Frequency of Internet usage & Participants who have indicated ``rarely'' as the answer to this question do not belong to our target group and thus are filtered out. \\ 
	  \hline\ Used Internet applications  &  The listed applications include, for example, browser, e-mail, shopping as well as banking. Any service of the Internet is potentially endangered by phishing. For this reason we do not use this question to filter out participants.\\ 
    \hline\ Owning a smartphone  & With the app we particularly target smartphoner owners. For this reason participants who do not own any kind of smartphone are filtered out. \\
		\hline\ Used smartphone applications in the Internet  & The listed applications include, for example, browser, e-mail, shopping as well as banking. Any service of the Internet, especially on a smartphone, is potentially endangered by phishing. For this reason we do not use this question to filter out participants. \\
    \hline\ Number of received commercial e-mails per week  & We do not filter out any participant with this question. \\
    \hline\ Number of received e-mails asking for personal data  & We do not filter out any participant with this question. \\
    \hline\ User reads up on topics related to dangers in the Internet  &  Participants who have chosen ``no'' as answer are filtered out. We specifically target users who are interested in getting safer in the Internet. As the participants who have indicated ``no'' do not seem to have any interest in doing so, they will most likely do not show interest in our app. For this reason we regard them as not belonging to our target group and exclude them from the analysis and evaluation.\\
    \hline\   &  \\
		\hline\   &  \\
    \hline\   &  \\
    \hline\   &  \\
    \hline\   &  \\
    \hline\   &  \\


    \hline
    \end{tabular}
		
\end{center}
%%%TO ADD: CAPTION AND LABEL (UND OBEN REFERENZIEREN)

%============================================
\subsection{Results and Evaluation}
%============================================