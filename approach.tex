
%*******************************************
\section{Approach for Our Anti-Phishing Education App}
%*******************************************
\label{s:approach}
This chapter presents our final approach for the Anti-Phishing Education App. In the following sections we will elaborate on the app design in detail.
%===========================================
\subsection{App Design}
%===========================================
We have decided to divide our education app into two main parts. The first part of the education app is intended to increase the user awareness. The second part of the app then covers the actual educational part. The following enumeration summarizes the functions of our twofold app structure.


\begin{enumerate}
	\item \textbf{Awareness Part} The first part of the education app is intended to increase the user awareness regarding how easy it is to spoof e-mails and mislead users with such fake messages. This part is supposed to motivate the user to do something to counter the danger of the Internet, in this particular case, against phishing.
	\begin{enumerate}
		\item \textbf{Receive Fake E-Mail} We want to illustrate to the user how easy it is to spoof the "from" field of an e-mail as well as the content of this e-mail. For this purpose the user has to send a fake e-mail with an arbitrary sender address of his choice to himself. The user will also have the option to type in a free text. Upon submitting the form the user will receive an e-mail with the e-mail address he had indicated as sender. The free text will also be part of the received e-mail. We believe that the user will be very surprised about how easy even he himself could send a fake e-mail. The user will learn that he cannot fully trust the "from" field and the content of the e-mails he is receiving.
		\item \textbf{Linktext Unequal Target URL} The awareness part of the app is also supposed to show the user that he cannot trust the texts of a link he is clicking on. To illustrate this, the user is asked to click on a link with the text "https://www.google.de/". Clicking on this link, the user will expect to land on the Google website, what will not happen. In fact, the user is linked back to our app, where he is told that link texts are not trustful as well. 
	\item \textbf{Fake Website} Finally, the user is told that creating a copy of a website is also very easy. He is told that a reliable way to decide whether a website is a fake or not is to analyze the URL of the website he is visiting. He is also told that this app focuses on exactly this.
	\end{enumerate}
	\item \textbf{Educational Part} The second part of the app covers the actual educational part. Here, the user is learning about various spoofing techniques of the attacker.
	\begin{enumerate}
		\item \textbf{Information Material} The second part of the app is divided into levels of increasing difficulty. Here the user is first taught how to access and analyze the URL of the web browser. Subsequently, the user learns about the general structure of a URL. This is done in a very simplified way, so that even unexperienced users can follow. In particular, the user is told how to find the second- and top-level domain of a URL. In all succeeding levels the user is introduced to various URL spoofing techniques of a phisher. The learning content of each level can be consulted in \textbf{....(add label ref)}.
		\item \textbf{Exercise to Information Material} After every introductory material in each level an exercise section is followed. For the "access and analyze URL part", for example, the user is forwarded to a website. There he has to apply all important steps he has learnt in the introductory part. After successful completion of the tasks the user is linked back to the app. For the "find second- and top-level domain" information material the user gets a couple of valid URLs of which he has to identify the second- and top-level domains. All subsequent level exercises are structured as follows: the user is presented a URL. He has to decide whether the presented URL is a phish or a valid URL. If the URL is a phish, and the user has correctly identified the phish, the user has to show the second- and top-level domain. Only if the user correctly identifies the second- and top-level domain he receives the points for this URL, otherwise the answer to this URL is considered wrong, because we assume that the user has just guessed in this case.
		\item \textbf{Repeat 2.a and 2.b With Increasing Difficulty} There is an increase of difficulty in each level. That is to say, in each level it gets more difficult to distinguish phishing URLs from valid ones. The learning content of each level can be consulted in \textbf{....(add label ref)}.
	\end{enumerate}
\end{enumerate}

%===========================================
\subsection{Game Rules}
%===========================================

%===========================================
\subsection{Leveling Strategy}
%===========================================
Three approaches...

%===========================================
\subsection{Knowledge Transfer Per Level}
%===========================================
This section summarizes the learning objectives of each level. Note that we generally do not use technical term like URL, domain, subdomain, protocol or the like.

\begin{description}[marginleft=0cm]
	\item[Introduction 1] This part is the awareness part described in \textbf{add reference}. Here, the user learns how easy e-mail spoofing is. Additionally, the user is informed about the simplicity of setting up fake websites and that he should not trust the texts of the links he is clicking on.
	\item[Introduction 2] In this part the user is explained how he can access the URL of a web browser and how exactly he has to look at the whole URL. In particular, the user is told that he has to scroll up the whole website to make the generally hidden address bar re-appear. Then he has to tap the text field of the address bar and scroll to the start of the URL. At the end of the exercise for this the user is told that he always has to analyze the URL like this, because all other displayed URLs or links might be fake too.
	\item[Level 1] The actual game starts with level 1, where the user learns about the structure of a URL. First of all, the user gets an overview of the single components of a URL. To make the comprehension of these components easier to understand we used an analogy which is summarized in Figure~\textbf{Figure referenzieren and include graphic} with an example URL. We told the user that he has to imagine that the website he is visiting is his dialog partner. The user is told that the section between "http(s)://" and the third slash "/", i.e. the hostname, reveals information about his dialog partner. In particular, we explain that he has to read this part from right to left. The top-level and second-level domain is introduced as "Who-Section" (company + location of company), from which the user knows who he is actually talking to. All succeeding parts in this area are to be considered as "departments" of the company of ther user's dialog partner. The protocol part is introduced as "Security Level" of the dialog with the partner and the path part of a URL, i.e. the part after the third slash "/", is introduced as the topic of the conversation with the dialog partner. The main objective of the level 1 exercise is to be able to identify the second- and top-level domain of a URL.
	\item[Level 2] With level two we start introducing the spoofing tricks of a phisher. We considered the subdomain attack \textbf{ref to  URL categorization} as a good starting point to introduce the phisher as the user has just learnt about the importance of the "Who-Section" (top-level and second-level domain) in level 1.
	\item[Level 3] In level 3 the user is first told what an IP address is. To facilitate the comprehensibility, we used the analogy of house addresses. The user is explained that like addressing our houses with street names and numbers, computers in the Internet are addressed by so called IP addresses. The IP address itself is defined as a 4-place sequence of numbers, separated by dots. Finally, the user is warned against URLs with IP addresses in the host part.
	\item[Level 4] In this level we deal with nonsense in the second-level domain \textbf{ref to section}.
	\item[Level 5] In this level we deal with second-level domain names which sound trustworthy, but are in fact unrelated to the company name \textbf{ref to section categories}.
	\item[Level 6] Here misleading and deceiving names in the second-level domain of a URL are covered. This includes typos, scrambled letters or other similar and deceptive names in the second-level domain.
	\item[Level 7] In this level we focus on homographic attacks, where the user is able to visually distinguish a fake second-level domain from the original one~\textbf{ref to categroies}.
	\item[Level 8] In this level the user is introduced to an attack where the brand name of the visited website or even the whole legitimate URL is placed in the path of a fake URL~\textbf{ref to categories}.
	\item[Level 9] Here we introduce the difference between the usage of http:// and https://. In particular, the user is told that the usage of https:// means that his conversation with the website is encrypted and that the dialog partner indicated in the "Who-Section" is authenticated. As an analogy we say that the https:// represents a higher security level. This means, the conversation cannot be eavesdroppbed by a third party and the dialog partner indictated in the "Who-Section" has proved his identity to a trusted third party. With http:// this security level is not established.
	\item[Level 10] This level does not include an exercise. It mainly serves as a section with some important additional input for the user. Specifically, we tell the user two things: First, we explain to him that he might encounter URLs which actually look very phishy. In such a case, we suggest him to directly contact the company and ask for the authenticity of the specific website. Furthermore, we introduce extended validation certificates. We provide the user with a link to further information to this subject.
\end{description}

%===========================================
\subsection{URL Generation}
%===========================================

%===========================================
\subsection{Gamification}
%===========================================
User motivation

\begin{description}
	\item[Show Leaderboard Rate]
	\item[Show Leaderboard Total]
	\item[...]
\end{description}
