\documentclass[article,type=msc,colorback,accentcolor=tud9c]{tudthesis}
\usepackage{ngerman}
\usepackage[american,ngerman]{babel}
\usepackage{tabularx} % better tables
\usepackage{colortbl}
\usepackage{hyperref}	% urls
\usepackage{enumitem}
\usepackage{listings}	% nicer lists
\usepackage{cleveref}
\usepackage{lineno}
\usepackage{color, colortbl}

\newcounter{dummy} % necessary for correct hyperlinks (to index, bib, etc.)


\newcommand{\getmydate}{%
  \ifcase\month%
    \or Januar\or Februar\or M\"arz%
    \or April\or Mai\or Juni\or Juli%
    \or August\or September\or Oktober%
    \or November\or Dezember%
  \fi\ \number\year%
}

\definecolor{rowColorHead}{rgb}{0.7,0.7,0.7}
\definecolor{rowColor1}{rgb}{0.9,0.9,0.9}
\definecolor{rowColor2}{rgb}{255,255,255}

\begin{document}

  \thesistitle{Anti-Phishing Education App}%
    {Design, Implementation and Evaluation}
  \author{Clemens Bergmann und Gamze Canova}
  %\birthplace{Darmstadt}
  \referee{Professor Dr. Melanie Volkamer}{Arne Renkema-Padmos}
  \department{Fachbereich Informatik}
  \group{Security, Usability and Society}
  \dateofexam{\today}{\today}
  \tuprints{12345}{1234}
  \makethesistitle
  \affidavit{C. Bergmann}
	\affidavit{G. Canova}

 \tableofcontents

	%======================================================
	% CONTENT
	%======================================================
	\cleardoublepage
	\pagenumbering{arabic}
	
	
	% !!!!!!!!!!! WOERTER VEREINHEITLICHEN: !!!!!!!!!!!
	%anti-phishing
	%capitalization
	%website / web site
	%...
	%	\linenumbers

	%Abstract:
	\begin{abstract}	
    ...
	\end{abstract}
	
	
	%*******************************************
\section{Introduction}
%*******************************************
\label{s:introduction}
%subject
This chapter introduces the target of this work, which is to design, implement and evaluate an educational app which is supposed to teach unexperienced people to detect phishing attacks. At first we are going to motivate the benefit of our work and how we envision our approach to achieve our goal. Next, we define our specific objectives and point out the major challenges security education poses. Finally, we provide an overview of the following chapters.



%===========================================
\subsection{Motivation}
%===========================================

%-------------------------------------------
\subsubsection{Statistics}
%-------------------------------------------

%-------------------------------------------
\subsubsection{Consequences}
%-------------------------------------------

%-------------------------------------------
\subsubsection{Technical Solutions}
%-------------------------------------------

%-------------------------------------------
\subsubsection{Anti-Phishing Education on the Smartphone}
%-------------------------------------------

%===========================================
\subsection{Goals}
%===========================================
\label{s:goals}
We begin with stating our primary goals of this thesis and describe them in more depth subsequently. The goals of this thesis are to extend, not replace, the technical solutions by
\begin{enumerate}
	\item Increasing the user awareness
	\item Educating the user
	\item ...
\end{enumerate}

Goal description ...


%===========================================
%\section{Challenges}
%===========================================
%DO WE NEED THHIS???

%\label{s:Challenges}
%Educating end users and motivating them to be actually taught something is a complex and challenging task. We divide. The following listing summarizes these challenges: 
%\begin{enumerate}
%	\item Challenge 1
%	\item Challenge 2
%	\item ...
%\end{enumerate}


%===========================================
\subsection{Our Approach}
%===========================================
In the succeeding, we elaborate on how we are going to approach the challenges mentioned before. The reasoning for our approach follows in Section~\ref{related_work:discussion}.

...


%===========================================
\subsection{Outline}
%===========================================

This thesis consists of ... main chapters: .... Their purpose is as follows:

Chapter 1 motivates this work...

Chapter 2 ...

Chapter 3 ...

...

Chapter ... finally summarizes this work and provides an outlook on future work.






	%*******************************************
\section{Background}
%*******************************************
\label{s:background}
Introducing sentences...

%===========================================
\subsection{Definition of Phishing}
%===========================================
\label{s:phishing_def}
The goal of this work is helping users to distinguish phishing websites from legitimate ones. Since phishing is important in the scope of this work, we are going to define the term first. The following definition is intendedly kept abstract.
%if needed look for good source to quote
%Countermeasure Techniques for Deceptive Phishing Attack 
%Huajun Huang, Junshan Tan, Lingxi Liu

\begin{center}
\textit{``Phishing is the practice of obtaining confidential information from users and describes a form of identitfy theft. Targeted confidential information includes, but is not limited to user names, passwords, social security numbers, credit card numbers, account information, and other personal information.''}
\end{center}

There exist several techniques how phishers can steal users' personal data. In the following section we dwell on some of these techniques. 

%-------------------------------------------
\subsection{Phishing Techniques}
%-------------------------------------------
\label{s:phishing_techs}
There are various possibilities how phishers can obtain users' confidential information. In the following we describe phishing techniques that can be distinguished~\cite{emigh2005online}.
%Online Identity Theft: Phishing Technology, Chokepoints and Countermeasures. ITTC Report on Online Identity Theft Technology and Countermeasures
%master_thesis/notes/phishing

\begin{description}[leftmargin=0cm]
	\item[Deceptive Phishing] In deceptive phishing social engineering plays a decisive role. Here, users are lured to disclose their confidential data directly to the phisher without being aware of it. A typical scenario is the unsuspecting user receiving an e-mail from an institution he trusts. In fact this e-mail is malicious and links to a fake website, where the phisher intends to steal the user's data. Once the phisher obtains the user's data, he is able to impersonate the victim's identity and benefit from this.
	\item[Malware-Based Phishing] As the term already reveals, malware-based phishing embraces some kind of malicious software running on the user's computer. There are several ways of infecting the user's computer with such malware. Social engineering techniques can be used to convince the user to open malicious e-mail attachments or download malevolent files from a website. Another possibility is to exploit security vulnerabilities. Various technologies can be utilized to get at the users' data. Keyloggers and screenloggers, for example, track users' data input and send relevant information to a phishing server. Another way is to make use of so-called web trojans, which appear when users intend to log in. While the user thinks he is logging in on a website of his trust, the entered information is actually transmitted to the phisher.
	\item[DNS Based Phishing] This kind of phishing is also referred to as pharming and includes the manipulation of a system's host file or domain name system. These kinds of tampering result in returning a fraudulent IP address for URL requests and thus leading the user to a malicious website, even though the URL of a legitimate website had been entered. As a consequence the unaware user enters his credentials into this fake website and the attacker obtains these and can misuse them.
	\item[Man-in-the-Middle Phishing] In this form of attack the phisher positions himself between the legitimate website and the user. The user's data input is delivered to the phisher, where he stores the information and then forwards it to the legitimate website. Responses are also forwarded back to the user so that the interference of the phisher does not affect the user's interactions. The gained sensitive information can then be sold or misused in any other way. As everything works as usual for the user, it is very difficult for him to detect such an attack. 
	\item[Content Injection Phishing] Content injection phishing refers to the practice of embedding additional harmful content into legitimate websites. This content can, for example, be malvolent code to log users' sensitive information and deliver the input to the phishing server. Well-known types of content-injection phishing include, for example, cross-site scripting. Cross-site scripting vulnerabilities result from a web application's usage of content from external sources, such as search terms, auctions or user reviews of a product. This type of data supply can be misused and instead of delivering the expected kind of data malicious scripts can be injected.
	\item[Search Engine Phishing] Other phishing attempts involve search engines. Here, websites with offers for fake low cost products and/or services are legitimately indexed with search engines. Thus, users reach these websites when using the search engine. These offers, which are often too good to be true, then lure the user to buy those fake products which in turn leads to the disclosure of their sensitive information, such as the credit card number, to the phisher.

\end{description}
Within the scope of this work we focus on deceptive phishing. Besides the different kinds of techniques of phishing there also exist a number of attack channels a phisher can make use of. The following section deals with these attack channels.

%(eventuell liste oder aufzählung) 
%EXAMPLE TABLE WHICH MIGHT BE USEFUL :D
%\begin{table}
%	\centering
%	\begin{tabularx}{.9\textwidth}{m{2.6cm} m{3.8cm} m{4.0cm} m{4.12cm}}
%	\hline	
%	\rowcolor{rowColorHead}
%										& Spalte 1 												& Spalte 2 			& Spalte 3\\
%	\hline
%	\rowcolor{rowColor1}
%	Zeile 1 					& Inhalte, \newline Inhalte			&	Inhalt			 		&	Inhalt \\		
%	\rowcolor{rowColor2}
%	Zeile 2 			& Inhalt, \newline Inhalt			&	Inhalt					&	Inhalt, \newline Inhalt	\\	
%	\hline
%	\end{tabularx}
%	\caption{Description}
%	\label{table:label}
%\end{table}
%-------------------------------------------
\subsection{Phishing Attack Channels}
%-------------------------------------------
Several attack channels that can be used by phishers to reach their victims. This section intends to introduce possible attack channels.
\label{s:attack_channels}
\begin{description}[leftmargin=0cm]
	\item[E-Mail] E-Mail spoofing is a well-known way of a phisher to reach his victims. These e-mails usually imitate renowned institutions, organizations, companies or banks the recipients trust. They usually contain a text which will deceive the recepient into doing what it says. Usually these e-mails link to a malicious website, whose look and feel is almost identical to the original one. There the user is lured to enter his sensitive data which is captured by the phisher. Other alternatives are embedded forms in the e-mail the user has to fill in. Sometimes users are even asked to directly send back their confidential data.
	\item[SMS] An alternative to acquire confidential user data is making use of cell phone text messages. As with e-mails, the text message may contain a link to a fake website, where the user is induced to divulge his sensitive information. The user may also be asked to send back the information directly. Another common way is to be asked to call back a telephone number. This number leads to an automated voice response system which is intended to gain the confidential information from the calling user.
	\item[Instant Messaging] Eher break in and spread links to friends --> glaubwürdig
	\item[Online Social Networks] Eher break in and spread links to friends --> glaubwürdig
	\item[VoIP] 
\end{description}


We focus on fake websites. attack channel egal. ob über SMS instant, email.. covered alles. Usually, the links to fake websites are distributed via e-mails, SMS, instant messengers or online social networks, Thus, our approach automatically covers the attack channels e-mail, sms, instant messaging and online social networks.

%-------------------------------------------
\subsection{Variations of Phishing}
%-------------------------------------------
\label{s:phishing_variations}
Do we need this subsection?
\begin{description}[leftmargin=3cm]
	\item[Mass Phishing]
	\item[Spear Phishing]  
	\item[Persistent Spear Phishing]
	\item[Clone Phishing]
	\item[Whaling]
\end{description}

We cover in particular mass phishing. However, the URL checking can be applied in case of any variant, as long as the attack is executed via a fake website.
%===========================================
\subsection{Scope}
%===========================================
\label{s:scope}

%eine grafik wäre nice.. die zusammenfasst, was für eine Art von Phishing wir hier betrachten
	\selectlanguage{american}

%*******************************************
\section{Related Work}
%*******************************************
\label{s:related_work}

In the following, we present a survey of approaches to anti-phishing education....
We divided the related work we have found in literature into two dimensions: the \textit{communicated content} 
%WHAT WAS THE USER TOLD$ and the 
\textit{way of communication} 
%HOW WAS THE USER TOLD ABOUT THE WHAT%. %EVENTUELL ENUM OR SO

%============================================
\subsection{Communicated Content}
%============================================

\begin{description}
	\item[General Knowledge Transfer]
	\item[E-Mail Based Knowledge]
	\item[URL Based Knowledge]
\end{description}

%============================================
\subsection{Way of Communication}
%============================================
\begin{description}
	\item[Game Based Learning]
	\item[Quiz Based Learning]
	\item[Comparison Based Learning]
	\item[Emdedded Learning]
\end{description}

%============================================
\subsection{State of the Art}
%============================================
Examples here ... (e.g. Anti-Phishing Phil and Phyllis)
%============================================
\subsection{Pro and Contra Discussion}
%============================================
\label{related_work:discussion}
	%*******************************************
\section{Focus}
%*******************************************
\label{s:assumptions}
Introductory sentences...
Based on the discussion of the previous section we decided to...... (not only based on previous section)

%===========================================
\subsection{Coverage}
%===========================================
\begin{description}[marginleft=0]
	\item[Phishing Technique - Deceptive Phishing] move above text here
	\item[Attack Channel - E-Mail] or all channels with links to websites? move above text here
	\item[Variation of Phishing - Mass Phishing] move above text here
	\item[Game and Quiz Based Learning] wenn oben geschrieben, hier auch schreiben.			\item[URL Based Knowledge] The advantages of telling the user what to pay attention to within e-mails are the following: if the user recognizes the phishing e-mail before clicking on a link he does not even get onto a fake website where he could be lured to divulge his credentials. This also would mean, that the user would not be forwarded to a page where a malicious download might be started. On the other hand, these fake e-mails become more and more sophisticated and thus it becomes harder to distinguish them from legitimate ones~\cite{http://office.microsoft.com/en-001/outlook-help/identify-fraudulent-e-mail-and-phishing-schemes-HA001140002.aspx, http://www.spamfighter.com/News-18495-Becoming-More-Difficult-to-Detect-Phishing-Email-Attack-says-Security-Experts.htm}. Additionally, e-mail is not the only attack channel where links to phishing websites can be distributed.  Those links are also spread via instant messaging systems, online social networks, or SMS, where the messages would differ from those in e-mails. Moreover, phishing websites can also be reached by surfing~\cite{kaspersky report}, where the e-mail based knowledge approach would completely fail. For these reasons we decided to focus on communicating URL based knowledge to the user. This way, the disadvantages of e-mail based knowledge are mitigated. Furthermore, we believe that URL based knowledge gives the most reliable hint regarding its "belonging", i.e. whether a URL in fact belongs to a legitimate website or not.
	\item["After Click" URL Analysis] We have decided to consider the "after click" scenario for the following reasons: Firstly, we cannot hinder users from clicking on links and make them type in the whole URL into the address bar. This is too effortful, especially on smartphones, and thus will not be followed by them. Secondly, many links contain redirects. Such redirects are not recognizable before the click. A further problem the "before click" scenario raises is that the stock e-mail client of Android does not provide the functionality of viewing the destination URL before clicking on it. The only way to have a look at the URL before clicking on it is to make a long press onto the link, copy it into the clipboard, paste it somewhere else and then analyze it. Then, after the analysis the URL has to be sent to the browser. However, as this is also involves too much effort, no user will follow such a suggestion. Finally, even if there are many other e-mail clients which offer viewing the destination URL via long press only, we believe that this should not be communicated to the user for two reasons. Firstly, we do not know how many Android users actually make use of the stock e-mail client, which does not offer this functionality. Secondly, and most importantly, this functionality has the potential to mislead the user. A drawback of the URL destination preview is that the end of it is cut in case the URL is too long. Well-crafted URLs might thus look legitimate even though they are not because the most important part of the URL was cut out. For example, the subdomains of the URL can be long and well-crafted so that a legitimate looking subdomain is exactly at the end of the preview. This will cause the user think, that the subdomain at the end of the preview is the domain of the URL. Ultimately, the user will trust this URL even he should not. For the reasons explained above we have decided to consider the "after click" scenario. This approach suffers the disadvantage that users might click on a link which has a download of malware behind it.  For now, we consider this as future work, as there is no possibility to detect whether there is a download behind a URL before requesting the site.
%However, this drawback is mitigated by the fact that such downloaded malicious software is %not harmful as long as it is not opened and installed. Thus, the user should be told that %downloads he did not intend should immediately be deleted and should not be opened %under any circumstances.
% SCHÄDLICHE SW KANN SICH AUCH AUTOMATISCH AUSFÜRHEN / ÖFFNEN VON PDF IN BROWSER IST DANN AUCH SCHON GEFÄHRLICH
%Add to future work.
%Falls unverständlich vielleicht ein screenshot mit einem Beispiel ;) das mit well-crafted url
	\item[Considered Browser] für screenshots benutzen wir Android standard browser, aber kann auf jeden browser übertragen werden. wir haben überlegt, sahcen wie lock icon Blabla einzubauen, aber da immer sehr unterschiedlich haben wir uns dagegen entschieden. um unsere Methode möglichst allg. zu halten-- siehe bitte section blablabla
	
TODO: hinzufügen zu warum smartphone: •	Das Erlernte zu Smartphones kann einfach auf PCs übertragen werden. Umgekehrt schwieriger, weil erste Herausforderung: wie kommt man an die wichtigen Teile der URL.

//IDEE: UNSERE ZIELE/COVERAGE VON SCOPE UND BACKGROUND RAUSNEHMEN UND SCHIEBEN IN FOCUS!! OBEN SCHREIBEN: WIRD IN SECTION BLABLA BESCHRIEBEN WAS WIR COVERN UND WARUM. UND HIER NOCH DAZU PACKEN.


%===========================================
\subsection{System Requirements}
%===========================================
In the following we are listing the system requirements which need to be met for the final app.

\begin{description}[leftmargin=0cm]
	\item[Android] We have decided to develop an app for the Android operation system as we believe we have greater freedom here compared to an iOS app. The publication of an iOS app, for example, is connected with more requirements, which is not the case for Android apps~\cite{publishios, publishandroid}.
	\item[Version] Our original intention was to develop an Android app for version 4.0 and upward. However, during the app development we have encountered that about 24\% of all Android users still have Android 2.3.3 to 2.3.7~\cite{}. For this reason we have decided to modify the code so that these users can also install and use our app. 
	\item[Android Standard Browser] Android standard browser is kein system requirement - raus. muss irgendwoanders erwähnt werden. (was betrachten wir beim erklären oder so)...
\end{description}

%===========================================
\subsection{Assumptions}
%===========================================
\begin{description}
	\item[Secure DNS] ...
	\item[Secure Smartphone] ...
	\item[No Before-Click URL Analysis] ...
	\item[Download URLs Possible] ...
\end{description}

%===========================================
\subsection{Limitations of Our Approach}
%===========================================
\begin{description}
	\item[Cross-Site Scripting] ...
	\item[URL Hiding Techniques] ...
\end{description}


	\input{target_group}
	%*******************************************
\section{Pre-Survey}
%*******************************************
\label{s:prestudy}
Before elaborating on the concrete app design we ran a small pre-survey. To the best of our knowledge there do not exist other surveys which resemble ours and additionally were conducted in Germany. This chapter deals with the main objectives of the pre-survey. Furthermore, it provides some details and finally presents the results and evaluates the questionnaire.

%============================================
\subsection{Main Objectives}
%============================================
Our main objectives of this pre-survey were twofold:

\begin{enumerate}
	\item \textbf{Awareness and Knowledge} One goal of the pre-survey was to comprehend what exactly Internet users understand under phishing. With a Likert scale we furthermore tried to figure out how they evaluate their on knowledge on the topic of Internet security.
	\item \textbf{Preferences of Users} Another purpose of the survey was to get an idea of the users' preferences with regard to an educational app. For example, they were asked whether they found a quiz based game appropriate for learning purposes.
\end{enumerate}
%============================================
\subsection{Survey Details}
%============================================
This section provides some details about our questionnaire, how we distributed it and how we filtered the surveys in order to consider our target group for the results and evaluation.
\textbf{SURVEY IN APPENDIX???}

\subsubsection{Questionnaire}
In the following we present the structure of our questionnaire and the function of each section.
\begin{enumerate}
	\item \textbf{General Information} In this section the participant is asked to provide information regarding his gender, age, his professional qualification as well as his field of study or work. The main purpose of this section is to exclude participants which do not fit into our target group.
	\item \textbf{Internet Usage} Here, the participant is asked how often he uses the Internet, whether he owns a smartphone and which applications he uses on his desktop computer and which ones he uses on his smartphone. This section is intended to give us an overview of the users' Internet usage and helps us to exclude participants who do not fit into our target group.
	\item \textbf{Self-Assessment} In this part of the survey, the participant has to indicate how much he agrees to the presented statements with the aid of a Likert scale. The statements mainly refer to their self-assessment regaring their knowledge about Internet security. For example, they have to assess, whether they think they have enough knowledge, to avoid the dangers of the Internet or whether they think it is easy for them to distinguish legitimate e-mails from fake ones.
	\item \textbf{Phishing} Here, the participant gets concrete questions to the topic of phishing. In particular, he is asked which services and which user information are endangered by phishing attacks. This section purposes to find out what the participants know about and think of phishing.
	\item \textbf{Anti-Phishing App} This section asks the user for his preferences regarding an anti-phishing education app. With the aid of a Likert scale he is requested to assess, for example, whether the would like having a game with a fish, or whether he finds a text-based approach meaningful as well as whether he would have fun with a question-answer quiz game.
	\item \textbf{Further Survey Progress} In this part of the pre-survey the user can provide us his e-mail address in case he wants to get information about the further progress of the survey or would like to test the app.
\end{enumerate}

\subsubsection{Distribution}
In total 253 persons participated in our pre-survey. We set up an online survey as well as asked students to fill out our printed survey. In the following we briefly explain our distribution process.

\begin{description}
	\item[Printed Survey] To reach participants for our printed pre-survey we contacted multiple professors and asked them whether we could have 10 minutes of their lecture time to have their students fill out our printed survey. Moreover, we asked our friends and parents whether they can ask their friends, colleague or customers to fill out the questionnaire.
	\item[Online Survey] The online survey was mainly distributed digitally. We contacted our friends and asked them to participate in the survey. We also demanded to forward the link to their friends so we could reach a wider range of people.
\end{description}

\subsubsection{Filtering for Evaluation}

The following Table~\ref{table:prestudy_filer} summarizes what kind of answers we used in order to exclude participants from the survey who do not fit into our target group.

\begin{center}
    \begin{tabular}{ | p{5cm} | p{10cm} |}
    \hline\textbf{Question} & \textbf{Filtering}  \\  \hline
		\hline\  Age & We consider all adults ranging from 18 - 65 years. \\
    \hline\  Gender & We do not exclude any gender. \\ 
    \hline\  Professional qualification & The participant does not have to exhibit a specific professional qualification to be considered for the results and evaluation. \\ 
		\hline\  Field of study/work & Students, employees or employers in the field of computer science or electrical engineering are filtered out as they do not belong to our target group. \\ 
	  \hline\ Frequency of Internet usage & Participants who have indicated ``rarely'' as the answer to this question do not belong to our target group and thus are filtered out. \\ 
	  \hline\ Used Internet applications  &  The listed applications include, for example, browser, e-mail, shopping as well as banking. Any service of the Internet is potentially endangered by phishing. For this reason we do not use this question to filter out participants.\\ 
    \hline\ Owning a smartphone  & With the app we particularly target smartphoner owners. For this reason participants who do not own any kind of smartphone are filtered out. \\
		\hline\ Used smartphone applications in the Internet  & The listed applications include, for example, browser, e-mail, shopping as well as banking. Any service of the Internet, especially on a smartphone, is potentially endangered by phishing. For this reason we do not use this question to filter out participants. \\
    \hline\ Number of received commercial e-mails per week  & We do not filter out any participant with this question. \\
    \hline\ Number of received e-mails asking for personal data  & We do not filter out any participant with this question. \\
    \hline\ User reads up on topics related to dangers in the Internet  &  Participants who have chosen ``no'' as answer are filtered out. We specifically target users who are interested in getting safer in the Internet. As the participants who have indicated ``no'' do not seem to have any interest in doing so, they will most likely do not show interest in our app. For this reason we regard them as not belonging to our target group and exclude them from the analysis and evaluation.\\
    \hline\   &  \\
		\hline\   &  \\
    \hline\   &  \\
    \hline\   &  \\
    \hline\   &  \\
    \hline\   &  \\


    \hline
    \end{tabular}
		
\end{center}
%%%TO ADD: CAPTION AND LABEL (UND OBEN REFERENZIEREN)

%============================================
\subsection{Results and Evaluation}
%============================================
	%*******************************************
\section{Teaching and Learning Content}
%*******************************************
In this section we will describe and elaborate on different teaching and learning contents which can potentially be communicated to the user. At the same time we will reason our decision whether to communicate the specific content or not.
%Documents master_thesis/notes/android_browser bla -> BEGRÜNDUNG WARUM manches nicht sinnvoll ist (diese Sachen vielleicht eher in Appendix vor allem versionsunterschiede)
%Documents master_thesis/konzepte/android browser elemente UND browser comparison
%CHECK IF I FORGOT SOMETHING!!!!
%===========================================
\subsection{Phishing URLs}
%===========================================
Focus on distinguishing phishing URLs from legitimate ones.

%...........................................
\subsubsection{Phishing URL Categorization}
%...........................................

Potential phishing URL categories/phishing attcks on URLs
\begin{description}
	\item[Subdomain] covered
	\item[IP Address] covered
	\item[Nonsense Domain] covered
	\item[Trustworthy, But Unrelated Domain] covered
	\item[Similar and Deceptive Domains] covered Typo, Typosquatting (Buchstabendreher), Misspelling
	\item[Homographic Attack] covered (the type of homographic visible by user...)
	\item[Tiny URLs] Not covered
	\item[Cloaked URLs] Not covered - because redirect (use of @)
	\item[Encoding Tricks] Not covered - because redirect
\end{description}

%...........................................
\subsubsection{Problems and Challenges With The Categorization}
%...........................................

%...........................................
\subsection{Android Elements}
%...........................................

\begin{description}
	\item[Invisible Address Bar] Find URL Bar, Browser
	\item[Use of Https Within Websites] Browser
	\item[Analyze Complete URL Via Address Bar] Browser
	\item[Show URL Before Click] In E-Mail (not always possible), while surfing (long touch)
	\item[Copy and Paste URL] too much effort, additionally: redirects still possible
\end{description}

%...........................................
\subsection{Android Browser Security Indicators}
%...........................................

\begin{description}
		\item[Https Padlock] Browser
		\item[Displayed Webaddress on Https Sites] Browser
		\item[Certificate Verification]
		\item[Touch Padlock] to see whole URL.. problems: see document...
\end{description}

%...........................................
\subsection{E-Mail Spoofing}
%...........................................

\begin{description}
	\item{From Field} not trustworthy
	\item{E-Mail Content} in hand of attacker
	\item{Links in E-Mails} do not necessarily go where it claims to go (not only in e-mail links).
\end{description}
\subsubsection{General Recommended Behavior}
\begin{description}
	\item[Do Not Click]
	\item[Do Not Download Attachment]
	\item[Look at URL]
	\item[Data Economy]
	\item[Date Entry Via Https]
\end{description}

%...........................................
\subsection{Conclusion / Summary}
%...........................................

Summarize what to communicate to user here...


	
%*******************************************
\section{Approach for Our Anti-Phishing Education App}
%*******************************************
\label{s:approach}
This chapter presents our final approach for the Anti-Phishing Education App....
%===========================================
\subsection{App Design}
%===========================================
\begin{enumerate}
	\item Awareness Part
	\begin{enumerate}
		\item From is not from...
		\item Linktext unequal actual target URL
	\end{enumerate}
	\item Education Part
	\begin{enumerate}
		\item Information Material
		\item Exercise to Information Material
		\item Repeat 2.1 and 2.2 with increasing difficulty
	\end{enumerate}
\end{enumerate}

%===========================================
\subsection{Game Rules}
%===========================================

%===========================================
\subsection{Leveling Strategy}
%===========================================
Three approaches...

%===========================================
\subsection{Knowledge Transfer Per Level}
%===========================================
What is taught in each level ... 

%===========================================
\subsection{URL Generation}
%===========================================

%===========================================
\subsection{Gamification}
%===========================================
User motivation

\begin{description}
	\item[Show Leaderboard Rate]
	\item[Show Leaderboard Total]
	\item[...]
\end{description}

	%*******************************************
\section{Evaluation}
%*******************************************
\label{s:evaluation}
The goal of this chapter is to evaluate our Anti-Phishing Education App which we described in the previous chapter. 

%===========================================
\subsection{Hypotheses}
%===========================================

%===========================================
\subsection{Measurement}
%===========================================

%===========================================
\subsection{Participant Recruitment}
%===========================================

%===========================================
\subsection{Study Design}
%===========================================

%===========================================
\subsection{Results and Analysis}
%===========================================
%===========================================
\subsection{Discussion}
%===========================================
%===========================================
\subsection{Conclusion}
%===========================================

	\input{conclusion}
	
	\bibliographystyle{abbrv} % <--- layout of the bib
	\bibliography{bibliography} % file name of your bib

\end{document}
