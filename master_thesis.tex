\documentclass[article,type=msc,colorback,accentcolor=tud9c]{tudthesis}
\usepackage{ngerman}
\usepackage[american,ngerman]{babel}
\usepackage{tabularx} % better tables
\usepackage{colortbl}
\usepackage{hyperref}	% urls
\usepackage{enumitem}
\usepackage{listings}	% nicer lists
\usepackage{cleveref}
\usepackage{lineno}
\usepackage{color, colortbl}

\newcounter{dummy} % necessary for correct hyperlinks (to index, bib, etc.)


\newcommand{\getmydate}{%
  \ifcase\month%
    \or Januar\or Februar\or M\"arz%
    \or April\or Mai\or Juni\or Juli%
    \or August\or September\or Oktober%
    \or November\or Dezember%
  \fi\ \number\year%
}

\definecolor{rowColorHead}{rgb}{0.7,0.7,0.7}
\definecolor{rowColor1}{rgb}{0.9,0.9,0.9}
\definecolor{rowColor2}{rgb}{255,255,255}

\begin{document}

  \thesistitle{Anti-Phishing Education App}%
    {Design, Implementation and Evaluation}
  \author{Clemens Bergmann und Gamze Canova}
  %\birthplace{Darmstadt}
  \referee{Professor Dr. Melanie Volkamer}{Arne Renkema-Padmos}
  \department{Fachbereich Informatik}
  \group{Security, Usability and Society}
  \dateofexam{\today}{\today}
  \tuprints{12345}{1234}
  \makethesistitle
  \affidavit{C. Bergmann}
	\affidavit{G. Canova}

 \tableofcontents

	%======================================================
	% CONTENT
	%======================================================
	\cleardoublepage
	\pagenumbering{arabic}
	
	
	% !!!!!!!!!!! WOERTER VEREINHEITLICHEN: !!!!!!!!!!!
	%anti-phishing
	%capitalization
	%website / web site
	%...
	%	\linenumbers

	%Abstract:
	\begin{abstract}	
    ...
	\end{abstract}
	
	
	%*******************************************
%*******************************************
\section{Introduction}
%*******************************************
\label{s:introduction}
%subject
This chapter introduces the target of this work, which is to design, implement and evaluate an educational app. The app is supposed to help unexperienced users to detect phishing attacks. At first we are going to motivate the benefit of our work and how we envision our approach to achieve our goal. Next, we define our specific objectives and finally, we provide an overview of the following chapters.

%===========================================
\subsection{Motivation}
%===========================================
Phishing is the practice of luring confidential information from users, cf.~Section~\ref{s:phishing_def}. Usually, this happens through fake websites which imitate the original ones. On these so called phishing websites the users are asked to enter their personal data. In this section we elaborate on the importance of countering such phishing attacks with the aid of user education. 

%-------------------------------------------
\subsubsection{Statistics of Phishing}
%-------------------------------------------
Nowadays, a world without Internet is unimaginable for many people. However, it is undeniable that the Internet brings at least as much threats with it as it brings benefits. One major issue of today's digitalized world is phishing, which is also reflected by many statistics of various reports. According to the Anti-Phishing Working Group~(APWG) approximately 40,000 unique phishing websites are detected each month~\cite{antiphishingtrendreport2013}. Statistics published by Kaspersky Lab, a well-respected provider for IT security solutions, state that from year 2011-2012 to 2012-2013 the number of attacked users increased by about 87\%. While in 2011-2012 the number of users, who were subject to phishing attacks, was 19.9 million, in 2012-2013 the numbers climbed up to 37.3 million. Every day about 100,000 Internet users are victims of phishing attacks, which is twice as many compared to the previous period of 2011-2012. An immense increase can also be observed in the number of unique sources (i.e. IPs) of attacks, which has tripled from 2012 to 2013~\cite{kasperskyreport2013}. The amount of target institutions also rose. While in 2011 the APWG counted about 500 target institutions, in the fist quarter of 2013 720 target institutions were identified~\cite{antiphishingglobalreport2013}. Finally, the estimated worldwide costs caused by phishing are about \$1.5 billion for the year 2012~\cite{rsa2013}.


%-------------------------------------------
\subsubsection{Consequences of Phishing}
%-------------------------------------------
Falling for a phishing attack has several consequences for the victim as well as for the target company or organization. Phishing is the practice of tricking users to disclose their personal data. That is to say, a possible consequence of falling for a phishing attack is identity theft. With the data unknowingly provided by the victims, the attacker can impersonate his victims. Financial loss is another problem resulting from phishing attacks. Not only users who are subject to phishing attacks can suffer financial loss, but also the institutions, organizations and companies targeted by the phisher can. Financial loss can be the result of users' banking accounts being plundered or increased support costs for the targeted institutions due to their customers who fell for an attack. Moreover, the targeted institutions may sustain a damaged reputation due to phishing attacks. Customers who actually became a victim of such a phishing attack will be displeased about the money or account loss and the resulting efforts they have to make in consequence of such an attack. Furthermore, they will tell other people about this displeasant experience. Finally, these victims will lose their trust in the institution targeted by the phiser. Moreover, they might lose confidence in eCommerce operations and the Internet in general.

%-------------------------------------------
\subsubsection{Technical Solutions to Counter Phishing}
%-------------------------------------------
%Quellen vom diesem dokument: 17065505

Several technical solutions to counter phishing have already been suggested in literature~\cite{purkait2012phishing}. In the following some of these solutions are briefly summarized.

\begin{description}[leftmargin=0cm]
	\item[Spam filters] Not rarely, the phisher sends out a mass of emails to users which link to fake websites, where the users are lured to disclose their personal data. Consequently, one possible countermeasure to stop phishing is to filter these e-mails before they even reach the receiver. Various approaches for such spam filters do already exist~\cite{bergholz2010new,chandrasekaran2006phishing,fette2007learning}, but also have their drawbacks. First, it is not possible to make sure that all users make use of such spam filters. Second, even if spam filters are used by the majority, one can not make sure that they are updated regurlarly. In addition, phishers are able to adapt to improved technology. Consequently, such filters can not assure 100\% accuracy. On the one hand it is possible that phishing e-mails can make it through these filters. As a result, the user might still fall victim to such an attack. On the other hand there are legitimate e-mails which might not reach the user. This might result in a user's loss of confidence, which in turn can result in the user not making use of the spam filter anymore~\cite{olivo2011obtaining}.
	\item[Blacklists] Fake websites are a common way for phishers to get at users' data. Thus, another alternative to protect endagered users from phishing attacks is to restrict the access to such phishing websites with the aid of so called blacklists. Here, the browsers hold a list of revealed phishing websites. If a requested URL is contained in such a blacklist the access to this website can be restricted or the user can be alerted about the phishing website. Several blacklisting approaches have been suggested in literature~\cite{ma2009beyond, zhang2008highly}. The major downside of blacklists is that most of them work reactively. That is to say, there is a certain time frame where phishing websites are active without being blacklisted. In this time frame users can access these website without being warned or restricted and thus are vulnerable to fall for the attack. To resolve this problem multiple dynamic and predictive approaches have been proposed to restrict and/or warn the user from accessing phishing websites~\cite{prakash2010phishnet, obied2009fraudulent}. Nevertheless, there is no flawless blacklisting approach, as there are always malicious websites which can bypass such protective systems. Moreover, these systems require a high effort, since a regular and realtime update is inevitable in order to make the system effective~\cite{purkait2012phishing}. Finally, there is the weakest link in the security chain, the users who are very often unsure about what to do when getting such security warnings~\cite{bakhshi2009social}. In case of disregard of these warnings such systems are useless.
	\item[Visual distinction] A further technical approach against phishing is the visual distinction of phishing websites from legitimate ones. For this purpose it is necessary to identify malicously duplicated websites mainly based on visual similarities~\cite{liu2006antiphishing}. Various solutions can be found in literature to approach this~\cite{chen2009fighting,chen2010detecting,zhang2011textual}. However, there is no foolproof solution. In particular, if approaches rely on visual similarities many of them will fail if the phishing website is not a duplicate of the original site. Moreover, phishers will always be able to adapt to sophisticated solutions in order to bypass these security levels. Finally, as always the human factor plays a huge role here: if users keep misunderstanding or ignoring such visual security indicators such techniques will remain of no use.
	\item[Takedown] Another possibility to protect users from accessing phishing websites it to take them down~\cite{moore2007examining}. Here, hosting service providers are urged to take down such malicious websites by for example banks, other organizations or specialist takedown companies. In this way, a visitor will not see anything of the phishing website on this particular site and thus will not provide his data to the phisher. The removal of phishing website is an effective solution, since it implicitly solves the problem with the human factor, where users ignore security warnings. However, this approach can not defeat phishing completely, since it is not fast enough~\cite{moore2007examining}. During the uptime of the fraudulent website falling for these attacks remains possible.
\end{description}

As a conclusion, there are two major issues of technical solutions. First, technical solutions do not assure 100\% accuracy. There is always the potential of false positives and false negatives. Furthermore, attackers will find a way around sophisticated solutions and be able to bypass these somehow. The second major problem with these approaches is the user behavior. As indicated above users tend to overlook or intendedly ignore security warnings. If the user behavior does not change such approaches will remain useless. The main reason why users overlook or ignore such security indicators is that security is not their primary goal. Consequently, they give their attention to other things, for example, shopping, online banking and so on. Another factor for overlooking and ignoring these warnings is the lack of user awareness. Some users are just not aware of how easy it is for even unexperienced attackers to duplicate a website and send out fake e-mails. Even if users are aware that there is a certain degree of threat in the Internet, people tend to think the probability that they will face such an attack is very low and that it will not happen to them, until it happens to them. Thus, an important step towards changing the user behavior is increasing the user awareness.


%-------------------------------------------
\subsubsection{Anti-Phishing Education on the Smartphone}
%-------------------------------------------
\label{s:antiphishing_on_smartphone}
There are several reasons why we chose to educate users on the smartphone. The main characteristic of a smartphone is that it is enormously smaller than the well-known desktop computers. As a consequence there is much less space in the screen. Many browsers, for example, generally hide their address bars due to the lack of space. With the address bar, the URL and other potential security indicators are hidden. There is also the fact that users often use their smartphones while on the move, for example, when walking, during a train or a bus ride. These circumstances include distractions from the environment which are unavoidable. These distractions obviously will influence the user's attentiveness. As a consequence smartphone users are even more vulnerable to phishing attacks than the traditional desktop user. This is also indicated by a report of 2011, which says that mobile users are three times more likely to access phishing websites than desktop users~\cite{trusteer2011}. Evidently, there is a need for the protection of smartphone users. Additionally, educating the user on the smartphone provides two major benefits. First, the user can use the app on the move. Thus, the app is accessible outside of the user's desktop environment, where he potentially has better things to do than learning how not to fall for phishing attacks. The app can be used while train or bus rides, while waiting for a friend or while waiting for any other appointment. The app can be used any time as a sideline, so that probably more users would be willing to use it. Also, to the best of our knowledge there does not exist a smartphone application to educate users about phishing yet. Finally, it is easy to transfer the knowledge of smartphones to desktop computers as the screen is bigger and the URL is easy to find. Transferring knowledge from desktop computers to smartphones, on the other hand, is not that simple.

%===========================================
\subsection{Goals}
%===========================================
\label{s:goals}
We begin with stating our primary goals of this thesis and describe them in more depth subsequently. The major goals of this thesis are to extend, not replace, technical solutions to counter phishing by
\begin{enumerate}
	\item Increasing the user awareness
	\item Educating users about phishing 
\end{enumerate}

As already indicated in the previous section the lack of user awareness seems to be a major issue concerning the secure user behaviour. For this reason we want to raise the user awareness by showing our app users that faking e-mail senders and content is very easy. Additionally, we want to make them aware that links do not necessarily lead to the target the link displays to the user. This should happen at the beginning of the app so that the user realizes that the threat of the Internet is prevalent and that he needs to learn to protect himself. Furthermore, the user should practically experience these aspects and not only told, since being told will not suffice to motivate the user to go on with the app.
Increasing the user awareness will not be enough to help the user not to fall for phishing attacks. Besides technical solutions valuable information has to be made available to the user. In particular, we want to qualify our app users to detect phishing URLs so they can distinguish phishing websites from legitimate ones. 

%===========================================
%\section{Challenges}
%===========================================
%DO WE NEED THHIS???

%\label{s:Challenges}
%Educating end users and motivating them to be actually taught something is a complex and challenging task. We divide. The following listing summarizes these challenges: 
%\begin{enumerate}
%	\item Challenge 1
%	\item Challenge 2
%	\item ...
%\end{enumerate}


%===========================================
%\subsection{Our Approach}
%===========================================
%In the succeeding, we elaborate on how we are going to approach the challenges mentioned before. The reasoning for our approach follows in Section~\ref{related_work:discussion}.

%...


%===========================================
\subsection{Outline}
%===========================================

This thesis consists of ... main chapters: .... Their purpose is as follows:

Chapter 1 motivates this work...

Chapter 2 ...

Chapter 3 ...

...

Chapter ... finally summarizes this work and provides an outlook on future work.






	%*******************************************
\section{Background}
%*******************************************
\label{s:background}
Introducing sentences...

%===========================================
\subsection{Definition of Phishing}
%===========================================
\label{s:phishing_def}
Our goal is to educate users to detect phishing websites. Since phishing is important in our work, we are going to define our understanding of the term. 

\begin{center}
\textit{``Definition of Phishing''}
\end{center}

The next section dwells on different phishing types. 


%-------------------------------------------
\subsection{Phishing Techniques}
%-------------------------------------------
\label{s:phishing_techs}
In this section we are going to describe the different phishing techniques that are distinguished in literature. Furthermore we state and reason which technique(s) of phishing we focus on in our work.. Phishing techniques include, but are not limited to:
%master_thesis/notes/phishing
\begin{description}[leftmargin=3cm]
	\item[Deceptive Phishing]
	\item[Malware Based Phishing] (including keyloggers and screenloggers)			
	\item[Host File Poisoning]
	\item[DNS Based Phishing] (Pharming)
	\item[Man-in-the-Middle Phishing]  	
 
\end{description}
For our research, we focus on deceptive phishing...
%(eventuell liste oder aufzählung)

 
%EXAMPLE TABLE WHICH MIGHT BE USEFUL :D
%\begin{table}
%	\centering
%	\begin{tabularx}{.9\textwidth}{m{2.6cm} m{3.8cm} m{4.0cm} m{4.12cm}}
%	\hline	
%	\rowcolor{rowColorHead}
%										& Spalte 1 												& Spalte 2 			& Spalte 3\\
%	\hline
%	\rowcolor{rowColor1}
%	Zeile 1 					& Inhalte, \newline Inhalte			&	Inhalt			 		&	Inhalt \\		
%	\rowcolor{rowColor2}
%	Zeile 2 			& Inhalt, \newline Inhalt			&	Inhalt					&	Inhalt, \newline Inhalt	\\	
%	\hline
%	\end{tabularx}
%	\caption{Description}
%	\label{table:label}
%\end{table}



%-------------------------------------------
\subsection{Phishing Attack Channels}
%-------------------------------------------
\label{s:attack_channels}
\begin{description}[leftmargin=3cm]
	\item[E-Mail]
	\item[SMS]  			
	\item[Instant Messaging]
	\item[Online Social Networks]
	\item[Fake Website]
	\item[VoIP]
	\item[Malicious Downloads]
\end{description}

We focus on fake websites. Usually, the links to fake websites are distributed via e-mails, SMS, instant messengers or online social networks, Thus, our approach automatically covers the attack channels e-mail, sms, instant messaging and online social networks.

%-------------------------------------------
\subsection{Variations of Phishing}
%-------------------------------------------
\label{s:phishing_variations}
Do we need this subsection?
\begin{description}[leftmargin=3cm]
	\item[Mass Phishing]
	\item[Spear Phishing]  
	\item[Persistent Spear Phishing]
	\item[Clone Phishing]
	\item[Whaling]
\end{description}

We cover in particular mass phishing. However, the URL checking can be applied in case of any variant, as long as the attack is executed via a fake website.
%===========================================
\subsection{Scope}
%===========================================
\label{s:scope}

%eine grafik wäre nice.. die zusammenfasst, was für eine Art von Phishing wir hier betrachten
	\selectlanguage{american}

%*******************************************
\section{Related Work}
%*******************************************
\label{s:related_work}

This chapter deals with previous work on anti-phishing education. We divided the related work we have found in literature into two categories: the \textit{content}, i.e. what the user is taught, and the 
%WHAT WAS THE USER TOLD$ and the 
\textit{medium}, i.e. how the user is taught. In the following, we will provide an overview of this content and medium classification. Subsequently, we will provide examples of previous work.
%HOW WAS THE USER TOLD ABOUT THE WHAT%. %EVENTUELL ENUM OR SO

%============================================
\subsection{Content Classification}
%============================================
The objective of this section is to introduce the different classes of learning content which we could identify in previous work.
\textbf{ICH BIN MIR UNSICHER OB MELANIE DIESE KLASSIFIZIERUNG GEFIEL. VIELLEICHT SOLLTEN WIR TATSCÄHLCIH EINFACH NUR RUNTERZÄHLEN WAS ES SO GIBT UND AUF VOR UND NACHTEILE EINGEHEN}
\begin{description}[leftmargin=0cm]
	\item[General Knowledge Transfer] Renowned and targeted websites, such as PayPal, eBay or Microsoft provide general and superficial  information about phishing~\cite{generalknowledgemicrosoft, generalknowledgepaypal, generalknowledgeebay}. Usually, they deal with questions like what is phishing, how does phishing happen, what the symptoms of phishing are and how to report phishing attempts. Providing the user only with text to the topic of phishing makes it possible to communicate any kind of content, so that the learning objectives can get as complex as one wishes. However, it is likely that users do not like reading too much, especially when it gets complex and difficult to comprehend.
	\item[E-Mail Based Knowledge] In this class of content, the user is told about the ``anatomy'' of phishing e-mails~\cite{antiphishingphyllis, sonicwall}. Particularly, they are informed about what kind of hints in an e-mail give indications for a phishing attempt. Indications for a phishing e-mail can be impersonal salutation, requesting personal and confidential information as well as exerting pressure and threatening the user with, for example, account closure. The benefit of detecting phishing attempts before even clicking on a link in an e-mail is that the user would not confirm the existence and active usage of his e-mail address to the phisher. More importantly, the user would not unknownlingly download malicious software. The problem with the e-mail based approach is that detecting phishing e-mails by looking at their content becomes more and more difficult~\cite{microsoftphishing,spamfighter}. Even if today still many phishing e-mails exhibit the obvious characteristic of having no personal salutation or being urgent and threatening, it is likely that these obvious hints will not remain in future.
	\item[URL Based Knowledge] Sending spoofed e-mails with links to fake websites is a common trick of phishers. On the target website then, the user is lured to disclosing his credentials. Thus, detecting such fake websites is another possibility to protect oneself against phishing. Here the user is taught to distinguish phishing URLs from legitimate ones~\cite{sheng2007antiphishingphil, arachchilage2012designing}. Links to phishing websites are not only distributed by phishing e-mails. Such links can be spread via any communication channel. It is even possible to land on a phishing website by just surfing. Thus, for these cases knowing how to determine whether an e-mail is fake or legitimate is of no use. In these situations knowing how to distinguish phishing URLs from valid ones will help. The problem with this approach is that as soon as the DNS or host file is attacked, cf.~Section~\ref{s:phishing_techs}, even for experts it will get difficult to distinguish a phishing website from the legitimate one.
\end{description}

%HIER SCHON NACHTEILE ODER UNTEN BEI EXAMPLES ODER BEIDES?!


%============================================
\subsection{Medium Classification}
%============================================
The objective of this section is to introduce the different classes of learning media which we could identify in previous work.

\begin{description}[leftmargin=0cm]
	\item[Game Based Learning] One way to communicate the learning content to the user is to use the traditional game. Such a game usually has a ``background story'' and a ``mission'' the user has to accomplish~\cite{sheng2007antiphishingphil,antiphishingphyllis}. The game design is important and depends on the target group. Previous work, for example, has focused on a fish as starring role in their game, cf.~Section~\ref{s:prev_work}. This might work well for a target group of young age, but will most likely not be appealing to a larger audience. This is also reflected by our prestudy, cf.~Section~\ref{s:prestudy}.
	\item[Quiz Based Learning] The quiz based approach is a form of a game which relies on a question-answer cycle without using a specific background story~\cite{onguardonline}. The advantage of a quiz based approach is that it seems more appropriate for adults and thus will likely be appealing to a larger audience.
	\item[Comparison Based Learning] A further way to teach users is to let them compare legitimate websites, URLs or e-mails with fake ones. Here the user has to decide which of the shown examples are the secure ones~\cite{staysafeonline}. We believe that this form of learning would increase the user awareness, as with this approach one could visualize to the user how difficult it can be to distinguish an original from a fake, since they look almost identical. However, this way of learning does not reflect the reality, which is a major drawback in our point of view. In real life the user does not have the luxury of chosing between two options, he has only one and has to decide whether this option is trustful or not.
	\item[Emdedded Learning] The aim of embedded learning is to educate the user on the topic of phishing during his every day life. For this reason the user is sent simulated phishing e-mails. In case the user falls for this simulated phishing attempt he is notified and gets more information regarding phishing and how to protect himself~\cite{embedded2011jansson, kumaraguru2009phishguru}. This approach benefits from the so called ``teachable moment''. The moment the user realizes that he has almost become a victim to a phishing attack, he will be highly motivated to prevent this happening again and thus be highly receptive for input related to this topic. However, the missing positive feedback is a major flaw of this strategy. The user is only notified in case of a mistake and not in case he has successfully rejected to react to the simulated phishing e-mail. A further problem is raised with the realization of such an approach. Legal issues will arise when sending simulated phishing e-mails which claim to com from a reputable vendor, for example, Amazon.
\end{description}

%============================================
\subsection{Previous Work}
%============================================
\label{s:prev_work}
Previous work here ... (e.g. Anti-Phishing Phil and Phyllis)
%%%ANTI PHISHING LANDING PAGE ALS PREV WORK EXAMPLE EINBAUEN FÜR EMBEDDED LEARNING

%============================================
%\subsection{Open Questions}
%============================================
%\label{related_work:open_questions}
	%*******************************************
\section{Focus}
%*******************************************
\label{s:assumptions}
Introductory sentences...
Based on the discussion of the previous section we decided to...... (not only based on previous section)

%===========================================
\subsection{Coverage}
%===========================================
\begin{description}[marginleft=0]
	\item[Phishing Technique - Deceptive Phishing] move above text here
	\item[Attack Channel - E-Mail] or all channels with links to websites? move above text here
	\item[Variation of Phishing - Mass Phishing] move above text here
	\item[Game and Quiz Based Learning] wenn oben geschrieben, hier auch schreiben.			\item[URL Based Knowledge] The advantages of telling the user what to pay attention to within e-mails are the following: if the user recognizes the phishing e-mail before clicking on a link he does not even get onto a fake website where he could be lured to divulge his credentials. This also would mean, that the user would not be forwarded to a page where a malicious download might be started. On the other hand, these fake e-mails become more and more sophisticated and thus it becomes harder to distinguish them from legitimate ones~\cite{http://office.microsoft.com/en-001/outlook-help/identify-fraudulent-e-mail-and-phishing-schemes-HA001140002.aspx, http://www.spamfighter.com/News-18495-Becoming-More-Difficult-to-Detect-Phishing-Email-Attack-says-Security-Experts.htm}. Additionally, e-mail is not the only attack channel where links to phishing websites can be distributed.  Those links are also spread via instant messaging systems, online social networks, or SMS, where the messages would differ from those in e-mails. Moreover, phishing websites can also be reached by surfing~\cite{kaspersky report}, where the e-mail based knowledge approach would completely fail. For these reasons we decided to focus on communicating URL based knowledge to the user. This way, the disadvantages of e-mail based knowledge are mitigated. Furthermore, we believe that URL based knowledge gives the most reliable hint regarding its "belonging", i.e. whether a URL in fact belongs to a legitimate website or not.
	\item["After Click" URL Analysis] We have decided to consider the "after click" scenario for the following reasons: Firstly, we cannot hinder users from clicking on links and make them type in the whole URL into the address bar. This is too effortful, especially on smartphones, and thus will not be followed by them. Secondly, many links contain redirects. Such redirects are not recognizable before the click. A further problem the "before click" scenario raises is that the stock e-mail client of Android does not provide the functionality of viewing the destination URL before clicking on it. The only way to have a look at the URL before clicking on it is to make a long press onto the link, copy it into the clipboard, paste it somewhere else and then analyze it. Then, after the analysis the URL has to be sent to the browser. However, as this is also involves too much effort, no user will follow such a suggestion. Finally, even if there are many other e-mail clients which offer viewing the destination URL via long press only, we believe that this should not be communicated to the user for two reasons. Firstly, we do not know how many Android users actually make use of the stock e-mail client, which does not offer this functionality. Secondly, and most importantly, this functionality has the potential to mislead the user. A drawback of the URL destination preview is that the end of it is cut in case the URL is too long. Well-crafted URLs might thus look legitimate even though they are not because the most important part of the URL was cut out. For example, the subdomains of the URL can be long and well-crafted so that a legitimate looking subdomain is exactly at the end of the preview. This will cause the user think, that the subdomain at the end of the preview is the domain of the URL. Ultimately, the user will trust this URL even he should not. For the reasons explained above we have decided to consider the "after click" scenario. This approach suffers the disadvantage that users might click on a link which has a download of malware behind it.  For now, we consider this as future work, as there is no possibility to detect whether there is a download behind a URL before requesting the site.
%However, this drawback is mitigated by the fact that such downloaded malicious software is %not harmful as long as it is not opened and installed. Thus, the user should be told that %downloads he did not intend should immediately be deleted and should not be opened %under any circumstances.
% SCHÄDLICHE SW KANN SICH AUCH AUTOMATISCH AUSFÜRHEN / ÖFFNEN VON PDF IN BROWSER IST DANN AUCH SCHON GEFÄHRLICH
%Add to future work.
%Falls unverständlich vielleicht ein screenshot mit einem Beispiel ;) das mit well-crafted url
	\item[Considered Browser] für screenshots benutzen wir Android standard browser, aber kann auf jeden browser übertragen werden. wir haben überlegt, sahcen wie lock icon Blabla einzubauen, aber da immer sehr unterschiedlich haben wir uns dagegen entschieden. um unsere Methode möglichst allg. zu halten-- siehe bitte section blablabla
	
TODO: hinzufügen zu warum smartphone: •	Das Erlernte zu Smartphones kann einfach auf PCs übertragen werden. Umgekehrt schwieriger, weil erste Herausforderung: wie kommt man an die wichtigen Teile der URL.

//IDEE: UNSERE ZIELE/COVERAGE VON SCOPE UND BACKGROUND RAUSNEHMEN UND SCHIEBEN IN FOCUS!! OBEN SCHREIBEN: WIRD IN SECTION BLABLA BESCHRIEBEN WAS WIR COVERN UND WARUM. UND HIER NOCH DAZU PACKEN.


%===========================================
\subsection{System Requirements}
%===========================================
In the following we are listing the system requirements which need to be met for the final app.

\begin{description}[leftmargin=0cm]
	\item[Android] We have decided to develop an app for the Android operation system as we believe we have greater freedom here compared to an iOS app. The publication of an iOS app, for example, is connected with more requirements, which is not the case for Android apps~\cite{publishios, publishandroid}.
	\item[Version] Our original intention was to develop an Android app for version 4.0 and upward. However, during the app development we have encountered that about 24\% of all Android users still have Android 2.3.3 to 2.3.7~\cite{}. For this reason we have decided to modify the code so that these users can also install and use our app. 
	\item[Android Standard Browser] Android standard browser is kein system requirement - raus. muss irgendwoanders erwähnt werden. (was betrachten wir beim erklären oder so)...
\end{description}

%===========================================
\subsection{Assumptions}
%===========================================
\begin{description}
	\item[Secure DNS] ...
	\item[Secure Smartphone] ...
	\item[No Before-Click URL Analysis] ...
	\item[Download URLs Possible] ...
\end{description}

%===========================================
\subsection{Limitations of Our Approach}
%===========================================
\begin{description}
	\item[Cross-Site Scripting] ...
	\item[URL Hiding Techniques] ...
\end{description}


	%*******************************************
\section{Target Group}
%*******************************************
\label{s:target_group}
Introductory sentences...
DIVSI


	%*******************************************
\section{Pre-Survey}
%*******************************************
\label{s:prestudy}
Introductory sentences...

%============================================
\subsection{Main Objective}
%============================================

%============================================
\subsection{Survey Details}
%============================================
%============================================
\subsection{Evaluation}
%============================================
	%*******************************************
\section{Teaching and Learning Content}
%*******************************************
In this section we will describe and elaborate on different teaching and learning contents which can potentially be communicated to the user. At the same time we will reason our decision whether to communicate the specific content or not.
%Documents master_thesis/notes/android_browser bla -> BEGRÜNDUNG WARUM manches nicht sinnvoll ist (diese Sachen vielleicht eher in Appendix vor allem versionsunterschiede)
%Documents master_thesis/konzepte/android browser elemente UND browser comparison
%CHECK IF I FORGOT SOMETHING!!!!
%===========================================
\subsection{Phishing URLs}
%===========================================
Focus on distinguishing phishing URLs from legitimate ones.

%...........................................
\subsubsection{Phishing URL Categorization}
%...........................................

Potential phishing URL categories/phishing attcks on URLs
\begin{description}
	\item[Subdomain] covered
	\item[IP Address] covered
	\item[Nonsense Domain] covered
	\item[Trustworthy, But Unrelated Domain] covered
	\item[Similar and Deceptive Domains] covered Typo, Typosquatting (Buchstabendreher), Misspelling
	\item[Homographic Attack] covered (the type of homographic visible by user...)
	\item[Tiny URLs] Not covered
	\item[Cloaked URLs] Not covered - because redirect (use of @)
	\item[Encoding Tricks] Not covered - because redirect
\end{description}

%...........................................
\subsubsection{Problems and Challenges With The Categorization}
%...........................................

%...........................................
\subsection{Android Elements}
%...........................................

\begin{description}
	\item[Invisible Address Bar] Find URL Bar, Browser
	\item[Use of Https Within Websites] Browser
	\item[Analyze Complete URL Via Address Bar] Browser
	\item[Show URL Before Click] In E-Mail (not always possible), while surfing (long touch)
	\item[Copy and Paste URL] too much effort, additionally: redirects still possible
\end{description}

%...........................................
\subsection{Android Browser Security Indicators}
%...........................................

\begin{description}
		\item[Https Padlock] Browser
		\item[Displayed Webaddress on Https Sites] Browser
		\item[Certificate Verification]
		\item[Touch Padlock] to see whole URL.. problems: see document...
\end{description}

%...........................................
\subsection{E-Mail Spoofing}
%...........................................

\begin{description}
	\item{From Field} not trustworthy
	\item{E-Mail Content} in hand of attacker
	\item{Links in E-Mails} do not necessarily go where it claims to go (not only in e-mail links).
\end{description}
\subsubsection{General Recommended Behavior}
\begin{description}
	\item[Do Not Click]
	\item[Do Not Download Attachment]
	\item[Look at URL]
	\item[Data Economy]
	\item[Date Entry Via Https]
\end{description}

%...........................................
\subsection{Conclusion / Summary}
%...........................................

Summarize what to communicate to user here...


	
%*******************************************
\section{Approach for Our Anti-Phishing Education App}
%*******************************************
\label{s:approach}
This chapter presents our final approach for the Anti-Phishing Education App....
%===========================================
\subsection{App Design}
%===========================================
\begin{enumerate}
	\item Awareness Part
	\begin{enumerate}
		\item From is not from...
		\item Linktext unequal actual target URL
	\end{enumerate}
	\item Education Part
	\begin{enumerate}
		\item Information Material
		\item Exercise to Information Material
		\item Repeat 2.1 and 2.2 with increasing difficulty
	\end{enumerate}
\end{enumerate}

%===========================================
\subsection{Game Rules}
%===========================================

%===========================================
\subsection{Leveling Strategy}
%===========================================
Three approaches...

%===========================================
\subsection{Knowledge Transfer Per Level}
%===========================================
What is taught in each level ... 

%===========================================
\subsection{URL Generation}
%===========================================

%===========================================
\subsection{Gamification}
%===========================================
User motivation

\begin{description}
	\item[Show Leaderboard Rate]
	\item[Show Leaderboard Total]
	\item[...]
\end{description}

	%*******************************************
\section{Evaluation}
%*******************************************
\label{s:evaluation}
As a final step the Anti-Phishing Education we have designed and implemented needs to be evaluated which is the goal of this chapter. The app will be evaluated with the aid of a user study. After introducing our study design, we will state our hypothesis and explain how we are going to measure our statements in order to prove that they are true or false. Finally, we will analyze our results and state our conclusion.

%===========================================
%\subsection{Participant Recruitment}
%===========================================
%Oder was soll hier rein? ;)
%osn - freundes freunde
%telefon -freundes freunde
%flyer ausgeteilt
%flyer per e-mail an sehr viele professoren geschickt und gebeten weiter an studenten zu leiten
%ein paar leute auf der straße angesprochen - da erfolglos
%Verlosung eines amazin gutscheins


%===========================================
\subsection{Study Design}
%===========================================
For time reasons and lack of participants we decided to run a "Before and After App" Study with the same groups of people. Specifically, our user study is structured as follows:

\begin{enumerate}
	\item \textbf{General Before-Survey} At the beginning the participants have to fill out a general survey, where they have to judge their own knowledge on the topic of Internet security in general. For instance, they are asked whether it is easy for them to distinguish legitimate e-mails and websites from fake ones.
	\item \textbf{Website-Survery Before} In this part of the user study the participants gets a list of screenshots of websites. The screenshots had been taken with the standard browser of an Android tablet. In total, the user is shown 16 screenshots, with 8 phishing and 8 valid URLs. The user has to decide whether he would enter confidential data on the shown website. Additionally, he has to encircle the part of the screenshot which was the primary reason for his decision. Then, the user has to indicate how sure he was about his answers on a Likert scale. Finally, the user is asked whether he knows the vendor of the website and whether he has a account there.
	\item \textbf{Play App} After the "Website-Survey Before" the users get the smartphones in order to play the app. To save time, we skipped the introduction 2 part ("access address bar") for the user study. The user has half an hour to play the app. After half an hour they are asked to put the smartphones aside. Then, we collect the smartphones and note the reached points in each level.
	\item \textbf{Website-Survery After} After playing the app, the participants get a second website-survey. In this, all examples of the previous survey are included. Moreover, it contains 8 further website screenshots of which 4 have phishing and the remaining 4 have valid URLs.
	\item \textbf{General After-Survey} Finally, the participants are asked to fill out a form with questions to their demographics. This form does also contain questions related to the SUS and some other questions regarding their impression of the app.


\end{enumerate}

%===========================================
\subsection{Hypotheses}
%===========================================
In order to evaluate the effectiveness and usability of our app we have formulated the following hypotheses for the user study:

\begin{enumerate}
	\item \textbf{Hypothesis 1 - Mistakes} After playing the app, the users make significantly less mistakes in detecting phishing websites compared to before playing the app.
	\item \textbf{Hypothesis 2 - URL Based Decision} After playing the app, the users base their primary decision on whether a website is a phishing website or not significantly more often based on the URL compared to before playing the app.
	\item \textbf{Hypothesis 3 - URL Comprehension} After playing the app the user understands the importance of the second- and top-level domain of a URL as the only criteria to detect phishing websites.
	\item \textbf{Hypothesis 4 - Good Usability} The app is easy to understand and to use.
\end{enumerate}


%===========================================
\subsection{Measurement}
%===========================================
In the following we will elaborate on how we are going to measure the statements of our hypothesis and show that they are true or false.

\begin{enumerate}
	\item \textbf{Hypothesis 1 - Mistakes} Correct answers in "Website-Survery After" $>>$ correct answers in "Website-Survery Before" 
	\item \textbf{Hypothesis 2 - URL Based Decision} Number of URL markings in "Website-Survery After" $>>$ number of URL markings in "Website-Survery Before" 
	\item \textbf{Hypothesis 3 - URL Comprehension} Number of marked second- and/or top-level domains of URLs in "Website-Survery After"  $>>$ number of marked second- and/or top-level domains of URLs in "Website-Survery Before" 
	\item \textbf{Hypothesis 4 - Good Usability} System Usability Scale (SUS) $>$ 68
\end{enumerate}

%===========================================
\subsection{Results and Analysis}
%===========================================
%===========================================
\subsection{Discussion}
%===========================================
%===========================================
\subsection{Conclusion}
%===========================================

	%*******************************************
\section{Conclusion and Outlook}
%*******************************************
\label{s:conclusion}

This chapter provides a short summary of what we achieved in the scope of this thesis and presents an outlook on future work.


%===========================================
\subsection{Conclusion}
%===========================================

The objectives of this thesis... 

%===========================================
\subsection{Outlook}
%===========================================
This section deals with a prospect on future work for our Anti-Phishing Education App. In particular, we present ideas that might be beneficial and which we were not able to realize due to time and resource limitations.
	
	\bibliographystyle{abbrv} % <--- layout of the bib
	\bibliography{bibliography} % file name of your bib

\end{document}
